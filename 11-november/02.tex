% Options for packages loaded elsewhere
\PassOptionsToPackage{unicode}{hyperref}
\PassOptionsToPackage{hyphens}{url}
%
\documentclass[
]{article}
\usepackage{lmodern}
\usepackage{amssymb,amsmath}
\usepackage{ifxetex,ifluatex}
\ifnum 0\ifxetex 1\fi\ifluatex 1\fi=0 % if pdftex
  \usepackage[T1]{fontenc}
  \usepackage[utf8]{inputenc}
  \usepackage{textcomp} % provide euro and other symbols
\else % if luatex or xetex
  \usepackage{unicode-math}
  \defaultfontfeatures{Scale=MatchLowercase}
  \defaultfontfeatures[\rmfamily]{Ligatures=TeX,Scale=1}
\fi
% Use upquote if available, for straight quotes in verbatim environments
\IfFileExists{upquote.sty}{\usepackage{upquote}}{}
\IfFileExists{microtype.sty}{% use microtype if available
  \usepackage[]{microtype}
  \UseMicrotypeSet[protrusion]{basicmath} % disable protrusion for tt fonts
}{}
\makeatletter
\@ifundefined{KOMAClassName}{% if non-KOMA class
  \IfFileExists{parskip.sty}{%
    \usepackage{parskip}
  }{% else
    \setlength{\parindent}{0pt}
    \setlength{\parskip}{6pt plus 2pt minus 1pt}}
}{% if KOMA class
  \KOMAoptions{parskip=half}}
\makeatother
\usepackage{xcolor}
\IfFileExists{xurl.sty}{\usepackage{xurl}}{} % add URL line breaks if available
\IfFileExists{bookmark.sty}{\usepackage{bookmark}}{\usepackage{hyperref}}
\hypersetup{
  hidelinks,
  pdfcreator={LaTeX via pandoc}}
\urlstyle{same} % disable monospaced font for URLs
\usepackage{graphicx,grffile}
\makeatletter
\def\maxwidth{\ifdim\Gin@nat@width>\linewidth\linewidth\else\Gin@nat@width\fi}
\def\maxheight{\ifdim\Gin@nat@height>\textheight\textheight\else\Gin@nat@height\fi}
\makeatother
% Scale images if necessary, so that they will not overflow the page
% margins by default, and it is still possible to overwrite the defaults
% using explicit options in \includegraphics[width, height, ...]{}
\setkeys{Gin}{width=\maxwidth,height=\maxheight,keepaspectratio}
% Set default figure placement to htbp
\makeatletter
\def\fps@figure{htbp}
\makeatother
\usepackage[normalem]{ulem}
% Avoid problems with \sout in headers with hyperref
\pdfstringdefDisableCommands{\renewcommand{\sout}{}}
\setlength{\emergencystretch}{3em} % prevent overfull lines
\providecommand{\tightlist}{%
  \setlength{\itemsep}{0pt}\setlength{\parskip}{0pt}}
\setcounter{secnumdepth}{-\maxdimen} % remove section numbering

\date{}

\begin{document}

\hypertarget{header-n0}{%
\section{Nov 02 Sat}\label{header-n0}}

调休\ldots{}

\hypertarget{header-n3}{%
\subsection{SE-342::CV}\label{header-n3}}

How can a computer has its vision?

孙焱惊爆提问法

\hypertarget{header-n6}{%
\subsubsection{Split}\label{header-n6}}

我们希望能把一幅图简单地分成前景/背景两部分。

为了实现一个有效好用的算法,我们需要一些辅助知识。

\hypertarget{header-n9}{%
\paragraph{Geodesic Distance}\label{header-n9}}

所谓「侧壁距离」。就是在一个限定「可以经过的区域」内,两点之间的最短通路,就叫做
Geo. Dist.

\hypertarget{header-n11}{%
\paragraph{Geodesic Influence Zone (IZ)}\label{header-n11}}

IEEE 的定义:

The \textbf{geodesic influence zone} is A(B;) of a connected component
B; of B in A is the set of the points of A for which the
\textbf{geodesic} distance to B; is smaller than the \textbf{geodesic}
distance to other connected components of B.

\hypertarget{header-n14}{%
\paragraph{Skeleton by Influence Zone (SKIZ)}\label{header-n14}}

SKIZ 就是不属于任何一个 IZ
的点集合;直观上的感受就是:处在边缘上,跟谁都不挨着的算法。

\hypertarget{header-n16}{%
\subsubsection{Watershed-Catchment Basin}\label{header-n16}}

\hypertarget{header-n17}{%
\paragraph{How it works?}\label{header-n17}}

\begin{enumerate}
\def\labelenumi{\arabic{enumi}.}
\item
  把所有的像素进行排序,按照其 intensity(灰度级别)进行;
\item
  寻找亮度波形的局部最小值,作为起始点;(在不存在这样的点时,也可以手动指定一个起始点进行。)
\item
  将 Threshold 每每 + 1(涨水操作);
\item
  当出现第二个局部最小值的时候,将其加入,并且计算 SKIZ;
\item
  否则,就在已有的最小值里计算 SKIZ。
\item
  重复进行 3 \textasciitilde{}
  5,直到所有像素都有了它对应的大坝(Basin)或者是水位达到了最大的亮度级别。
\end{enumerate}

\hypertarget{header-n31}{%
\paragraph{Note}\label{header-n31}}

不同的起始点可能带来不同的分割结果。

\hypertarget{header-n33}{%
\paragraph{Thoughts}\label{header-n33}}

传统的分水岭分割方法,是一种基于拓扑理论的数学形态学的分割方法,其基本思想是把图像看作是测地学上的拓扑地貌,图像中每一像素的灰度值表示该点的海拔高度,每一个局部极小值及其影响区域称为集水盆地,而集水盆地的边界则形成分水岭。分水岭的概念和形成可以通过模拟浸入过程来说明。在每一个局部极小值表面,刺穿一个小孔,然后把整个模型慢慢浸人水中,随着浸入的加深,每一个局部极小值的影响域慢慢向外扩展,在两个集水盆汇合处构筑大坝如下图所示,即形成分水岭。

\begin{figure}
\centering
\includegraphics{http://www.images.studyai.com/blog/20171027194322350.png}
\caption{}
\end{figure}

水位不断上升,汇水盆地之间及其与背景之间的坝也越来越长。构筑水坝的目的是阻止盆地之间及其与背景之间的水汇聚。该过程一直持续,直到到达水的最高水位。最终水坝就是我们希望的分割结果。它的一条重要性质就是水坝组成一条连通的路径,这样,在两个区域之间就给出了连续的边界。

因为灰度变化较小的区域有较小的梯度值,因此,分水岭分割通常用于一幅图像的梯度,而不是图像本身。这样,汇水盆地的区域最小值就能很好地与待分割区域的较小梯度值联系起来。

\hypertarget{header-n38}{%
\paragraph{Pros \& Cons}\label{header-n38}}

优点是我们产生的「大坝」(Basin)一定是等高的;一定可以产生封闭的分割区域。

缺点是在存在很微弱而复杂的灰度级别变化时,会产生过分割(分割的区域数量过多过细)的现象。

\begin{figure}
\centering
\includegraphics{http://www.images.studyai.com/blog/20171027195320985.png}
\caption{}
\end{figure}

因此通常我们会先将图形进行一个平滑(Smoothing)之后再执行「分水岭算法」。

\hypertarget{header-n43}{%
\subsubsection{Edge Detection Kernel}\label{header-n43}}

目前公认最常用的边缘检测算子是
\href{https://zh.wikipedia.org/wiki/Canny算子}{Canny算子}(及其变体)。

除此之外,经典的边缘检测算子还有:

\hypertarget{header-n46}{%
\paragraph{Sobel 算子}\label{header-n46}}

其主要用于边缘检测,在技术上它是以离散型的差分算子,用来运算图像亮度函数的梯度的近似值,
Sobel
算子是典型的基于一阶导数的边缘检测算子,由于该算子中引入了类似局部平均的运算,因此对噪声具有平滑作用,能很好的消除噪声的影响。Sobel
算子对于象素的位置的影响做了加权,与 Prewitt 算子、Roberts
算子相比因此效果更好。

Sobel 算子包含两组 3×3
的矩阵,分别为横向及纵向模板,将之与图像作平面卷积,即可分别得出横向及纵向的亮度差分近似值。实际使用中,常用如下两个模板来检测图像边缘。

检测水平边沿 横向模板
:\(G_{x}=\begin{bmatrix} -1 & 0 & 1\\ -2 & 0 & 2\\ -1 & 0 & 1 \end{bmatrix}\)
检测垂直平边沿
纵向模板:\(G_{y}=\begin{bmatrix} 1 & 2 & 1\\ 0 & 0 & 0\\ -1 & -2 & -1 \end{bmatrix}\)

图像的每一个像素的横向及纵向梯度近似值可用以下的公式结合,来计算梯度的大小。

\(G=\sqrt[2]{G_{x}^{2}+G_{y}^{2}}\)

然后可用以下公式计算梯度方向。

\(\Theta =\arctan (\frac{G_{y}}{G_{x}})\)

在以上例子中,如果以上的角度Θ等于零,即代表图像该处拥有纵向边缘,左方较右方暗。

缺点是 Sobel 算子并没有将图像的主题与背景严格地区分开来,换言之就是
Sobel 算子并没有基于图像灰度进行处理,由于 Sobel
算子并没有严格地模拟人的视觉生理特征,所以提取的图像轮廓有时并不能令人满意。

\hypertarget{header-n58}{%
\paragraph{Isotropic Sobel 算子}\label{header-n58}}

Sobel 算子另一种形式是 (Isotropic Sobel)
算子,加权平均算子,权值反比于邻点与中心点的距离,当沿不同方向检测边缘时梯度幅度一致,就是通常所说的各向同性
Sobel(Isotropic Sobel) 算子。模板也有两个,一个是检测水平边沿的
,另一个是检测垂直平边沿的 。各向同性 Sobel 算子和普通 Sobel
算子相比,它的位置加权系数更为准确,在检测不同方向的边沿时梯度的幅度一致。

\hypertarget{header-n60}{%
\paragraph{Roberts 算子}\label{header-n60}}

罗伯茨算子、Roberts
算子是一种最简单的算子,是一种利用局部差分算子寻找边缘的算子,他采用对角线方向相邻两象素之差近似梯度幅值检测边缘。检测垂直边缘的效果好于斜向边缘,定位精度高,对噪声敏感,无法抑制噪声的影响。1963年,Roberts
提出了这种寻找边缘的算子。

Roberts 边缘算子是一个 2×2
的模板,采用的是对角方向相邻的两个像素之差。从图像处理的实际效果来看,边缘定位较准,对噪声敏感。适用于边缘明显且噪声较少的图像分割。Roberts
边缘检测算子是一种利用局部差分算子寻找边缘的算子,Robert
算子图像处理后结果边缘不是很平滑。经分析,由于 Robert
算子通常会在图像边缘附近的区域内产生较宽的响应,故采用上述算子检测的边缘图像常需做细化处理,边缘定位的精度不是很高。

\hypertarget{header-n63}{%
\paragraph{Prewitt 算子}\label{header-n63}}

Prewitt算子是一种一阶微分算子的边缘检测,利用像素点上下、左右邻点的灰度差,在边缘处达到极值检测边缘,去掉部分伪边缘,对噪声具有平滑作用
。其原理是在图像空间利用两个方向模板与图像进行邻域卷积来完成的,这两个方向模板一个检测水平边缘,一个检测垂直边缘。

对数字图像f(x,y),Prewitt 算子的定义如下:

G(i)=\textbar{[}f(i-1,j-1)+f(i-1,j)+f(i-1,j+1){]}-{[}f(i+1,j-1)+f(i+1,j)+f(i+1,j+1){]}\textbar{}

G(j)=\textbar{[}f(i-1,j+1)+f(i,j+1)+f(i+1,j+1){]}-{[}f(i-1,j-1)+f(i,j-1)+f(i+1,j-1){]}\textbar{}

则 P(i,j)=max{[}G(i),G(j){]}或 P(i,j)=G(i)+G(j)

经典 Prewitt
算子认为:凡灰度新值大于或等于阈值的像素点都是边缘点。即选择适当的阈值T,若P(i,j)≥T,则(i,j)为边缘点,P(i,j)为边缘图像。这种判定是欠合理的,会造成边缘点的误判,因为许多噪声点的灰度值也很大,而且对于幅值较小的边缘点,其边缘反而丢失了。

Prewitt
算子对噪声有抑制作用,抑制噪声的原理是通过像素平均,但是像素平均相当于对图像的低通滤波,所以
Prewitt 算子对边缘的定位不如 Roberts 算子。

因为平均能减少或消除噪声,Prewitt
梯度算子法就是先求平均,再求差分来求梯度。水平和垂直梯度模板分别为:

检测水平边沿横向模板
\(G_{x}=\begin{bmatrix} -1 & 0 & 1\\ -1 & 0 & 1\\ -1 & 0 & 1 \end{bmatrix}\)
检测垂直平边沿纵向模板:\(G_{y}=\begin{bmatrix} 1 & 1 & 1\\ 0 & 0 & 0\\ -1 & -1 & -1 \end{bmatrix}\)

该算子与 Sobel
算子类似,只是权值有所变化,但两者实现起来功能还是有差距的,据经验得知
Sobel 要比 Prewitt 更能准确检测图像边缘。

\hypertarget{header-n74}{%
\paragraph{LoG 算子}\label{header-n74}}

Laplacian of Gaussian Operator

以二阶导数为基础。

\hypertarget{header-n77}{%
\paragraph{边缘检测是什么?}\label{header-n77}}

是一种操作过程,它可以从一张图片中提取出有意义的构成边缘的像素。

所谓边缘,就是图像的颜色、亮度、材质、反射系数等等信息发生改变的明显边界。

但这是一个非常主观的概念;怎么去找出一种形式化的方法来计算它呢?

\hypertarget{header-n81}{%
\paragraph{边缘是什么?}\label{header-n81}}

越问越玄乎\ldots{}

边缘分很多类:

突变型(突然 A 跳到 B)、缓变型(A 平滑地变化到 B)、跳变型(A 变化到 B
之后又回到 A)。

可以直接对其求微分(一阶甚至二阶),观察其变化率来确定边缘。

或者也可以用卷积,采用合适的、特别用来检测边缘的卷积核就可以做到。

\hypertarget{header-n87}{%
\subsubsection{Observations}\label{header-n87}}

\hypertarget{header-n88}{%
\paragraph{Kernel 看起来如何?}\label{header-n88}}

我们来观察一下上面用到的一般卷积核。

我们所用的 Edge Detection 算子,存在相反的符号,而且和为 0;

(这确保了在平坦的区域,求得的卷积接近为 0。)

而反过来,在值变化较大的区域,就会因此而被激发出(绝对值)\textbf{较大}的卷积值。

\hypertarget{header-n93}{%
\paragraph{先 Smooth 一下怎么样?}\label{header-n93}}

为了保证我们不会因为噪音而意外得出过多的破碎片,我们会对整个图像进行平滑(Smoothing)。

另外,先平滑再求导,或者是先求导再平滑都一样(效果上)。没有任何问题。

\hypertarget{header-n96}{%
\paragraph{到了二维的情况下怎么办?}\label{header-n96}}

\hypertarget{header-n97}{%
\subparagraph{偏微分}\label{header-n97}}

留意到在跳开一维的情况之后,我们就不能简单地做一个「微分」了。要考虑一个方向的问题。

所以扩展地,我们可以用「梯度」这个概念;也就是在 x、y
两个方向上的偏微分。

\hypertarget{header-n100}{%
\subparagraph{幅值}\label{header-n100}}

从梯度中还能提取出我们更关心的信息:幅值及其方向:通过 x、y
两个方向上的偏微分我们可以知道某一点 (x, y)
处变化率最大的方向,在这个方向上其变化率是多少。

\hypertarget{header-n102}{%
\subparagraph{Laplacian 变换}\label{header-n102}}

对一个二维图形来说,做 Laplacian of Gaussian 操作就是对 x、y
两个方向上求其二阶导数之和。

但 LoG 用的很少,主要是下面两项原因:

一来他对噪音非常敏感,很容易产生误判断;二来他是\textbf{各向同性}的一个算法,无法检测出边缘的方向。

但总归可以大略地判断出边缘所在的位置;再结合其他方法来检测边缘。

\hypertarget{header-n107}{%
\subsubsection{Canny Edge Detector}\label{header-n107}}

世界上最好的边缘检测算法(打死)

Canny 不仅给出了一个很不错的边缘检测算法;

Canny 还给出了一套判断「边缘检测好不好」的评判标准。

自从 John F. Canny 1980s 提出这一算子以来,非常流行。

\hypertarget{header-n112}{%
\paragraph{评判标准}\label{header-n112}}

\begin{itemize}
\item
  Low error rate?
\end{itemize}

有没有错过某些边缘?有没有错误加入某些边缘?

\begin{itemize}
\item
  Well localized?
\end{itemize}

检测到的边缘位置准确吗?

\begin{itemize}
\item
  Single edge point response?
\end{itemize}

能否把构成边缘的点集合最小化,乃至单点化?

\hypertarget{header-n125}{%
\paragraph{操作步骤}\label{header-n125}}

\begin{enumerate}
\def\labelenumi{\arabic{enumi}.}
\item
  老规矩,先做一次 Smoothing 再说
\item
  计算出每一个像素的梯度及其方向
\item
  压制那些幅值非最大的梯度(使用 Non-maxima Supression,见下)
\item
  跟踪轮廓,通过预指定的较大阈值得到一个较粗的轮廓
\end{enumerate}

\begin{quote}
这一步所得到的仅仅是候补点。
\end{quote}

\begin{enumerate}
\def\labelenumi{\arabic{enumi}.}
\item
  仅仅保留这些集合中共同的像素(强边缘),而移除其他那些较弱的边缘点(弱边缘)
\end{enumerate}

\hypertarget{header-n140}{%
\paragraph{Non-maxima Supression}\label{header-n140}}

非最大值压制法。对一定长度线段上的点,我们检查每个点的梯度的幅值。仅仅其中最大的那个值有权加入候补边缘点集合;其他的都无权加入。

\hypertarget{header-n142}{%
\paragraph{双阈值法}\label{header-n142}}

为了保留那些相对较弱的边缘,阈值并非一成不变的;在那些候选点较多的位置上,会调整为较高的那个阈值;而在候选点稀疏时,则调整为较低的阈值进行工作。

这样,可以保证明显的边缘不会过粗,且不太明显的边缘也不至于消失。

留意:不同的图像对阈值非常敏感。不同的图像一般需要重新定参;这是後话。

\hypertarget{header-n146}{%
\paragraph{Edge Linking}\label{header-n146}}

对那些因为阈值过大而被切断成很多段的边缘线,我们采用 Edge Linking
算法来进行「破碎线段的拟合」。

这个问题大到值得新开一个大节。

\hypertarget{header-n149}{%
\subsubsection{Edge Linking}\label{header-n149}}

\hypertarget{header-n150}{%
\paragraph{Hough Transform}\label{header-n150}}

霍夫变换:专门用来实现 Edge Tracking / Linking 的算法。

不仅可以 Link 直线,还可以 Link 其他更多线型!

以下所有的东西都是基于 Hough Transform 的。

\hypertarget{header-n154}{%
\paragraph{Main Idea}\label{header-n154}}

主要的思路是:用一组参数模型来表示一组特征。

\hypertarget{header-n156}{%
\subparagraph{Line}\label{header-n156}}

从最简单的直线开始。

问题来了:

\begin{itemize}
\item
  如何能判断出哪些点组成的是同一条直线,还是说他们只是碰巧 k / b 相同?
\item
  有一些微弱的 k/b 不相同,能否把他们归为一条直线?阈值取多少合适?
\item
  通常线段的两端都存在一些随机噪声。怎么处理这些噪声点?
\end{itemize}

\hypertarget{header-n166}{%
\paragraph{Voting 机制}\label{header-n166}}

\sout{CPC 绝不走老路歪路邪路}

为了确定这些点最可能对应出哪一种图像元,主要还是采取投票机制。

谁得到的票数最多,就采取哪一种图形元进行拟合。

\hypertarget{header-n170}{%
\paragraph{Hough Transform (HT)}\label{header-n170}}

基本原理就是点和线的对偶性;一条直线在参数空间(k - b
图像)中对应的是一个点;

更有意思的是,图像中的一个点在参数空间中对应的是一条直线

(无数条经过该点的直线的 k/b 在 k - b 图像中的投影)

因此我们把参数空间里的复杂的一切东西都对应到 ``k/b''
参数空间中的点事宜;这样就方便很多了。

要判断两个点是否可能属于同一条直线,只需要看他们在 k-b
图像中对应的两条直线是否有交点就知道了。

更多的点也一样;变化成求直线交点的问题。多条直线交汇处称为「Cluster」。

所谓「原始图像」跟「k - b 图像」之间的变换就称作 Hough's
Transformation。

\hypertarget{header-n178}{%
\paragraph{Corner Cases}\label{header-n178}}

上面说的「一一对应」其实是句假话。

留意到 k - b 图像无法表示 k = ∞ 的直线(即和 y 轴垂直的直线)。

因此我们还是采用直线的极坐标表示法来表示;

这样对应到 r - \(\theta\) 平面上是真·一一对应;没有无法表示的 Corner
Case。

\hypertarget{header-n183}{%
\paragraph{Fitting}\label{header-n183}}

在投票完成之后,就可以开始拟合模型了。

方法很多,用最小二乘法之类的就足够了。

\hypertarget{header-n186}{%
\subsubsection{Questions}\label{header-n186}}

\begin{itemize}
\item
  三种边缘检测算法
\item
  三种噪音消除算法
\item
  Canny 是怎么一回事
\end{itemize}

宫商角(jué)徵(zhǐ)羽

\end{document}
