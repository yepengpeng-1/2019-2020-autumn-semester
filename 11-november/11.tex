% Options for packages loaded elsewhere
\PassOptionsToPackage{unicode}{hyperref}
\PassOptionsToPackage{hyphens}{url}
%
\documentclass[
]{article}
\usepackage{lmodern}
\usepackage{amssymb,amsmath}
\usepackage{ifxetex,ifluatex}
\ifnum 0\ifxetex 1\fi\ifluatex 1\fi=0 % if pdftex
  \usepackage[T1]{fontenc}
  \usepackage[utf8]{inputenc}
  \usepackage{textcomp} % provide euro and other symbols
\else % if luatex or xetex
  \usepackage{unicode-math}
  \defaultfontfeatures{Scale=MatchLowercase}
  \defaultfontfeatures[\rmfamily]{Ligatures=TeX,Scale=1}
\fi
% Use upquote if available, for straight quotes in verbatim environments
\IfFileExists{upquote.sty}{\usepackage{upquote}}{}
\IfFileExists{microtype.sty}{% use microtype if available
  \usepackage[]{microtype}
  \UseMicrotypeSet[protrusion]{basicmath} % disable protrusion for tt fonts
}{}
\makeatletter
\@ifundefined{KOMAClassName}{% if non-KOMA class
  \IfFileExists{parskip.sty}{%
    \usepackage{parskip}
  }{% else
    \setlength{\parindent}{0pt}
    \setlength{\parskip}{6pt plus 2pt minus 1pt}}
}{% if KOMA class
  \KOMAoptions{parskip=half}}
\makeatother
\usepackage{xcolor}
\IfFileExists{xurl.sty}{\usepackage{xurl}}{} % add URL line breaks if available
\IfFileExists{bookmark.sty}{\usepackage{bookmark}}{\usepackage{hyperref}}
\hypersetup{
  hidelinks,
  pdfcreator={LaTeX via pandoc}}
\urlstyle{same} % disable monospaced font for URLs
\usepackage[normalem]{ulem}
% Avoid problems with \sout in headers with hyperref
\pdfstringdefDisableCommands{\renewcommand{\sout}{}}
\setlength{\emergencystretch}{3em} % prevent overfull lines
\providecommand{\tightlist}{%
  \setlength{\itemsep}{0pt}\setlength{\parskip}{0pt}}
\setcounter{secnumdepth}{-\maxdimen} % remove section numbering

\date{}

\begin{document}

\hypertarget{header-n0}{%
\section{Nov 11 Mon}\label{header-n0}}

\hypertarget{header-n2}{%
\subsection{SE-342::CV}\label{header-n2}}

Well done. 今天要講的內容是「CV::Texture」。

\hypertarget{header-n4}{%
\subsubsection{紋理}\label{header-n4}}

何謂紋理(Texture)?如何給出一個精確的定義?

\begin{quote}
說明:直到這門課結束我們都給不出一個精確的定義。
\end{quote}

但我們大概可以給一個主觀的定義。

\hypertarget{header-n9}{%
\subsubsection{紋理貼圖}\label{header-n9}}

我們可以將一些可以大致表徵某種特定材質的紋理圖片「貼」到幾何體表面,使用較少的代價來實現真實物體的模擬。

Wiki
說:\textbf{材質貼圖},又稱\textbf{紋理貼圖},在\href{https://zh.wikipedia.org/wiki/计算机图形学}{電腦圖學}中是把儲存在記憶體里的\href{https://zh.wikipedia.org/wiki/位图}{點陣圖}包裹到
3D
彩現物體的表面。紋理貼圖給物體提供了豐富的細節,用簡單的方式類比出了複雜的外觀。一個圖像(紋理)被貼(對映)到場景中的一個簡單形體上,就像印花貼到一個平面上一樣。這大大減少了在場景中製作形體和紋理的計算量。例如,可以建立一個球並把臉的紋理貼上去,這樣就不用處理鼻子和眼睛的形狀了。

紋理貼圖的確定和觀測尺度有關係;不同的尺度下應當採用不同的貼圖才能大概地還原原有圖像。

紋理沒有一個確切的定義;具有類似結構的小微元按照一定的規律重複出現,就構成了一種紋理。

\hypertarget{header-n14}{%
\paragraph{Structural}\label{header-n14}}

結構化來講,可以分析出每個小細節的結構,將紋理整體看作是一系列的基元(Texels,
Texture Elements)構成的。

\hypertarget{header-n16}{%
\paragraph{Statistical}\label{header-n16}}

統計上來說,可以從紋理整理的角度來研究一些問題,如紋理整體的灰度級別,整體的亮度分佈等問題。

\hypertarget{header-n19}{%
\subsubsection{紋理應用}\label{header-n19}}

\begin{itemize}
\item
  紋理匹配
\item
  紋理分割
\item
  紋理合成
\item
  紋理分析
\end{itemize}

\hypertarget{header-n29}{%
\paragraph{Texture Comparison / Analysis}\label{header-n29}}

有一些算法可以對不同類型的紋理進行比較,判斷其是否屬於同一種紋理。思考一下有沒有什麼方法來判斷紋理之間的「SIMILAR」或是「DIFFERENT」?

例如,在成像圖片中尋找肝硬化的蹤跡,利用的就是紋理分析。

\hypertarget{header-n32}{%
\paragraph{Filters}\label{header-n32}}

通過濾波器可以分析出紋理材質的特性。

通常我們設計的濾波器是把 Spots(孤立點)和 Oriented
Bars(有向條狀物)作為標的物的。

\hypertarget{header-n35}{%
\paragraph{Filter Window}\label{header-n35}}

留意到,濾波窗口很重要,應當使用多種不同方向、形狀和尺寸的窗口來進行濾波,確定最合適的那一種。

習慣上來說,一般採用兩個點狀濾波器(一大一小),六個條狀濾波器,在 0
\textasciitilde{} 90 度之間,每隔 18
度安排一條。通常來說這樣的濾波檢查結果可以大致地覆蓋全部的需求,又不太佔用資源。

\hypertarget{header-n38}{%
\paragraph{Gabor Filters}\label{header-n38}}

Gabor 濾波器僅僅提一句。可以滿足不同的尺度、不同的方向的紋理檢測。

分為非對稱濾波器和對稱濾波器兩種。

\hypertarget{header-n41}{%
\paragraph{Fourier Transformation}\label{header-n41}}

這個在實際生產中用的也很多的。就是「傅立葉變換」而已。

\hypertarget{header-n43}{%
\paragraph{Finalé Texture Representation}\label{header-n43}}

構造一組有向的、不同尺度的圖像濾波器;

\hypertarget{header-n45}{%
\paragraph{Texture Synthesis}\label{header-n45}}

所謂紋理合成,要做的事情是給定很小的一塊紋理貼圖,我們利用這一塊輸入生成大塊自相似的大塊紋理圖。即為
Texture
Synthesis(紋理合成)。並且,儘量不要展示出不和諧的「接縫」、「邊緣」、「鋸齒」等效果。

當然,那些要規避的「接縫」、「鋸齒」等等東西都跟觀察距離有關。離得太近總歸會看到一些細節,這就是要求過分了。

\hypertarget{header-n48}{%
\subsubsection{Local Binary Patterns}\label{header-n48}}

紋理特徵提取算子。

對每一個 Pixel,創建一個 8-bit \sout{guy} 的數字 b1 ~ b8。每一位的 0 或
1 取決於八個相鄰像素跟當前像素的大小關係。大於時取 1;小於等於時取 0。

這樣就得到了一個特殊的算子。用來進行紋理特徵提取很好用。

\end{document}
