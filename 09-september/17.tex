% Options for packages loaded elsewhere
\PassOptionsToPackage{unicode}{hyperref}
\PassOptionsToPackage{hyphens}{url}
%
\documentclass[
]{article}
\usepackage{lmodern}
\usepackage{amssymb,amsmath}
\usepackage{ifxetex,ifluatex}
\ifnum 0\ifxetex 1\fi\ifluatex 1\fi=0 % if pdftex
  \usepackage[T1]{fontenc}
  \usepackage[utf8]{inputenc}
  \usepackage{textcomp} % provide euro and other symbols
\else % if luatex or xetex
  \usepackage{unicode-math}
  \defaultfontfeatures{Scale=MatchLowercase}
  \defaultfontfeatures[\rmfamily]{Ligatures=TeX,Scale=1}
\fi
% Use upquote if available, for straight quotes in verbatim environments
\IfFileExists{upquote.sty}{\usepackage{upquote}}{}
\IfFileExists{microtype.sty}{% use microtype if available
  \usepackage[]{microtype}
  \UseMicrotypeSet[protrusion]{basicmath} % disable protrusion for tt fonts
}{}
\makeatletter
\@ifundefined{KOMAClassName}{% if non-KOMA class
  \IfFileExists{parskip.sty}{%
    \usepackage{parskip}
  }{% else
    \setlength{\parindent}{0pt}
    \setlength{\parskip}{6pt plus 2pt minus 1pt}}
}{% if KOMA class
  \KOMAoptions{parskip=half}}
\makeatother
\usepackage{xcolor}
\IfFileExists{xurl.sty}{\usepackage{xurl}}{} % add URL line breaks if available
\IfFileExists{bookmark.sty}{\usepackage{bookmark}}{\usepackage{hyperref}}
\hypersetup{
  hidelinks,
  pdfcreator={LaTeX via pandoc}}
\urlstyle{same} % disable monospaced font for URLs
\usepackage{color}
\usepackage{fancyvrb}
\newcommand{\VerbBar}{|}
\newcommand{\VERB}{\Verb[commandchars=\\\{\}]}
\DefineVerbatimEnvironment{Highlighting}{Verbatim}{commandchars=\\\{\}}
% Add ',fontsize=\small' for more characters per line
\newenvironment{Shaded}{}{}
\newcommand{\AlertTok}[1]{\textcolor[rgb]{1.00,0.00,0.00}{\textbf{#1}}}
\newcommand{\AnnotationTok}[1]{\textcolor[rgb]{0.38,0.63,0.69}{\textbf{\textit{#1}}}}
\newcommand{\AttributeTok}[1]{\textcolor[rgb]{0.49,0.56,0.16}{#1}}
\newcommand{\BaseNTok}[1]{\textcolor[rgb]{0.25,0.63,0.44}{#1}}
\newcommand{\BuiltInTok}[1]{#1}
\newcommand{\CharTok}[1]{\textcolor[rgb]{0.25,0.44,0.63}{#1}}
\newcommand{\CommentTok}[1]{\textcolor[rgb]{0.38,0.63,0.69}{\textit{#1}}}
\newcommand{\CommentVarTok}[1]{\textcolor[rgb]{0.38,0.63,0.69}{\textbf{\textit{#1}}}}
\newcommand{\ConstantTok}[1]{\textcolor[rgb]{0.53,0.00,0.00}{#1}}
\newcommand{\ControlFlowTok}[1]{\textcolor[rgb]{0.00,0.44,0.13}{\textbf{#1}}}
\newcommand{\DataTypeTok}[1]{\textcolor[rgb]{0.56,0.13,0.00}{#1}}
\newcommand{\DecValTok}[1]{\textcolor[rgb]{0.25,0.63,0.44}{#1}}
\newcommand{\DocumentationTok}[1]{\textcolor[rgb]{0.73,0.13,0.13}{\textit{#1}}}
\newcommand{\ErrorTok}[1]{\textcolor[rgb]{1.00,0.00,0.00}{\textbf{#1}}}
\newcommand{\ExtensionTok}[1]{#1}
\newcommand{\FloatTok}[1]{\textcolor[rgb]{0.25,0.63,0.44}{#1}}
\newcommand{\FunctionTok}[1]{\textcolor[rgb]{0.02,0.16,0.49}{#1}}
\newcommand{\ImportTok}[1]{#1}
\newcommand{\InformationTok}[1]{\textcolor[rgb]{0.38,0.63,0.69}{\textbf{\textit{#1}}}}
\newcommand{\KeywordTok}[1]{\textcolor[rgb]{0.00,0.44,0.13}{\textbf{#1}}}
\newcommand{\NormalTok}[1]{#1}
\newcommand{\OperatorTok}[1]{\textcolor[rgb]{0.40,0.40,0.40}{#1}}
\newcommand{\OtherTok}[1]{\textcolor[rgb]{0.00,0.44,0.13}{#1}}
\newcommand{\PreprocessorTok}[1]{\textcolor[rgb]{0.74,0.48,0.00}{#1}}
\newcommand{\RegionMarkerTok}[1]{#1}
\newcommand{\SpecialCharTok}[1]{\textcolor[rgb]{0.25,0.44,0.63}{#1}}
\newcommand{\SpecialStringTok}[1]{\textcolor[rgb]{0.73,0.40,0.53}{#1}}
\newcommand{\StringTok}[1]{\textcolor[rgb]{0.25,0.44,0.63}{#1}}
\newcommand{\VariableTok}[1]{\textcolor[rgb]{0.10,0.09,0.49}{#1}}
\newcommand{\VerbatimStringTok}[1]{\textcolor[rgb]{0.25,0.44,0.63}{#1}}
\newcommand{\WarningTok}[1]{\textcolor[rgb]{0.38,0.63,0.69}{\textbf{\textit{#1}}}}
\usepackage{longtable,booktabs}
% Correct order of tables after \paragraph or \subparagraph
\usepackage{etoolbox}
\makeatletter
\patchcmd\longtable{\par}{\if@noskipsec\mbox{}\fi\par}{}{}
\makeatother
% Allow footnotes in longtable head/foot
\IfFileExists{footnotehyper.sty}{\usepackage{footnotehyper}}{\usepackage{footnote}}
\makesavenoteenv{longtable}
\usepackage{graphicx,grffile}
\makeatletter
\def\maxwidth{\ifdim\Gin@nat@width>\linewidth\linewidth\else\Gin@nat@width\fi}
\def\maxheight{\ifdim\Gin@nat@height>\textheight\textheight\else\Gin@nat@height\fi}
\makeatother
% Scale images if necessary, so that they will not overflow the page
% margins by default, and it is still possible to overwrite the defaults
% using explicit options in \includegraphics[width, height, ...]{}
\setkeys{Gin}{width=\maxwidth,height=\maxheight,keepaspectratio}
% Set default figure placement to htbp
\makeatletter
\def\fps@figure{htbp}
\makeatother
\setlength{\emergencystretch}{3em} % prevent overfull lines
\providecommand{\tightlist}{%
  \setlength{\itemsep}{0pt}\setlength{\parskip}{0pt}}
\setcounter{secnumdepth}{-\maxdimen} % remove section numbering

\date{}

\begin{document}

\hypertarget{header-n0}{%
\section{2019/09/17}\label{header-n0}}

\hypertarget{header-n2}{%
\subsection{SE-302}\label{header-n2}}

现在我们开始讨论如何实现上节课中提到的「分割」、「识别」。

首先,我们给出一个字串 S,且已知一个正则表达式 R。我们总得先实现如何判断
S 是否属于 R 的 Closure。

这个该如何实现(Implementation)呢?

需要用到一个东西:

\hypertarget{header-n7}{%
\subsubsection{Finite Automata(有限自动机)}\label{header-n7}}

有限自动机(自动机里面简单的那种)

之前提过,一个 RegEx 就是一个 Specification,描述性的。

而这里的 Automata,是实现自动机的一种途径。

一个有限自动机,一个五元组: -\/-\/- 三个重要元素

\begin{enumerate}
\def\labelenumi{\arabic{enumi}.}
\item
  输入的一个字母表(Alphabet Σ),说明了输入一定是这些字母构成的
\item
  一组状态(States S),状态之间可以互相转移(Transition)。
  如果把这些状态画成顶点圆,状态之间转移的可能性看作边箭头,那你就有了一个有向图。
\item
  一个初始状态 n。
  在图中表示为一个圆圈,上面有一个不从任何状态指向他的箭头。 -\/-\/-
  两个辅助元素
\item
  一组可接受的状态 F (是 S 的子集),在达到 F 的时候停止。
  在上面的有向图中,终态画成双圈。
\item
  状态之间转移的可能性,即 Transitions;对应到图里的边。
\end{enumerate}

有一些输入会导致状态机卡死,在某一点出无处可去。

遇到这种情况,我们就说这些输入不属于这个有限状态机。

\hypertarget{header-n25}{%
\subsubsection{存在不可确定的有限自动机}\label{header-n25}}

(中文书 P15)
有一些自动机的定义中,从单个状态中存在多种相同符号的转移方式。也可能存在一些标志有ε的边,这种边的转换不需要吃进任何字符,就可以完成状态的转换。

这种不完全决定于输入的有限自动机叫做 Non-Deterministic Finite
Automana,简称为 NFA。

而上一小节里面的,不存在这种奇怪情况的自动机叫做 Deterministic Finite
Automana,简称为 DFA。

区分它们的区别。

\begin{quote}
另外,从理论上来说,我们的图灵自动机需要的内存是无限的;而在这里的
Finite Automata 需要的内存理论上是有限的。只需要编码当前状态就够了。
\end{quote}

\hypertarget{header-n32}{%
\subsubsection{区分 NFA / DFA}\label{header-n32}}

DFA:Deterministic。要么走不动,要么一条路走到黑。不存在什么随机状况。

但对 NFA:有一些地方的走法(Transaction)不唯一,即存在 Multiple
Choice(多种走法且都合法),甚至有的地方还能不读取字符直接走(Epsilon
Move)。

当然,不是每种走法都可以到达终点。

但是对 NFA 来说,只要有一种情况可以到达 Final State 就算 Accept 了。

通常来说,NFA 的有向图表示比 DFA 要简单。

理论上 NFA 跟 DFA 能力相同,都可以识别 RegEx。

但事实上 DFA 容易实现(因为没有 Non-Deterministic
的分支,方便写代码),可惜难以构造(难以用 DFA 来写出 RegEx 的表达)

而 NFA 刚好相反。用 NFA 可以很方便表示出 RegEx 的形式,但是难以去实现
NFA 本身。

所以我们讨巧一些,先用 NFA 表示 RegEx 的规则,再将其转换为 DFA
来实现。如何?

\hypertarget{header-n42}{%
\subsubsection{大象放进冰箱三部曲}\label{header-n42}}

\begin{enumerate}
\def\labelenumi{\arabic{enumi}.}
\item
  正则表达式转换为 NFA;
\item
  NFA 转换为 DFA;
\item
  验证正则表达式,直接查 DFA 构造出的表就可以了。
\end{enumerate}

\hypertarget{header-n50}{%
\paragraph{第一步:如何把正则表达式转换为 NFA?}\label{header-n50}}

\begin{figure}
\centering
\includegraphics{https://raw.githubusercontent.com/yuetsin/private-image-repo/master/2019/09/17-08-34-49-RegEx.to.NFA.png}
\caption{}
\end{figure}

Example:

\texttt{(1\textbar{}0)*1} 这个 RegEx 如何?

\hypertarget{header-n55}{%
\paragraph{第二步:NFA =\textgreater{} DFA?}\label{header-n55}}

对于那些不确定转换方向的节点,我们选择迭代地搜索出 T
所可能到达的所有状态,我们叫它 edge(边缘);另外,还有那些 只经过 ε
边,不接受任何字符所能到达的顶点。我们称之为 closure(闭包)。

我们现在从 Start
开始寻找。寻找什么呢?寻找那些同样字符对应不同转换的情况,以及 epsilon
不吃输入就转换的情况,我们统统把他们构成一整个状态,确保不再出现一个输入字符不能确定的状态。注意,同一个编号的状态现在可能在多处出现;一个大状态可能包含多个不同编号的「原状态」。总之现在,仅仅根据
Input Char 来决断分类。最后看起来应该像这样:

\begin{figure}
\centering
\includegraphics{https://raw.githubusercontent.com/yuetsin/private-image-repo/master/2019/09/17-08-58-46-dfa.shape.png}
\caption{}
\end{figure}

然后,对那些不定方向的 T,我们就能构造出 DFA了。算法参见中文书 P19:

\begin{figure}
\centering
\includegraphics{https://raw.githubusercontent.com/yuetsin/private-image-repo/master/2019/09/17-08-53-32-dfa.code.png}
\caption{}
\end{figure}

注意,这里抛弃了那些不可到达的边。由于 DFAedge/Closure
保证了我们不会去访问那些 Inaccessible
的边,所以防止了我们的解释器指数爆炸。否则的话,所有状态全部可以随机组合,数量太爆炸了。

但是,最多的状态也只会有 2\^{}n - 1
种,从而转换的方向也是有限种(虽然也很大了)。

到底是有限的。

DFA 可以用一种神奇的表来表示:

\begin{longtable}[]{@{}lll@{}}
\toprule
& 0 & 1\tabularnewline
\midrule
\endhead
S & T & U\tabularnewline
T & T & U\tabularnewline
U & T & U\tabularnewline
\bottomrule
\end{longtable}

读法是:先看行,以(S,T,U)为例,说的是从 S 点开始,吃到一个 0
字符可以前往 T,吃到一个 1 字符可以前往 U。其他的同理。

注意,只有 DFA 才能用这种方式表示。NFA 嘛\ldots{} epsilon
先不谈,首先一个状态吃一个同字符可以前往不同的地方。这就没办法用这种表格来表示了吧。

\hypertarget{header-n84}{%
\paragraph{工具}\label{header-n84}}

NFA -\textgreater{} DFA 的过程可以使用 Flex/Jlex
完成。计算机可以帮忙完成这个任务,因为是非常机械的一个操作。最好变成上面那张表的样子,那样我就更好写了。

P.S. DFA 通常比 NFA
更大,而且不好阅读(废话一堆状态组合在一起成大状态怎么会好读)

\hypertarget{header-n87}{%
\paragraph{Tweaks:DFA 太大怎么办?}\label{header-n87}}

划分等价类方法。首先分成两部分;Accepting State/Non-Accepting State。

也就是 Final State 跟 Non-Final State。

终结状态跟非终结状态肯定不可能等价(don't ask wsm

于是我们在 Final State Sets 中和 Non-Final State Sets
中分别比较,观察他们所受到的转化是否有不一样。注意,我们要看的只有输出,看到达这个状态之后,相同的吃入字符能否产生一样的结果,而不考虑指向他的状态是否一致(没有必要,因为如果他们被证明等价,我们就直接让他们指向一样的位置就好了)。

所以到底,我们可以直接比较上面那张表里的行。行一致就等价。

\hypertarget{header-n93}{%
\subsubsection{最后一步}\label{header-n93}}

我们只需要把 Regular Expression 用 DFA 实现就行了。

\hypertarget{header-n95}{%
\paragraph{回忆:RegEx 三件套}\label{header-n95}}

\begin{itemize}
\item
  最长匹配:尽可能长地匹配字串
\item
  优先级匹配:同时可以匹配两种规则,按优先级先后匹配
\item
  兜底匹配:遇到 error 不要结束,先兜着,往后看
\end{itemize}

正则表达式扫描符号:

⟙:Current State:当前位置

⟘:Last State:回退位置

\textbar:String Begin:当前串的起始位置,用\textbar 来标记。

为了实现贪婪匹配,需要同时维持 Current State 和 Last State。Current
State 往前走来判断是否死,Last State
用来记录上一个稳妥的可回退位置。String Begin
是整个串的开头。三个值组合在一起,这样才能实现圆满的 Max Munch。

达成一次匹配之后,我们将当前串的起始位置设定为 Last State,Current State
亦需要回退;然后将 DFA 放到初始状态,relsama houwy 重新开始吧

\hypertarget{header-n109}{%
\subsubsection{myth:优先级如何匹配?}\label{header-n109}}

我们回忆一下,NFA -\textgreater{} DFA
之后,很多原来的状态都会被组合在一起形成大状态。

那么这些状态本身的优先级该如何处理?

Answer:删掉全部低优先级的。(没人权)

\hypertarget{header-n113}{%
\subsubsection{Lex - 词法分析的生成器}\label{header-n113}}

Lex 能被 Compile \& Link 成一个 .c 程序,作用是一个词法分析器。

\begin{Shaded}
\begin{Highlighting}[]
\PreprocessorTok{#define ADJ (EM_tokPos=charPos, charPos+=yyleng)}
\end{Highlighting}
\end{Shaded}

这类似于一个 Macro,做的操作是调节上述三个 position pointer。

这还是需要写一点点 C的。

Lex 里面定义好的变量:

\begin{itemize}
\item
  \texttt{yylval}
\item
  \texttt{yyleng},长度
\item
  \texttt{yytext},token 的内容,\texttt{char\ *} 型
\end{itemize}

然后我们只需要简单写点 RegEx \& Actions 就好了。

\begin{figure}
\centering
\includegraphics{https://raw.githubusercontent.com/yuetsin/private-image-repo/master/2019/09/17-09-41-58-lex.standard.png}
\caption{}
\end{figure}

\hypertarget{header-n129}{%
\subsection{SE-227}\label{header-n129}}

计算机系统工程,啊。

CSE,CSE,系统工程 \texttt{of} 计算机

\hypertarget{header-n132}{%
\paragraph{Q\&A:}\label{header-n132}}

\begin{itemize}
\item
  What's file name? Data or Metadata?

  \begin{itemize}
  \item
    Neither. 它不是文件的一部分。
  \end{itemize}
\item
  What's the actual content of a directory (文件夹 or 目录)? What's its
  size?

  \begin{itemize}
  \item
    (大小仅仅由其中包含文件的数量有关。跟文件实际的大小无关。)
  \item
    (翻译成目录比文件夹更好。真的就是书的目录啊!)
  \end{itemize}
\item
  Is a directory data or metadata?

  \begin{itemize}
  \item
    (从文件的角度出发,it's data,目录就是个普通的文件。)
  \item
    (从更高的文件系统的角度出发,目录,即文件的组织结构就是
    metadata。)
  \end{itemize}
\end{itemize}

\hypertarget{header-n153}{%
\subsubsection{Directory Entry}\label{header-n153}}

\begin{Shaded}
\begin{Highlighting}[]
\CommentTok{// <ext4.h>}

\KeywordTok{struct}\NormalTok{ ext4_dir_entry \{}
\NormalTok{    __le32 	inode;}
    \CommentTok{// uint32_t}
    
\NormalTok{    __le16 	rec_len;}
    \CommentTok{// uint16_t}
    
\NormalTok{    __le16 	name_len;}
    \CommentTok{// uint16_t}
    
    \DataTypeTok{char}\NormalTok{ 	name [EXT4_NAME_LEN];}
\NormalTok{\}}


\CommentTok{//    File Type:}
\CommentTok{//    0x0 - Unknown}
\CommentTok{//    0x1 - Regular File}
\CommentTok{//    0x2 - Directory}
\CommentTok{//    0x3 - Character device file}
\CommentTok{//    0x4 - Block device file}
\CommentTok{//    0x5 - FIFO}
\CommentTok{//    0x6 - Socket}
\CommentTok{//    0x7 - Symbolic link}
\end{Highlighting}
\end{Shaded}

\hypertarget{header-n155}{%
\subsubsection{Two Approaches to Linking}\label{header-n155}}

\hypertarget{header-n156}{%
\paragraph{Hard Link}\label{header-n156}}

没有新文件被建立。仅仅是新增了一个 Struct 指向一个已经存在的 inode。

需要维护一个 \texttt{ref\_cnt} 来确定 inode 现在被多少次引用。

\begin{itemize}
\item
  Hard Link 可以用来实现快捷方式(但应该没什么人用这个吧)
\item
  Hard Link \emph{不能}给不同用户不同的权限(因为这是写在 inode 里的)
\item
  实现 . 和 .. (指向自己和上级目录)
\end{itemize}

事实上这个没什么用

\hypertarget{header-n167}{%
\paragraph{Soft Link}\label{header-n167}}

有新文件被建立,新文件就是 Soft Link
itself。大小非常小,只和原身文件路径的长度有关。

不需要维护
\texttt{ref\_cnt},删除本体,替身无法打开但不会消失。删除替身也不会影响原身。

更类似于我们的快捷方式/替身。

\hypertarget{header-n171}{%
\subsubsection{File System APIs}\label{header-n171}}

\texttt{A\_Pi}

我们来实际地看看这些 API 吧。不仅关注用法,还关注实现。

\hypertarget{header-n174}{%
\paragraph{chdir}\label{header-n174}}

change directory / cd:要实现它,必须找到这个目录或者爆一个不存在错

\hypertarget{header-n176}{%
\paragraph{mkdir}\label{header-n176}}

跟创建普通文件一样,只不过 type = 2 (DIRECTORY)

\hypertarget{header-n178}{%
\paragraph{creat}\label{header-n178}}

创建一个普通文件

\hypertarget{header-n180}{%
\paragraph{link}\label{header-n180}}

把一个存在的文件给链接上

\hypertarget{header-n182}{%
\paragraph{unlink}\label{header-n182}}

把一个文件从目录中拿走(取消链接)。如果 refcnt == 0,那就把文件也干掉。

\hypertarget{header-n184}{%
\paragraph{rename}\label{header-n184}}

重命名三步走:

unlink creat link

\hypertarget{header-n187}{%
\paragraph{open. read. write. close.}\label{header-n187}}

C 的 API,但是这些 API 不施加于目录上。这层抽象还是要有的!

\hypertarget{header-n189}{%
\paragraph{mount / unmount}\label{header-n189}}

挂载/卸载可移动介质。

\hypertarget{header-n191}{%
\subsubsection{Question \#2}\label{header-n191}}

如果我们挂载一个 \texttt{FAT} 格式的移动介质,不存在 inode
的情况下我们是如何对其进行索引的?

事实上 FAT 的分层做的是不怎么好的。

这里,OS 帮大忙了。

\texttt{vnode}:Virtualized iNode,软件虚拟出来的一种 inode。

就算实际上你使用的是 FAT 存储,根本没有 inode,

我也可以给你虚拟出一个 Virtual 的 inode,

让 API Caller 可以用上那些 inode 的 API。这些都是以 System Call
的形式调用的。

\hypertarget{header-n199}{%
\subsubsection{\texorpdfstring{\texttt{open}
or\texttt{fopen}?}{open orfopen?}}\label{header-n199}}

\texttt{open} 返回的是一个 \texttt{fd},而 \texttt{fopen} 返回的是
\texttt{FILE\ *}。

(\texttt{open} 是 Linux (Unix) 系统提供的 System Call,平台相关)

(\texttt{FILE\ *}其实也是个 libc 提供的宏展开表达式,平台无关)

另外实际上\ldots{}\texttt{fopen} 要比 \texttt{open}
更快一些。有缓存,有优化。非常棒。

\hypertarget{header-n204}{%
\subsubsection{三个 inode 时间戳}\label{header-n204}}

\begin{itemize}
\item
  atime:上次读取(READ)
\item
  mtime:上次修改(WRITE)
\item
  ctime:上次 inode 的修改(LINK)
\end{itemize}

\end{document}
