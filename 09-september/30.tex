% Options for packages loaded elsewhere
\PassOptionsToPackage{unicode}{hyperref}
\PassOptionsToPackage{hyphens}{url}
%
\documentclass[
]{article}
\usepackage{lmodern}
\usepackage{amssymb,amsmath}
\usepackage{ifxetex,ifluatex}
\ifnum 0\ifxetex 1\fi\ifluatex 1\fi=0 % if pdftex
  \usepackage[T1]{fontenc}
  \usepackage[utf8]{inputenc}
  \usepackage{textcomp} % provide euro and other symbols
\else % if luatex or xetex
  \usepackage{unicode-math}
  \defaultfontfeatures{Scale=MatchLowercase}
  \defaultfontfeatures[\rmfamily]{Ligatures=TeX,Scale=1}
\fi
% Use upquote if available, for straight quotes in verbatim environments
\IfFileExists{upquote.sty}{\usepackage{upquote}}{}
\IfFileExists{microtype.sty}{% use microtype if available
  \usepackage[]{microtype}
  \UseMicrotypeSet[protrusion]{basicmath} % disable protrusion for tt fonts
}{}
\makeatletter
\@ifundefined{KOMAClassName}{% if non-KOMA class
  \IfFileExists{parskip.sty}{%
    \usepackage{parskip}
  }{% else
    \setlength{\parindent}{0pt}
    \setlength{\parskip}{6pt plus 2pt minus 1pt}}
}{% if KOMA class
  \KOMAoptions{parskip=half}}
\makeatother
\usepackage{xcolor}
\IfFileExists{xurl.sty}{\usepackage{xurl}}{} % add URL line breaks if available
\IfFileExists{bookmark.sty}{\usepackage{bookmark}}{\usepackage{hyperref}}
\hypersetup{
  hidelinks,
  pdfcreator={LaTeX via pandoc}}
\urlstyle{same} % disable monospaced font for URLs
\usepackage{graphicx,grffile}
\makeatletter
\def\maxwidth{\ifdim\Gin@nat@width>\linewidth\linewidth\else\Gin@nat@width\fi}
\def\maxheight{\ifdim\Gin@nat@height>\textheight\textheight\else\Gin@nat@height\fi}
\makeatother
% Scale images if necessary, so that they will not overflow the page
% margins by default, and it is still possible to overwrite the defaults
% using explicit options in \includegraphics[width, height, ...]{}
\setkeys{Gin}{width=\maxwidth,height=\maxheight,keepaspectratio}
% Set default figure placement to htbp
\makeatletter
\def\fps@figure{htbp}
\makeatother
\setlength{\emergencystretch}{3em} % prevent overfull lines
\providecommand{\tightlist}{%
  \setlength{\itemsep}{0pt}\setlength{\parskip}{0pt}}
\setcounter{secnumdepth}{-\maxdimen} % remove section numbering

\date{}

\begin{document}

\hypertarget{header-n0}{%
\section{Sep 30 Mon}\label{header-n0}}

\hypertarget{header-n2}{%
\subsection{SE-342}\label{header-n2}}

本节课不讲特别重要的知识。

\begin{quote}
(反正大家也不会听)
\end{quote}

主要讲一下彩色图像。

\begin{quote}
(因为这部分内容今后的 Project 中或是应用之中不是特别重要的一部分。)
\end{quote}

\hypertarget{header-n9}{%
\subsubsection{彩色是如何产生的?}\label{header-n9}}

\hypertarget{header-n10}{%
\paragraph{Additive}\label{header-n10}}

在现实世界的颜色混合中,色彩的混合是取交集的

(因此所有颜色的颜料混在一起会得到黑色);

这个唤做 Additive 混色系统(\texttt{原色を重ね合わせる加色混合法の})。

\hypertarget{header-n14}{%
\paragraph{Subtractive}\label{header-n14}}

而在 CRT、液晶显示等光线状态中,色彩的混合是取并集的

(因此所有颜色的光线混在一起会得到白色)。

此为 Subtractive 混色系统。

\hypertarget{header-n18}{%
\subsubsection{颜色的表示}\label{header-n18}}

\hypertarget{header-n19}{%
\paragraph{RGB}\label{header-n19}}

Red + Green + Blue,采用的是一个正方体来表示颜色空间。

\hypertarget{header-n21}{%
\paragraph{HSV/HSL}\label{header-n21}}

HSL 即色相、饱和度、亮度(英语:Hue, Saturation, Lightness)。

HSV即色相、饱和度、明度(英语:Hue, Saturation, Value)。

又称HSB,其中B即英语:Brightness。
色相(H)是色彩的基本属性,就是平常所说的颜色名称,如红色、黄色等。
饱和度(S)是指色彩的纯度,越高色彩越纯,低则逐渐变灰,取0-100\%的数值。
明度(V),亮度(L),取0-100\%。

\begin{figure}
\centering
\includegraphics{https://raw.githubusercontent.com/yuetsin/private-image-repo/master/2019/09/30-14-13-03-800px-Hsl-hsv_models.svg.png}
\caption{}
\end{figure}

因此这个看起来是一个圆柱体。

\hypertarget{header-n27}{%
\paragraph{Conversion}\label{header-n27}}

这两种模型应该如何转换呢?

所谓转换,也就是HSL和HSV在数学上定义为在RGB空间中的颜色的\emph{R},\emph{G}和\emph{B}的坐标的变换。

\hypertarget{header-n30}{%
\subparagraph{RGB =\textgreater{} HSL/HSV}\label{header-n30}}

\begin{figure}
\centering
\includegraphics{https://raw.githubusercontent.com/yuetsin/private-image-repo/master/2019/09/30-14-14-52-rgb-\textgreater{}hsl:hsv.png}
\caption{}
\end{figure}

\begin{figure}
\centering
\includegraphics{https://raw.githubusercontent.com/yuetsin/private-image-repo/master/2019/09/30-14-15-21-rgb-\textgreater{}hsv.png}
\caption{}
\end{figure}

本质上是将这个正方体给立起来,变成一个类似于圆柱体的东西。然后再把它给变形成类似于圆柱体的样子就好了。

\hypertarget{header-n34}{%
\subparagraph{HSL =\textgreater{} RGB}\label{header-n34}}

\begin{figure}
\centering
\includegraphics{https://raw.githubusercontent.com/yuetsin/private-image-repo/master/2019/09/30-14-17-43-hsl-\textgreater{}rgb.png}
\caption{}
\end{figure}

色度在不同的范围之内,变化是有细微的不同的。

\hypertarget{header-n38}{%
\subsubsection{颜色均衡化}\label{header-n38}}

选择一个亮区、再选择一个暗区;将他们进行颜色拉伸,直到他们与第三区的颜色匹配,就算
OK。

通常来说不会只针对一个颜色空间进行颜色均衡化。一般会在 RGB、CMYK、HSL
等多个颜色空间中都进行颜色均衡化,使得在各个空间中的色度都很均衡。就很棒。

\hypertarget{header-n41}{%
\subsubsection{颜色恢复}\label{header-n41}}

给黑白图像重新上色,应该用什么算法?

首先在 RGB 下上色,

然后回到 HSL 下针对色度和饱和度进行中值滤波,

然后对亮度进行高斯滤波,

是一个效果比较好的方案。

针对 Hue 和 Saturation
中值滤波可以保证颜色倾向性的正确,不出现异常的诡色。

\hypertarget{header-n48}{%
\subsubsection{伪彩色图像}\label{header-n48}}

Pseudocolor Color Image

把颜色跟灰度级别(或其他参数)对应,生成假的彩色图像(本身没有彩色信息)。

RGB 值都由其他参数决定。图像本身没有颜色。

主要是方便理解和观测。

\hypertarget{header-n53}{%
\subsubsection{颜色参数}\label{header-n53}}

我们提出了更多的一些彩 色图 像信息:

\begin{itemize}
\item
  Average hue
\item
  Average Saturation
\item
  Average Value
\end{itemize}

\hypertarget{header-n62}{%
\subsubsection{颜色「PERCEPTION」(感觉)}\label{header-n62}}

人对颜色的感受在亮度变化的情况下会产生很大的差别。

在颜色突然变化时,或是前景背景颜色差异出现时,也会导致瞬间目标颜色感知差异。

Example::machBandsEffect

改变虹膜开口大小也可以调节进入眼睛的光线强度。

不同的感光细胞的敏感性也会随着颜色的强度改变。

\hypertarget{header-n68}{%
\subsubsection{White Balance}\label{header-n68}}

白平衡

一般来说,为了避免颜色倾向性偏差,我们一般会在保证整体的颜色没有偏离真实值太多。

\hypertarget{header-n71}{%
\section{世界和平。}\label{header-n71}}

\begin{figure}
\centering
\includegraphics{https://raw.githubusercontent.com/yuetsin/private-image-repo/master/2019/09/30-14-20-09-world peace.jpg}
\caption{}
\end{figure}

\end{document}
