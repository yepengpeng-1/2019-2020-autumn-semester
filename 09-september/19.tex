% Options for packages loaded elsewhere
\PassOptionsToPackage{unicode}{hyperref}
\PassOptionsToPackage{hyphens}{url}
%
\documentclass[
]{article}
\usepackage{lmodern}
\usepackage{amssymb,amsmath}
\usepackage{ifxetex,ifluatex}
\ifnum 0\ifxetex 1\fi\ifluatex 1\fi=0 % if pdftex
  \usepackage[T1]{fontenc}
  \usepackage[utf8]{inputenc}
  \usepackage{textcomp} % provide euro and other symbols
\else % if luatex or xetex
  \usepackage{unicode-math}
  \defaultfontfeatures{Scale=MatchLowercase}
  \defaultfontfeatures[\rmfamily]{Ligatures=TeX,Scale=1}
\fi
% Use upquote if available, for straight quotes in verbatim environments
\IfFileExists{upquote.sty}{\usepackage{upquote}}{}
\IfFileExists{microtype.sty}{% use microtype if available
  \usepackage[]{microtype}
  \UseMicrotypeSet[protrusion]{basicmath} % disable protrusion for tt fonts
}{}
\makeatletter
\@ifundefined{KOMAClassName}{% if non-KOMA class
  \IfFileExists{parskip.sty}{%
    \usepackage{parskip}
  }{% else
    \setlength{\parindent}{0pt}
    \setlength{\parskip}{6pt plus 2pt minus 1pt}}
}{% if KOMA class
  \KOMAoptions{parskip=half}}
\makeatother
\usepackage{xcolor}
\IfFileExists{xurl.sty}{\usepackage{xurl}}{} % add URL line breaks if available
\IfFileExists{bookmark.sty}{\usepackage{bookmark}}{\usepackage{hyperref}}
\hypersetup{
  hidelinks,
  pdfcreator={LaTeX via pandoc}}
\urlstyle{same} % disable monospaced font for URLs
\usepackage{color}
\usepackage{fancyvrb}
\newcommand{\VerbBar}{|}
\newcommand{\VERB}{\Verb[commandchars=\\\{\}]}
\DefineVerbatimEnvironment{Highlighting}{Verbatim}{commandchars=\\\{\}}
% Add ',fontsize=\small' for more characters per line
\newenvironment{Shaded}{}{}
\newcommand{\AlertTok}[1]{\textcolor[rgb]{1.00,0.00,0.00}{\textbf{#1}}}
\newcommand{\AnnotationTok}[1]{\textcolor[rgb]{0.38,0.63,0.69}{\textbf{\textit{#1}}}}
\newcommand{\AttributeTok}[1]{\textcolor[rgb]{0.49,0.56,0.16}{#1}}
\newcommand{\BaseNTok}[1]{\textcolor[rgb]{0.25,0.63,0.44}{#1}}
\newcommand{\BuiltInTok}[1]{#1}
\newcommand{\CharTok}[1]{\textcolor[rgb]{0.25,0.44,0.63}{#1}}
\newcommand{\CommentTok}[1]{\textcolor[rgb]{0.38,0.63,0.69}{\textit{#1}}}
\newcommand{\CommentVarTok}[1]{\textcolor[rgb]{0.38,0.63,0.69}{\textbf{\textit{#1}}}}
\newcommand{\ConstantTok}[1]{\textcolor[rgb]{0.53,0.00,0.00}{#1}}
\newcommand{\ControlFlowTok}[1]{\textcolor[rgb]{0.00,0.44,0.13}{\textbf{#1}}}
\newcommand{\DataTypeTok}[1]{\textcolor[rgb]{0.56,0.13,0.00}{#1}}
\newcommand{\DecValTok}[1]{\textcolor[rgb]{0.25,0.63,0.44}{#1}}
\newcommand{\DocumentationTok}[1]{\textcolor[rgb]{0.73,0.13,0.13}{\textit{#1}}}
\newcommand{\ErrorTok}[1]{\textcolor[rgb]{1.00,0.00,0.00}{\textbf{#1}}}
\newcommand{\ExtensionTok}[1]{#1}
\newcommand{\FloatTok}[1]{\textcolor[rgb]{0.25,0.63,0.44}{#1}}
\newcommand{\FunctionTok}[1]{\textcolor[rgb]{0.02,0.16,0.49}{#1}}
\newcommand{\ImportTok}[1]{#1}
\newcommand{\InformationTok}[1]{\textcolor[rgb]{0.38,0.63,0.69}{\textbf{\textit{#1}}}}
\newcommand{\KeywordTok}[1]{\textcolor[rgb]{0.00,0.44,0.13}{\textbf{#1}}}
\newcommand{\NormalTok}[1]{#1}
\newcommand{\OperatorTok}[1]{\textcolor[rgb]{0.40,0.40,0.40}{#1}}
\newcommand{\OtherTok}[1]{\textcolor[rgb]{0.00,0.44,0.13}{#1}}
\newcommand{\PreprocessorTok}[1]{\textcolor[rgb]{0.74,0.48,0.00}{#1}}
\newcommand{\RegionMarkerTok}[1]{#1}
\newcommand{\SpecialCharTok}[1]{\textcolor[rgb]{0.25,0.44,0.63}{#1}}
\newcommand{\SpecialStringTok}[1]{\textcolor[rgb]{0.73,0.40,0.53}{#1}}
\newcommand{\StringTok}[1]{\textcolor[rgb]{0.25,0.44,0.63}{#1}}
\newcommand{\VariableTok}[1]{\textcolor[rgb]{0.10,0.09,0.49}{#1}}
\newcommand{\VerbatimStringTok}[1]{\textcolor[rgb]{0.25,0.44,0.63}{#1}}
\newcommand{\WarningTok}[1]{\textcolor[rgb]{0.38,0.63,0.69}{\textbf{\textit{#1}}}}
\setlength{\emergencystretch}{3em} % prevent overfull lines
\providecommand{\tightlist}{%
  \setlength{\itemsep}{0pt}\setlength{\parskip}{0pt}}
\setcounter{secnumdepth}{-\maxdimen} % remove section numbering

\date{}

\begin{document}

\hypertarget{header-n0}{%
\section{2019/09/19}\label{header-n0}}

\hypertarget{header-n2}{%
\subsection{SE-344}\label{header-n2}}

\begin{quote}
计算机图形学
\end{quote}

\hypertarget{header-n5}{%
\subsubsection{理论性基础}\label{header-n5}}

主要讲一些计算机图形学是啥?研究些什么?跟几何学的渊源?发展和应用?

\hypertarget{header-n7}{%
\subsubsection{Definition}\label{header-n7}}

\texttt{CG} -\textgreater{} \texttt{Computer\ Graphics}

用计算机来生成、处理、显示「图形」。

\hypertarget{header-n10}{%
\subsubsection{Problems}\label{header-n10}}

主要问题:模型、动画、渲染。

\begin{itemize}
\item
  Modeling
\item
  Animating
\item
  Rendering
\end{itemize}

\hypertarget{header-n19}{%
\subsubsection{Solutions}\label{header-n19}}

CG 的研究范畴

\begin{itemize}
\item
  Physical Simulation (物理仿真)
\item
  Physical Hardware \& Software (图形软硬件)

  \begin{itemize}
  \item
    硬件部分

    \begin{itemize}
    \item
      数据采集和测量
    \end{itemize}

    \begin{quote}
    从真实世界中通过图形的方式来获取数据
    \end{quote}

    \begin{itemize}
    \item
      传感控制
    \end{itemize}

    \begin{quote}
    通过上面获取的数据来干预真实世界
    \end{quote}
  \item
    软件部分

    \begin{itemize}
    \item
      矢量绘图软件
    \end{itemize}

    \begin{quote}
    绘图软件不用多说,实在太多了。
    \end{quote}

    \begin{quote}
    矢量图指的是用顶点和曲线等几何信息来描述的图形,放大不走形
    \end{quote}

    \begin{itemize}
    \item
      3D 建模
    \end{itemize}

    \begin{quote}
    3ds max, Maya, etc.
    \end{quote}
  \item
    Graphics Algorithms (图形算法)

    算法就是上面各种应用程序的底层支撑了。

    包括矢量图形处理、渲染基础、动画制作云云。
  \end{itemize}
\item
  Graphics Standard (图形标准)
\end{itemize}

标准化是非常要紧的一件事情。

\hypertarget{header-n60}{%
\subsubsection{生成}\label{header-n60}}

\hypertarget{header-n61}{%
\subsubsection{处理}\label{header-n61}}

\hypertarget{header-n62}{%
\subsubsection{显示}\label{header-n62}}

\hypertarget{header-n63}{%
\subsection{SE-302}\label{header-n63}}

Compilers' Notes

\hypertarget{header-n65}{%
\subsection{Back to Lex}\label{header-n65}}

\begin{Shaded}
\begin{Highlighting}[]
\BaseNTok{%\{}

\BaseNTok{%\}}
\end{Highlighting}
\end{Shaded}

在这两个符号之间,我们会潜入一些类似于 C Macro 的东西进去。例如
\#include、\#define 以及 union 和 struct 的定义。

回忆一下上次里面出现的一些 magic variables:

\begin{itemize}
\item
  yylval: 匹配到的字符串所对应的 semantic value (语义值)
\item
  yytext: 匹配到的字符串值 (未进行语义解读)
\end{itemize}

我们的目的就是通过上面所提到的自动机来把我们的输入 String slice 并
categorize 到不同的类型里,并试图解读出其中的 semantic value
(语义值)。

我们提供给 Lex 的东西类似于之前的 \textless Token, Lexeme\textgreater{}
Pair。

观察程序 2-1,我们可以看到提供的一些 RegEx:

\begin{itemize}
\item
  \texttt{if}, 对应着 if
\item
  \texttt{{[}a-z{]}{[}a-z0-9{]}*}, starts with a-z, contains 0-9a-z. ID
  type.
\item
  \texttt{\{digits\}}, integer typed numeric value.
\item
  \texttt{(\{digits\}"."{[}0-9{]}*)\textbar{}({[}0-9{]}*"."\{digits\}},
  float typed numeric value.
\item
  \texttt{("-\/-"{[}a-z{]}*"\textbackslash{}n"\textbar{}("\ "\textbar{}"\textbackslash{}n"\textbar{}"\textbackslash{}t"))+},
  comments
\item
  最后一个永远不要忘记是 Error Token。用来兜底。
\end{itemize}

每次识别了一个 Token,就需要让 char pointer 往后移动一位。

这个操作定义为 \#define ADJ 宏。

ADJ 宏 =\textgreater{}
\texttt{EM\_tokPos\ =\ charPos,\ charPos\ +=\ yyleng}

\begin{itemize}
\item
  EM\_tokPos: 上次找到的 token 的 position,更新 token 位置,
\item
  charPos: 马上要去扫描的字符串的起始位置。
\end{itemize}

charPos 是自己定义的 Helper,EM\_tokPos 是 Lex 维护的。毕竟他负责解析
Tokens。

解析完了他知道每个 token 有多长,他是用 yyleng 来记录的。

注意在 macro 阶段我们定义了一种 yylval 类型,它既可以是 int 也可以是
string 或 double。他们共同占有一个 struct,用 union 来组合在一起。 union
别忘了。

\end{document}
