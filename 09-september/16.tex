% Options for packages loaded elsewhere
\PassOptionsToPackage{unicode}{hyperref}
\PassOptionsToPackage{hyphens}{url}
%
\documentclass[
]{article}
\usepackage{lmodern}
\usepackage{amssymb,amsmath}
\usepackage{ifxetex,ifluatex}
\ifnum 0\ifxetex 1\fi\ifluatex 1\fi=0 % if pdftex
  \usepackage[T1]{fontenc}
  \usepackage[utf8]{inputenc}
  \usepackage{textcomp} % provide euro and other symbols
\else % if luatex or xetex
  \usepackage{unicode-math}
  \defaultfontfeatures{Scale=MatchLowercase}
  \defaultfontfeatures[\rmfamily]{Ligatures=TeX,Scale=1}
\fi
% Use upquote if available, for straight quotes in verbatim environments
\IfFileExists{upquote.sty}{\usepackage{upquote}}{}
\IfFileExists{microtype.sty}{% use microtype if available
  \usepackage[]{microtype}
  \UseMicrotypeSet[protrusion]{basicmath} % disable protrusion for tt fonts
}{}
\makeatletter
\@ifundefined{KOMAClassName}{% if non-KOMA class
  \IfFileExists{parskip.sty}{%
    \usepackage{parskip}
  }{% else
    \setlength{\parindent}{0pt}
    \setlength{\parskip}{6pt plus 2pt minus 1pt}}
}{% if KOMA class
  \KOMAoptions{parskip=half}}
\makeatother
\usepackage{xcolor}
\IfFileExists{xurl.sty}{\usepackage{xurl}}{} % add URL line breaks if available
\IfFileExists{bookmark.sty}{\usepackage{bookmark}}{\usepackage{hyperref}}
\hypersetup{
  hidelinks,
  pdfcreator={LaTeX via pandoc}}
\urlstyle{same} % disable monospaced font for URLs
\usepackage{graphicx,grffile}
\makeatletter
\def\maxwidth{\ifdim\Gin@nat@width>\linewidth\linewidth\else\Gin@nat@width\fi}
\def\maxheight{\ifdim\Gin@nat@height>\textheight\textheight\else\Gin@nat@height\fi}
\makeatother
% Scale images if necessary, so that they will not overflow the page
% margins by default, and it is still possible to overwrite the defaults
% using explicit options in \includegraphics[width, height, ...]{}
\setkeys{Gin}{width=\maxwidth,height=\maxheight,keepaspectratio}
% Set default figure placement to htbp
\makeatletter
\def\fps@figure{htbp}
\makeatother
\setlength{\emergencystretch}{3em} % prevent overfull lines
\providecommand{\tightlist}{%
  \setlength{\itemsep}{0pt}\setlength{\parskip}{0pt}}
\setcounter{secnumdepth}{-\maxdimen} % remove section numbering

\date{}

\begin{document}

\hypertarget{header-n0}{%
\section{2019/09/16}\label{header-n0}}

\hypertarget{header-n2}{%
\subsection{SE-342}\label{header-n2}}

今天的课程内容:

\hypertarget{header-n4}{%
\subsection{Basic Image Operations}\label{header-n4}}

讲一些基础的图像操作。

主要分三类:

\hypertarget{header-n7}{%
\subsubsection{点运算 (Point Operation)}\label{header-n7}}

逐点运算,对每一个像素进行操作。

输入图像记做 I(x, y),输出图像记做 O(x, y)。

每个点操作的映射关系是 f(x, y) -\textgreater{} g(x, y)

每一个点的值只跟输入的对应那点的值有关,这个就叫做「点运算」。

\hypertarget{header-n12}{%
\paragraph{Examples}\label{header-n12}}

\begin{itemize}
\item
  调节亮度/对比度(Brightness / Saturation)
\item
  黑白化
\end{itemize}

公式表示: B(x, y) = f{[}A(x, y){]}

f() 就是这个变换函数 (Transform Function)

输出图像 \textless= 输入图像,逐点拷贝运算。

\hypertarget{header-n21}{%
\paragraph{例子}\label{header-n21}}

有线性的 Transform,也有非线性的。

\hypertarget{header-n23}{%
\paragraph{应用}\label{header-n23}}

逐点变换可以做的事情有:

\begin{itemize}
\item
  Photometric Calibration
\item
  Contrast Enhancement
\item
  Thresholding
\item
  Contour Lines
\item
  Clipping
\end{itemize}

\hypertarget{header-n36}{%
\subsubsection{代数运算(Algebraic Operation)}\label{header-n36}}

这也是像素级别的运算。输出图像也是一个点一个点算出来的。

但是呢,每个点的结果不仅仅依赖于原图像中对应的一个点(那就是点运算了),

而是由多幅图像进行集合操作,共同计算出结果图像。

几个图像加加减减\ldots 得到结果。

\hypertarget{header-n41}{%
\paragraph{例子}\label{header-n41}}

\begin{verbatim}
C(x, y) = A(x, y) + B(x, y),

C(x, y) = A(x, y) - B(x, y),

C(x, y) = A(x, y) * B(x, y),

C(x, y) = A(x, y) / B(x, y),
\end{verbatim}

就是由两幅图 \texttt{A(x,\ y)} 和 \texttt{B(x,\ y)} 共同构造了结果图
\texttt{C(x,\ y)}。

\hypertarget{header-n44}{%
\subparagraph{图像的加 (Addition)}\label{header-n44}}

可以进行 Noise Reduction:

多张相同角度的图片进行叠加,可以去除每次拍照中出现的随机噪音(Noise)。

也可以进行 Double-exposure effect:二次曝光效果,

在进光量很低的情况下,利用多次曝光积累信息来提高画面亮度。

\hypertarget{header-n49}{%
\subparagraph{图像的减 (Subtraction)}\label{header-n49}}

可以进行「背景去除」,也就是俗称的抠图。

还能进行 Motion
Detection(检测运动物体),通过「Subtraction」探知图像中的运动部分。变化了的(没被
Subtract 掉的)就是 Motion Object 了。

还还可以进行 Gradient Magnitude,梯度放大;

\href{https://www.sciencedirect.com/topics/engineering/gradient-magnitude}{Gradient
Magnitude}

imagic

\hypertarget{header-n55}{%
\subparagraph{图像的× (Multiplication)}\label{header-n55}}

乘算啥?乘就是逻辑与(Logical AND(\&));

实际应用:Mask,把需要的部分画成白色(1),不需要的部分画成黑色(0),得到一张面具(掩码);

跟实际图做与操作,就能把那一部分抠出来。

\hypertarget{header-n59}{%
\subparagraph{图像的除➗(Division)}\label{header-n59}}

常常用于医学;

通常我们除以一个 I(x, y),使得成像效果更明亮;

详见后面的同态滤波

\hypertarget{header-n63}{%
\subsubsection{几何操作(Geometric Operation)}\label{header-n63}}

啥是几何操作啊?

旋转、平移、缩放,等等;这些都是几何操作。

\hypertarget{header-n66}{%
\paragraph{特点}\label{header-n66}}

空间位置发生变化了。没有上面的那些一一对应/多一对应的点映射了。

新的坐标点取什么值?跟哪些点相关?

\hypertarget{header-n69}{%
\paragraph{例子}\label{header-n69}}

基本都要借助矩阵来表示这种几何操作。

缩放、旋转和平移、镜像反转等等都可以通过矩阵来表示。

计算顺序和矩阵一样,也是从右到左。

\hypertarget{header-n73}{%
\paragraph{如何确定一些空缺点的值?}\label{header-n73}}

例如,将一张图放大之后,如何才能确定产生的空缺点的像素值?

这里我们需要了解 Interpolation Algorithm

插值算法。

\begin{verbatim}
作业:了解

* Nearest Neighbor Interpolation、

* Bilinear Interpolation(也称作 Square Interpolation),以及

* Higher Order Interpolation
\end{verbatim}

课上我们先看最简单的那种插值算法「最近邻插值法」(Nearest Neighbor
Interpolation):

谁离这个目标点最近,我们就把这个值赋给他。

计算起来非常快,计算量很小。只是\ldots 效果比较难看,非常容易出现锯齿(Dog's
Teeth)效果。

那么双线性插值呢?

比刚才的好一点,不仅有一个点来确定,而一并考虑周围四个点对他的影响;

\texttt{f(x,\ y)\ =\ ax\ +\ by\ +\ cxy\ +\ d}

且各方向上的系数不同。

\begin{figure}
\centering
\includegraphics{/Users/yue/Desktop/图像.jpeg}
\caption{}
\end{figure}

(公式看 PPT)

由于这一方法依赖于更多的点,因此计算量更大,只是效果会比最近邻要好;交错部分更加平滑,不容易出现生硬的锯齿。

\hypertarget{header-n88}{%
\paragraph{Neighborhood Operations}\label{header-n88}}

淋浴操作(雾)

邻域操作

\hypertarget{header-n91}{%
\subparagraph{连通性检查}\label{header-n91}}

Pixel Connectivity

4 邻域:一个像素点的上下左右四个点。合称四邻域。

8 邻域:一个像素点上下左右左上左下右上右下八个点。合称八邻域。

所谓「4 连通」:只走上下左右四个方向,两个像素点可以连通。

(对于一个二值化的图,这里简单的连通定义是:可以通过四邻域内的行走达到另一个点,则称之为连通。)

「8连通」自己想。差不多。

问题来了:给定两个点,如何判断他们是否连通?

更进一步,如何根据点的连通性,把这些点给分组?

IC:Intensity Criterion,判断两个点是否可以相连的判断准则(一个范围)。

对这里的二值化简单形态,相等就是 True,不等就是 False

经典标记算法 Labeling Algorithm(顺序扫描式):

\begin{verbatim}
1 - First Scan
    从左上 (0, 0) 到右下 (width, height),一边扫描过来。
    如果所有的邻居都不满足 IC,那么给 P 一个新的 Label。
    如果只有一个邻居满足 IC,那么就给 P 这个邻居的 Label。
    如果两个到三个邻居满足 IC,那么随便给 P 其中一个 Label,而且在 Label 表中记录这几个邻居的 Label 等价。
    
这是第一次 Scan。注意经过了第一次 Scan 之后是不算完成的。

2 - Fix Conflicts
    Label 表中出现的等价 Label 全部合并,形成一个最终的 label。
\end{verbatim}

\hypertarget{header-n104}{%
\subsubsection{空间域图像增强}\label{header-n104}}

这是最常见的、最容易理解的图像增强方式。

Image Enhancement in Spacial Domain

\begin{itemize}
\item
  对比度的增强
\item
  图像平滑
\item
  图像锐化
\item
  同态滤波
\end{itemize}

主要的方法:

空间域处理,全局运算(对着整幅图进行运算)、局部运算(对一部分像素进行运算)、以及点运算(仅对单个像素进行运算)。

频域处理:在图像的 Fourier 变换域中处理。

\hypertarget{header-n119}{%
\subparagraph{具体怎么做?}\label{header-n119}}

\begin{itemize}
\item
  对比度增强

  \begin{itemize}
  \item
    灰度变换法

    \begin{itemize}
    \item
      线性变换
    \end{itemize}

    所谓线性的扩展呢,设原来的灰度范围在 \texttt{{[}a,\ b{]}}
    之间;我们想要把他扩展到\texttt{{[}c,\ d{]}},那么我们就使用这个公式进行变换:

    就是一个线性扩展而已\ldots{}

    \begin{itemize}
    \item
      扩展:分段线性灰度变换
    \end{itemize}

    只对其中一部分进行线性拉伸,另外的部分不进行拉伸甚至收窄。简而言之\ldots 只需把上面的
    \texttt{{[}a,\ b{]}\ -\textgreater{}\ {[}c,\ d{]}} 增加到
    \texttt{{[}a,\ b,\ c{]}\ -\textgreater{}\ {[}d,\ e,\ f{]}}
    就能更细致分段地控制每一段的亮度了。

    \begin{itemize}
    \item
      对数变换
    \item
      指数变换
    \end{itemize}
  \item
    直方图📊调整法

    啥是直方图啊?(Histogram)

    一种表示数据的可视化方法而已。

    这里特指的直方图是:数字图像中每一个灰度级别和他的出现的频率的统计关系。

    通常横坐标为灰度级别,纵坐标为频数(出现概率)。

    通过直方图,可以大致知道其亮度偏向:暗还是亮?

    但是其具体的空间信息是丢失了的。这些亮的像素暗的像素在哪?不知道。

    哪些图可以方便地二值化?那些直方图看起来像是马鞍形的(中间有一处灰度级别出现频率较低的)作为谷底可以找到二值化的阈值。

    那么,我们如何调节这个图像使它看起来更好看呢?

    \begin{itemize}
    \item
      直方图均衡化
    \end{itemize}

    如果说直方图中频率集中在比较小的一个灰度级别范围内的话,我们可以把它拉伸到一整个灰度级别范围,均衡化,使得图像看起来更好。

    \begin{itemize}
    \item
      直方图匹配
    \end{itemize}
  \end{itemize}
\end{itemize}

\end{document}
