% Options for packages loaded elsewhere
\PassOptionsToPackage{unicode}{hyperref}
\PassOptionsToPackage{hyphens}{url}
%
\documentclass[
]{article}
\usepackage{lmodern}
\usepackage{amssymb,amsmath}
\usepackage{ifxetex,ifluatex}
\ifnum 0\ifxetex 1\fi\ifluatex 1\fi=0 % if pdftex
  \usepackage[T1]{fontenc}
  \usepackage[utf8]{inputenc}
  \usepackage{textcomp} % provide euro and other symbols
\else % if luatex or xetex
  \usepackage{unicode-math}
  \defaultfontfeatures{Scale=MatchLowercase}
  \defaultfontfeatures[\rmfamily]{Ligatures=TeX,Scale=1}
\fi
% Use upquote if available, for straight quotes in verbatim environments
\IfFileExists{upquote.sty}{\usepackage{upquote}}{}
\IfFileExists{microtype.sty}{% use microtype if available
  \usepackage[]{microtype}
  \UseMicrotypeSet[protrusion]{basicmath} % disable protrusion for tt fonts
}{}
\makeatletter
\@ifundefined{KOMAClassName}{% if non-KOMA class
  \IfFileExists{parskip.sty}{%
    \usepackage{parskip}
  }{% else
    \setlength{\parindent}{0pt}
    \setlength{\parskip}{6pt plus 2pt minus 1pt}}
}{% if KOMA class
  \KOMAoptions{parskip=half}}
\makeatother
\usepackage{xcolor}
\IfFileExists{xurl.sty}{\usepackage{xurl}}{} % add URL line breaks if available
\IfFileExists{bookmark.sty}{\usepackage{bookmark}}{\usepackage{hyperref}}
\hypersetup{
  hidelinks,
  pdfcreator={LaTeX via pandoc}}
\urlstyle{same} % disable monospaced font for URLs
\setlength{\emergencystretch}{3em} % prevent overfull lines
\providecommand{\tightlist}{%
  \setlength{\itemsep}{0pt}\setlength{\parskip}{0pt}}
\setcounter{secnumdepth}{-\maxdimen} % remove section numbering

\date{}

\begin{document}

\hypertarget{header-n0}{%
\section{2019/09/12}\label{header-n0}}

\hypertarget{header-n2}{%
\subsection{SE-344}\label{header-n2}}

\hypertarget{header-n3}{%
\subsubsection{Teacher}\label{header-n3}}

计算机图形学

肖双九 @\texttt{xsjiu99@cs.sjtu.edu.cn}

\hypertarget{header-n6}{%
\subsubsection{TA}\label{header-n6}}

葛续荣、许泽资

\hypertarget{header-n8}{%
\subsubsection{Homework}\label{header-n8}}

按具体要求发邮件上交给
\texttt{cg\_dalab@163.com},或是上传到交大云盘里。

命名规范:学号+姓名+作业编号,打包成一个 .zip / .tar.gz。

\hypertarget{header-n11}!请务必认真进行。
\end{quote}

\begin{quote}
请不要抄袭互联网上或是同学的代码。
\end{quote}

\begin{quote}
⚠️Warning: 课堂笔记都是每次平时作业报告的一部分!且占比很高。

要求每次笔记都需要提出一个有价值的问题。而且最终需要回答他。
\end{quote}

\hypertarget{header-n21}{%
\subsubsection{课件}\label{header-n21}}

逐部分放置在交大云盘上。

\hypertarget{header-n23}{%
\subsubsection{Today Widget: Chapter 0}\label{header-n23}}

计算机图形学入门·计算机绘图基础

\hypertarget{header-n25}{%
\paragraph{Contents}\label{header-n25}}

\begin{itemize}
\item
  计算机图形学入门

  \begin{quote}
  今天的这节课
  \end{quote}
\item
  光栅化图形学

  \begin{quote}
  我们的显示器只是一堆光栅\ldots 怎么表现东西?
  \end{quote}
\item
  三维建模及其变换

  \begin{quote}
  在一个三维坐标系中,如何进行坐标变换,如何进行具体的计算?
  \end{quote}
\item
  光照计算

  \begin{quote}
  要有光,看起来能更真实些
  \end{quote}
\item
  图形的细节渲染

  \begin{quote}
  有了这些,看起来会更真实
  \end{quote}

  \begin{itemize}
  \item
    光影处理
  \item
    纹理
  \item
    图形特效
  \item
    阴影
  \item
    \ldots\ldots{}
  \end{itemize}
\end{itemize}

\hypertarget{header-n58}{%
\subsubsection{进入第一章:计算机图形学入门}\label{header-n58}}

\hypertarget{header-n59}{%
\paragraph{计算机绘图基础}\label{header-n59}}

\hypertarget{header-n60}{%
\subparagraph{Definition: 啥是计算机绘图?}\label{header-n60}}

为啥要绘图?对真实世界的映射。相比于语言文字,人类总能更轻易地理解图形。
另外,绘图能帮助人类完成许多了不起的工程,传达珍贵的工程经验。

\hypertarget{header-n62}{%
\subparagraph{为啥要用计算机绘图?}\label{header-n62}}

算力强,发展快。

自 1940s 开始,计算机辅助绘图已经非常常见且普遍了。

反过来,计算机图形界面也反过来使得计算机更加普及、易用了。

\hypertarget{header-n66}{%
\subparagraph{计算机图形硬件:支撑一切的基础}\label{header-n66}}

没有纸,没有笔,咱们怎么画东西?

\begin{itemize}
\item
  图形输出设备(Image Output Devices)
\end{itemize}

\begin{quote}
没有输出设备您画了半天画了个啥?
\end{quote}

最简单最常见的:显示器、绘图仪、打印机,等等。都可以把计算机绘图的结果展示出来。

论及显示器,到现在还是没有本质变化\ldots 还是由大量发光像素的组合来显示图形的。

无论是 CRT(阴极射线显示器)、LCD(液晶显示器),\ldots\ldots 皆如此。

目前基本都是给予 2D 显示。即,如果用显示器展示 3D
模型的话,等于是损失了一个维度。

\begin{itemize}
\item
  图形输入设备(Image Input Devices)
\end{itemize}

计算机输入设备基本都包括在内,如鼠标、手柄、触摸屏、触摸板、绘图板,等等\ldots\ldots{}

以及一些光学采集设备,如:摄像头、扫描仪、相机,等等\ldots\ldots{}

还有一些比较偏门的:数据手套、坐标数字化仪(没听说过)\ldots\ldots{}

\begin{itemize}
\item
  图形处理设备(Image Processing Devices)
\end{itemize}

好了,有了图形数据的输入输出,就该由计算机来进行处理,实现所需的制图效果了。

摩尔定律、并行计算的发展使得图形处理的能力爆炸式发展;

而人类对真实感和娱乐的强烈需求也刺激了图形处理的发展。

这部分工作该交给谁呢?

显卡!显卡!Graphics Card!专业点讲,Graphics Processing Unit。

\begin{quote}
插入:显卡的历史:
\end{quote}

\begin{quote}
最早的显卡真的就只是用来显示(还是单色的\ldots\ldots),由 IBM 推出于
1981 年,至于显示什么还是交给 CPU 的。
\end{quote}

\begin{quote}
但是 CPU 全才,并不是单单用来做显示计算的,所以其效率较低。
\end{quote}

\begin{quote}
那么,我们逐渐把那些跟图形有关的内容单独拉出来放在另一张芯片里,美其名曰「图形加速卡」,
\end{quote}

\begin{quote}
可以分担一部分 CPU 的功能\ldots 最后它的功能越来越多,反走样,3D
计算,光照计算,特效\ldots\ldots{}
\end{quote}

\begin{quote}
这些全部都写在硬件里,效率很高很高。(虽然很死板,但效率高啊)
\end{quote}

\begin{quote}
最后,这些东西都交给了 GPU。
\end{quote}

\begin{quote}
但是如果所有的这些东西全部写死了,那就失去了灵活性。我们希望他可以兼具灵活性和效率,因此就出现了
\end{quote}

\begin{quote}
可编程 GPU!(Yeah!)直接对硬件编程;加上 GPU 的晶体管已经大大超过了
CPU,Intel 感到极大压力(雾
\end{quote}

这就是一点简单的计算机图形历史了。

\hypertarget{header-n110}{%
\subparagraph{计算机图形软件}\label{header-n110}}

谈完硬件谈软件(用户和机器之间的交互层

\begin{itemize}
\item
  计算机图形 API
\end{itemize}

首先,咱们来介绍几个著名 API:

\begin{itemize}
\item
  跨平台、通用:OpenGL
\item
  Windows 专用、邪恶、微软家的:DirectX
\end{itemize}

现在所有的硬件产品、图形引擎基本都接受两种标准。

其他一些(没什么人用的)API:Java 3D、Java Mobile 3D、Windows
Mobile(已死)APIs

\begin{itemize}
\item
  高级图形渲染器
\end{itemize}

基于以上的那些 APIs,有人构建了一些更高层的引擎:

\begin{itemize}
\item
  Pixar 家的 RenderMan

  \begin{quote}
  只给 Disney/Pixar 自家用
  \end{quote}
\item
  NVidia 家的 Gelato

  \begin{quote}
  与 RenderMan 兼容的,(某些方面)更高效的渲染器(法务部警告
  \end{quote}
\item
  German mental images 公司推出的 MentalRay 渲染器

  \begin{quote}
  已经被 NVidia 买下来了(
  \end{quote}
\end{itemize}

\begin{quote}
Question:API 跟 图形渲染器的关系? A: 图形渲染器基于具体的 API。
\end{quote}

\begin{itemize}
\item
  游戏引擎(另一码事儿)
\end{itemize}

也在 API 上层,而且除去图形相关的东西,还包含了很多其他的
跟游戏内容有关的东西,包括交互、资源管理、逻辑等等等等,(像是
Unity,Unreal 之类的东西)有些杂。

因此这门课里不用游戏引擎来讲。

\hypertarget{header-n147}{%
\subparagraph{OpenGL}\label{header-n147}}

\begin{itemize}
\item
  开放的图形 API,工业化的标准
\item
  完全独立于窗口系统,OS,等等一切东西。
\item
  从点线面开始,构建一切东西
\item
  能力:创建二维、三维物体,布置(layout)并观察场景
\end{itemize}

\hypertarget{header-n157}{%
\subparagraph{OpenGL 的库函数}\label{header-n157}}

核心库:名为 gL,所有的函数名开头都是 GL。

实用库:GLU,所有函数名开头均为 GLU。

辅助库:GLaux,一些窗口相关的东西

工具库:GLUT,创建窗口和其他 GUI 的库函数

另有基于 GLUT 的 GUI 库名为 GLUI。

面向 Windows 的扩展:WGL,仅仅包括很旧版本的 OpenGL。

OpenGL 扩展库并没有它本身那么标准\ldots 基本也是群魔乱舞\ldots{}

Warning:提交作业时附上你使用的库。(否则 TA 跑不起来\ldots{}

\end{document}
