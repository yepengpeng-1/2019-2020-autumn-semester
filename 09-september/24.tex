% Options for packages loaded elsewhere
\PassOptionsToPackage{unicode}{hyperref}
\PassOptionsToPackage{hyphens}{url}
%
\documentclass[
]{article}
\usepackage{lmodern}
\usepackage{amssymb,amsmath}
\usepackage{ifxetex,ifluatex}
\ifnum 0\ifxetex 1\fi\ifluatex 1\fi=0 % if pdftex
  \usepackage[T1]{fontenc}
  \usepackage[utf8]{inputenc}
  \usepackage{textcomp} % provide euro and other symbols
\else % if luatex or xetex
  \usepackage{unicode-math}
  \defaultfontfeatures{Scale=MatchLowercase}
  \defaultfontfeatures[\rmfamily]{Ligatures=TeX,Scale=1}
\fi
% Use upquote if available, for straight quotes in verbatim environments
\IfFileExists{upquote.sty}{\usepackage{upquote}}{}
\IfFileExists{microtype.sty}{% use microtype if available
  \usepackage[]{microtype}
  \UseMicrotypeSet[protrusion]{basicmath} % disable protrusion for tt fonts
}{}
\makeatletter
\@ifundefined{KOMAClassName}{% if non-KOMA class
  \IfFileExists{parskip.sty}{%
    \usepackage{parskip}
  }{% else
    \setlength{\parindent}{0pt}
    \setlength{\parskip}{6pt plus 2pt minus 1pt}}
}{% if KOMA class
  \KOMAoptions{parskip=half}}
\makeatother
\usepackage{xcolor}
\IfFileExists{xurl.sty}{\usepackage{xurl}}{} % add URL line breaks if available
\IfFileExists{bookmark.sty}{\usepackage{bookmark}}{\usepackage{hyperref}}
\hypersetup{
  hidelinks,
  pdfcreator={LaTeX via pandoc}}
\urlstyle{same} % disable monospaced font for URLs
\usepackage{longtable,booktabs}
% Correct order of tables after \paragraph or \subparagraph
\usepackage{etoolbox}
\makeatletter
\patchcmd\longtable{\par}{\if@noskipsec\mbox{}\fi\par}{}{}
\makeatother
% Allow footnotes in longtable head/foot
\IfFileExists{footnotehyper.sty}{\usepackage{footnotehyper}}{\usepackage{footnote}}
\makesavenoteenv{longtable}
\usepackage{graphicx,grffile}
\makeatletter
\def\maxwidth{\ifdim\Gin@nat@width>\linewidth\linewidth\else\Gin@nat@width\fi}
\def\maxheight{\ifdim\Gin@nat@height>\textheight\textheight\else\Gin@nat@height\fi}
\makeatother
% Scale images if necessary, so that they will not overflow the page
% margins by default, and it is still possible to overwrite the defaults
% using explicit options in \includegraphics[width, height, ...]{}
\setkeys{Gin}{width=\maxwidth,height=\maxheight,keepaspectratio}
% Set default figure placement to htbp
\makeatletter
\def\fps@figure{htbp}
\makeatother
\setlength{\emergencystretch}{3em} % prevent overfull lines
\providecommand{\tightlist}{%
  \setlength{\itemsep}{0pt}\setlength{\parskip}{0pt}}
\setcounter{secnumdepth}{-\maxdimen} % remove section numbering

\date{}

\begin{document}

\hypertarget{header-n0}{%
\section{2019/09/24}\label{header-n0}}

\hypertarget{header-n2}{%
\subsection{SE-302 Compilers}\label{header-n2}}

今天我们要讲的是一种文法判断的方法。

\hypertarget{header-n4}{%
\subsubsection{方法一:Top Down Parsing}\label{header-n4}}

从顶向下的 parse,常用的实现是「递归下降法」。

假设我们现在已经有了一组 \texttt{Tokens}:

\begin{verbatim}
t1 t2 t3 t4 t5
\end{verbatim}

如果合法的话,应该能构成一个 \texttt{Parse\ Tree}。

因为我们用的是 Top -\textgreater{} Down,因此我们从根儿上开始构造。

基本上按照自顶向下,从左向右的顺序。

\begin{verbatim}
E -> T + E|T
T -> int | int * T | (E)
\end{verbatim}

留意到我们上面的文法是(加法和乘法)右结合的,且乘法优先级高于加法。

\hypertarget{header-n13}{%
\subsubsection{开始举例}\label{header-n13}}

我们用这个 Token Stream 来举例:

\begin{verbatim}
int5 * int2
\end{verbatim}

我们现在用这两个 Production 一个一个来尝试。

\begin{enumerate}
\def\labelenumi{\arabic{enumi}.}
\item
  E0 -\textgreater{} T1 + E2,

  \begin{enumerate}
  \def\labelenumii{\arabic{enumii}.}
  \item
    T1 -\textgreater{} (E3),显然原柿子里面没有做括号。完蛋。
  \item
    T1 -\textgreater{} int,这个 Match 了。但是发现 + 跟原来的
    \texttt{*} token 匹配不上,不 Match。再找!
  \item
    T1 -\textgreater{} \texttt{int\ *\ T},这个勉强 Match,但之后 T1
    是匹配不上的(哪来的 + 呢)
  \item
    没得选了,看来最开头一个 Production 就走错了。回溯!
  \end{enumerate}
\item
  我们现在用另一个产生柿来做。换成 \texttt{E0\ -\textgreater{}\ T}!

  \begin{enumerate}
  \def\labelenumii{\arabic{enumii}.}
  \item
    \ldots\ldots{}
  \item
    让我们来试试 \texttt{T\ -\textgreater{}\ int\ *\ T}。Match 了!

    \begin{enumerate}
    \def\labelenumiii{\arabic{enumiii}.}
    \item
      然后 T -\textgreater{} int。匹配出了 \texttt{int\ *\ int}!
    \end{enumerate}
  \end{enumerate}
\end{enumerate}

大概就是这样进行自顶向下的 Token Stream Match 了\ldots\ldots{}

如果找过了所有都还没成功\ldots{}

这种题么,带回溯的递归还是比较难编写的\ldots(想想你的 LC)

\hypertarget{header-n42}{%
\paragraph{情况说明}\label{header-n42}}

Top Down Parsing 啊,并不是总能用的。

我们刚才用的是右结合的文法。为什么?

因为左结合 Sometimes doesn't work. 并不是总对的。

考虑 \texttt{S\ -\textgreater{}\ S\ a} 这种文法。这会带来什么问题吗?

\texttt{S\ -\textgreater{}\ S\ a\ -\textgreater{}\ S\ S\ a\ -\textgreater{}\ S\ S\ S\ a\ -\textgreater{}\ ...}

我好像总是拿不出来一个 Token,总可能还有机会,总把自己拿去尝试\ldots{}

结果就是无限递归,停不下来。Left-recursive grammar 如果有了一个
Non-Terminal S,这种情况就会造成 Infinite Loop。

\hypertarget{header-n50}{%
\paragraph{解决方案}\label{header-n50}}

消左递归的方法:

以 \texttt{S\ -\textgreater{}\ S\ a\textbar{}b} 为例。将其重写为
\texttt{S\ -\textgreater{}\ bS,\ S\textquotesingle{}\ -\textgreater{}\ aS\textquotesingle{}\ \textbar{}\ epsilon}。

按照语义将其改写为不包含左递归的形式。这样的方法可以解决这个问题。

\hypertarget{header-n54}{%
\subsubsection{优缺点}\label{header-n54}}

因为回溯递归的原因,unpopular。大家都不爱使用这种方法。当年 70
年代提出的时候,大家觉得效率太低,又是穷举又是递归回溯的\ldots{}

但事实上这是最通用的办法了。如果想要简化,那就只能结合具体的文法针对性写算法了。

\hypertarget{header-n57}{%
\subsubsection{方法二:Predictive Parsers}\label{header-n57}}

预测式 Parse,类似上面的 Recursive Descent,但是我先从 Input 里面找
token,去猜测我该使用哪种产生式。

为了做预测,我们往前走一点,往前看 k
个字符来做猜测,进而比较有针对性地去尝试产生式的方法,同时可以筛选掉一些完全不符合条件的产生式。

因为你是去「猜」的,所以必定是需要了解特定的文法,比如他的哪些 Token
对应哪些产生式啊?Token 的格式是啥样啊?你总得知道。

提前 peek \texttt{k} 个字符的算法唤做 \texttt{LL(k)}。

(最常用的 \texttt{LL(1)}。也就是仅仅提前吃一个
Character。但这也跟文法有关\ldots)

这种代码适合手写\ldots 因为严重跟原文法有关,且难以自动生成。

另外,事实上 LL 不好处理表达式。对于固定的类似于 if then else print
begin end 等词法单位,还是很好处理的。

\hypertarget{header-n65}{%
\paragraph{问题解决}\label{header-n65}}

为了解决 LL 不好处理表达式的问题,我们进行一下 Left-Factored Grammer。

比如:当我们發现要搞一个 Expression 出来的时候,我们要去在一张 2 × 2
的表里头找。

这张表行代表当前要找的目标类型(Statement?Expression?Terminal?),列代表要拿到的下一个
Token 的类型(int,+,×,(,),¥ 等)

然后行和列组合拿到了要进行的 Action。于是就知道下一步怎么走了。

\begin{longtable}[]{@{}llllll@{}}
\toprule
& int & * & + & ( & )\tabularnewline
\midrule
\endhead
T & int Y & & & (E) &\tabularnewline
E & T X & & & T X &\tabularnewline
X & & & + E & & ε\tabularnewline
Y & & *T & ε & & ε\tabularnewline
\bottomrule
\end{longtable}

如果有了这张表,程序就能自然自动地进行 Tree Parse 工作了。

算出所有的 Non-T 的 First,这会为我们下面计算 Follow Sets 产生帮助。

然后进一步,我们还可以计算出 Follow(An),算出来一个 An 后面可能跟的
Token 的集合。例如:

\begin{verbatim}
Follow( + ) = { int, ( } 
Follow( * ) = { int, ( } 
Follow( ( ) = { int, ( } 
Follow( E ) = {), $} 
Follow( X ) = {$, ) } 
Follow( T ) = {+, ) , $} 
Follow( ) ) = {+, ) , $} 
Follow( Y ) = {+, ) , $} 
Follow( int ) = {*, +, ) , $}
\end{verbatim}

计算方法:

\begin{verbatim}
Add First(A1) – {epsilon} to Follow(X). 
Stop if epsilon !in First(A1)   

Add First(A2) – {epsilon} to Follow(X). 
Stop if epsilon !in First(A2)  

…  

Add First(An) – {epsilon} to Follow(X). 
Stop if epsilon !in First(An)  

// 如果到最后每一个 First(An) 里面都有 epsilon,
// 那么 Follow(Y) 也可以作为 Follow(X) 来用,补进去
Add Follow(Y) to Follow(X)
\end{verbatim}

直到最后,我们就可以构造那张 LL(1) Parsing Table 了。

\hypertarget{header-n113}{%
\subsubsection{Error Recovery}\label{header-n113}}

出错之后,我们总是希望尽可能多进行一些错误检查,多爆一些错误才好。

有两种思路可以往下进行:

\hypertarget{header-n116}{%
\paragraph{a. Insert a Token}\label{header-n116}}

缺了啥东西我给你补上,接着编译(

但有个问题:如果说你补的策略不恰当,补了又补,有可能产生一坨死循环。永远地往下补了下去。

\hypertarget{header-n119}{%
\paragraph{b. Delete Tokens}\label{header-n119}}

这个方案就是:这行写的不对?我给你全删掉(接着往下走

这样的好处就是:这种策略一定是收敛的(因为程序长度是有限的),不会删出死循环的。

\hypertarget{header-n123}{%
\subsection{SE-227}\label{header-n123}}

今天的主题:File System Design and Optimizations。

(还要讲文件系统啊)\includegraphics{https://raw.githubusercontent.com/yuetsin/private-image-repo/master/2019/09/24-10-02-33-2019-09-24-10-02-23-image.png}

首先我们不要忘记,FileSystem 始终是基于我们的 Disk 的。HDD 也好 SSD
也好,都是 FS 的基础。

HDD 的一个关键参数就是:转速(单位一般用的是 rpm,rounds per minute)。

转速越快,盘速越快。

寻道的速度一般在 ms 级别。

但是 HDD 最差的性能就是 RANDOM ACCESS
TIME。也就是\textbf{任意读取性能}是很差的。

如果两条数据在物理磁道上的保存位置差别很大的话,就需要移动机械臂,因此速度是相当慢的。

\hypertarget{header-n132}{%
\subsubsection{设计.Design}\label{header-n132}}

Design Guidelines and Dimensions:设计有多种维度,都有着不同的准则。

可以从普通用户、应用开发者、以及系统开发者的角度来分析设计准则。

\begin{itemize}
\item
  性能表现

  \begin{itemize}
  \item
    Latency(延迟)多大?(一般是延迟的平均值)
  \item
    Tail Latency(尾延迟,即最大的延迟)
  \item
    Throughput(吞吐量)
  \item
    Scalability(可扩展性)
  \item
    Aged Performance(上了年纪「用了很久」之后的性能表现变化)
  \end{itemize}
\item
  崩溃恢复(例如 Windows 的 \texttt{chkdsk})
\item
  恢复时间(从错误中恢复要花多久?)
\item
  可靠性

  \begin{itemize}
  \item
    Robustness
  \item
    Wear-leveling,擦了写写了擦把纸写穿之后 FS 能不能正确处理这种情况?
  \item
    Fragmentation,碎片化?
  \end{itemize}
\end{itemize}

\hypertarget{header-n162}{%
\subsubsection{优化.Opt}\label{header-n162}}

\hypertarget{header-n163}{%
\paragraph{FFS}\label{header-n163}}

咱们来看看 A Fast File System(FFS) for UNIX 是怎么优化的。

\begin{figure}
\centering
\includegraphics{https://raw.githubusercontent.com/yuetsin/private-image-repo/master/2019/09/24-10-30-30-2019-09-24-10-30-28-image.png}
\caption{}
\end{figure}

相对于原版 UNIX FileSystem,进行了这些优化:

\begin{enumerate}
\def\labelenumi{\arabic{enumi}.}
\item
  用类似于 ICS Chapter 9 的 Header/Footer,Bitmap,用字节位来记录是否
  occupied。
\item
  在每个文件的末尾处预留 10\% 的空间,保证文件可以尽量连续保存。
\item
  Skip-sector positioning(稍后再谈)
\end{enumerate}

\hypertarget{header-n174}{%
\paragraph{FFS 规避 Rotational Delay}\label{header-n174}}

HDD 在连续读的时候比任意读的速度更快一些(也就是一些)

然而因为磁盘可能一直在转,我们在连续读取 Block
之间可能磁头就已经转走了\ldots{}

因此我们可以用以下两种方案来解决这个问题:

\begin{enumerate}
\def\labelenumi{\arabic{enumi}.}
\item
  干脆不要连续读了,每次隔一点时间再读
\item
  一次读两个 Block(硬件支持或者软件实现都可以)
\end{enumerate}

\begin{figure}
\centering
\includegraphics{https://raw.githubusercontent.com/yuetsin/private-image-repo/master/2019/09/24-10-39-51-2019-09-24-10-39-49-image.png}
\caption{}
\end{figure}

\begin{quote}
Cylinder Group Diagram
\end{quote}

\hypertarget{header-n186}{%
\paragraph{FFS 采用 First-fit Block Allocation}\label{header-n186}}

最先匹配块分配原则:尽量把文件的 Block 给塞在从头到尾找的那些 Block 里。

总归是能先则填。

总归是有 Pros \& Cons 的。有的情况好而有的情况差。

\begin{figure}
\centering
\includegraphics{https://raw.githubusercontent.com/yuetsin/private-image-repo/master/2019/09/24-10-45-19-2019-09-24-10-45-17-image.png}
\caption{}
\end{figure}

\begin{quote}
Reading 2 就涉及到这部分的内容。阅读时多加注意。
\end{quote}

\begin{quote}
下面都是关于 MEDIA \& STORAGE,即存储介质的历史和未来的介绍。

直接去看 PPT 好了(P22 - 52)。
\end{quote}

\end{document}
