% Options for packages loaded elsewhere
\PassOptionsToPackage{unicode}{hyperref}
\PassOptionsToPackage{hyphens}{url}
%
\documentclass[
]{article}
\usepackage{lmodern}
\usepackage{amssymb,amsmath}
\usepackage{ifxetex,ifluatex}
\ifnum 0\ifxetex 1\fi\ifluatex 1\fi=0 % if pdftex
  \usepackage[T1]{fontenc}
  \usepackage[utf8]{inputenc}
  \usepackage{textcomp} % provide euro and other symbols
\else % if luatex or xetex
  \usepackage{unicode-math}
  \defaultfontfeatures{Scale=MatchLowercase}
  \defaultfontfeatures[\rmfamily]{Ligatures=TeX,Scale=1}
\fi
% Use upquote if available, for straight quotes in verbatim environments
\IfFileExists{upquote.sty}{\usepackage{upquote}}{}
\IfFileExists{microtype.sty}{% use microtype if available
  \usepackage[]{microtype}
  \UseMicrotypeSet[protrusion]{basicmath} % disable protrusion for tt fonts
}{}
\makeatletter
\@ifundefined{KOMAClassName}{% if non-KOMA class
  \IfFileExists{parskip.sty}{%
    \usepackage{parskip}
  }{% else
    \setlength{\parindent}{0pt}
    \setlength{\parskip}{6pt plus 2pt minus 1pt}}
}{% if KOMA class
  \KOMAoptions{parskip=half}}
\makeatother
\usepackage{xcolor}
\IfFileExists{xurl.sty}{\usepackage{xurl}}{} % add URL line breaks if available
\IfFileExists{bookmark.sty}{\usepackage{bookmark}}{\usepackage{hyperref}}
\hypersetup{
  hidelinks,
  pdfcreator={LaTeX via pandoc}}
\urlstyle{same} % disable monospaced font for URLs
\setlength{\emergencystretch}{3em} % prevent overfull lines
\providecommand{\tightlist}{%
  \setlength{\itemsep}{0pt}\setlength{\parskip}{0pt}}
\setcounter{secnumdepth}{-\maxdimen} % remove section numbering

\date{}

\begin{document}

\hypertarget{header-n0}{%
\section{2019/09/23}\label{header-n0}}

\hypertarget{header-n2}{%
\subsection{SE-342}\label{header-n2}}

\hypertarget{header-n3}{%
\subsubsection{计算机图形学笔记}\label{header-n3}}

回忆上节课:提到了「用直方图来进行图像增强」。没讲完。咱们接着来说?

\hypertarget{header-n5}{%
\subsubsection{直方图均衡化}\label{header-n5}}

对于 像素频率 - 灰度级别
直方图中,如果灰度级别比较集中,实际表现就是对比度低,没有什么亮-暗的对比。

我们想一些办法把原来的灰度级别做一个 (A, B) -\textgreater{} (C, D)
的拉伸映射,就可以让灰度分布范围更广,明暗对比更加强烈,显示效果相对更好一些。

具体一点的操作时:将原图像的直方图通过变换函数修整成较为均衡的直方图。

注意,这个修正和空间无关(因为直方图也不包含空间信息),所做的事情只是重新映射灰度级别频率图而已。因此是全图生效的。

\hypertarget{header-n10}{%
\paragraph{怎么实现?}\label{header-n10}}

r 代表归一化的灰度级别(最大值为 1),P(r) 为原图中对应灰度级别 r
中的频率。

现在我们想要把直方图给修正成一个更好看(更均衡、更平直)的直方图。并且要保证倾向性不变,即原来暗的地方也还是要相对较暗,亮的地方也要相对较亮。

又因为灰度的变换不应该影响像素的分布和像素数量,因此变换前后对 p(r)
做积分(如果是离散情况,那就是直接求和)得到的像素数量应该保持不变。

\hypertarget{header-n14}{%
\paragraph{问题}\label{header-n14}}

这种 Re-mapping
的问题是:我们的变换函数很可能会导致有一些原来分开的灰度级别合并,但是我们绝对没有可能去把原来合并的灰度级别给拆成两个(直方图均衡化的实质决定不能做到)。因此结果就是会造成
Discrete,丢失细节信息。甚至我们都不知道丢失了哪个位置的信息。因为直方图根本不知道位置信息。

\hypertarget{header-n16}{%
\subsubsection{局部直方图均衡化}\label{header-n16}}

我们有了一个解决方案:AHE(Adaptive Histogram Equalization)。

对于每一个像素点,我们都取出靠近他的一个稍大的方形,对其中这一小块做局部的直方图均衡化。最后,再移动到下一个像素,再次进行直方图均衡化。

这就是 AHE\ldots\ldots{}

这有什么好处呢?

之前我们的 Mapping
是会在全局损失信息的。也就是说,如果有一块很亮的地方,他就会吃掉暗部的大量信息。因为在直方图中他们并不能被区分开。

如果我们分开进行局部均衡化的话,因为在一小块的范围内很难出现很大的明暗反差,因此就能尽量少的损失细节,也能增加整体的灰度级别(不一样的小块可以产生不一样的灰度级别)。

1985 年 AHE 算法就由 Pizer 提出来了\ldots{}

但实际上分块计算的代价是更大的。极大地增加了计算复杂度。

\hypertarget{header-n25}{%
\paragraph{优化 AHE}\label{header-n25}}

对于那些亮度类似的部分,我们直接将其分成一整块是不是能节约些时间呢?

因此我们就使用聚类来进行分块,再对每一块进行 AHE
就能节约时间,并且不显著影响 AHE 的结果。

\hypertarget{header-n28}{%
\paragraph{AHE: Desired Histogram}\label{header-n28}}

我们还有另一种方法来实现 Re-mapping:

先给出你希望得到的(Desired)直方图图形,(但保证积分结果跟原来的直方图的积分结果相同),然后将每一个灰度级别按照积分的数量对应到另一个灰度级别(保证像素数量及以及相对关系不改变),一点点对应构成一张映射表(Table)。最后按照这张表重新分配灰度级别就实现了
HE。

\hypertarget{header-n31}{%
\subsubsection{图像的增强:图像平滑}\label{header-n31}}

平滑嘛\ldots 大概就是清除掉图像里面的一些噪音,一些不平滑的、坑坑洼洼的
Noise。

分为两种:一种是空间域的平滑,另外一种是频域的平滑。

我们今天仅仅考虑空间域的平滑。

\hypertarget{header-n35}{%
\paragraph{空间域平滑/局部平均法}\label{header-n35}}

计算快,实现简单,结果「比较令人满意」(看你要求高不高了)

简单局部平均法:

假定 \texttt{g(x,\ y)\ =\ f(x,\ y)\ +\ n(x,\ y)},其中 n(x, y)
就是噪音。

我们需要

\texttt{g(x,\ y)\ =\ 1/M\ *\ Sum(g(x,\ y)\ nearby\textquotesingle{}s\ M\ points)}

基本假设:图像的噪音可加的、均值为 0 的、与图像信号无关的。

简单进行这种操作毫无疑问会导致图像变模糊\ldots 轮廓边缘尤甚。

\hypertarget{header-n43}{%
\paragraph{空间域平滑/中值滤波法}\label{header-n43}}

这种方法实际上更加常用。实现简单,效果不错。

用局部的中值来代替局部的平均值,刚刚的局部平均法会导致模糊,现在我们改用中值。

(为了简单表示,这里仅仅考虑灰度中值)

计算量也小,计算也算简单。

仅仅对脉冲持续期小于窗口宽/2 的脉冲才有效果。

然而局部平均法会对任何变化都产生影响,也就是因此会在边缘产生模糊效果。

\hypertarget{header-n50}{%
\paragraph{空间域平滑/加权平均法}\label{header-n50}}

(讲卷积的时候还会回到这里的)

\hypertarget{header-n52}{%
\paragraph{空间域平滑/多帧平均法}\label{header-n52}}

要是你在同样的位置拍了很多张,那可以用这种方法的

\end{document}
