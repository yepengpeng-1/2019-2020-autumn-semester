\documentclass{article}

\usepackage[utf8]{inputenc}

\usepackage{xeCJK}

\usepackage{listings}

\usepackage{fontspec}

\setmonofont{Monaco}

\usepackage{indentfirst}

\setlength{\parindent}{2em}

\linespread{1.5} 

\newfontfamily\monaco{Monaco}

\begin{document}

\section*{Hands on 1}

\subsection*{Question 1}

In my word, the role that Domain Name Service plays is providing a mapping between the domain name and IP address, with which we could access the remote servers via simple semantic domain names rather than meaningless number tuples.

\subsection*{Question 2}

The ``dig" command returns as following in my environment:

\begin{lstlisting}[numberstyle=\tiny\monaco,
        basicstyle=\small\monaco]
; <<>> DiG 9.8.3-P1 <<>> www.sina.com
;; global options: +cmd
;; Got answer:
;; ->>HEADER<<- opcode: QUERY, status: NOERROR, id: 12851
;; flags: qr rd ra; QUERY: 1, ANSWER: 4, AUTHORITY: 0, ADDITIONAL: 0

;; QUESTION SECTION:
;www.sina.com.			IN	A

;; ANSWER SECTION:
www.sina.com.		59	IN	CNAME	us.sina.com.cn.
us.sina.com.cn.		59	IN	CNAME	spool.grid.sinaedge.com.
spool.grid.sinaedge.com. 59	IN	A	58.205.212.26
spool.grid.sinaedge.com. 59	IN	A	58.205.212.27

;; Query time: 482 msec
;; SERVER: 8.8.4.4#53(8.8.4.4)
;; WHEN: Sat Oct 19 15:15:47 2019
;; MSG SIZE  rcvd: 124
\end{lstlisting}

According to the manual, the fourth line in the ANSWER SECTION indicates the type of the DNS record.

``A" refers to the most plain address record: a domain name matched with an IPv4 address.

``AAAA" is also plain, but matched with an IPv6 address.

``CNAME" isn't so direct as ``A" and ``AAAA" type; it means this domain isn't mapped with an IP address directly. It is mapped with another domain name and the actual IP address should be obtained via that indirect domain name, though. So you can see those records with ``A" type have specific IP addresses on the right, however those with ``CNAME" type have other domain names on the right side.

There are also some unnatural types of records, like ``MX", ``NS", ``SOA", and ``TXT". Omit them to reduce the size of this PDF document.

\subsection*{Question 3}

The ``dig" command displays which Domain Name Server it is using on the third last line. For example, I'm currently using Google's DNS at $8.8.4.4\#53$, because of my earlier system configuration. But if we want to explicitly declare the DNS that ``dig" command uses, we could run the following command:

\begin{lstlisting}[numberstyle=\tiny\monaco,
        basicstyle=\small\monaco]
dig @4.2.2.1 google.com.hk
\end{lstlisting}

And here's the result:

\begin{lstlisting}[numberstyle=\tiny\monaco,
        basicstyle=\small\monaco]
; <<>> DiG 9.8.3-P1 <<>> @4.2.2.1 google.com.hk
; (1 server found)

......

;; ANSWER SECTION:
google.com.hk.		247	IN	A	31.13.86.1

;; Query time: 43 msec
;; SERVER: 4.2.2.1#53(4.2.2.1)
;; WHEN: Sat Oct 19 15:50:01 2019
;; MSG SIZE  rcvd: 47
\end{lstlisting}

\subsection*{Question 4}

Interesting question. First let's see if we can get 13 root domain name servers:

\begin{lstlisting}[numberstyle=\tiny\monaco,
        basicstyle=\small\monaco]
; <<>> DiG 9.8.3-P1 <<>> . ns
;; global options: +cmd
;; Got answer:
;; ->>HEADER<<- opcode: QUERY, status: NOERROR, id: 33807
;; flags: qr rd ra; QUERY: 1, ANSWER: 13, AUTHORITY: 0, ADDITIONAL: 0

;; QUESTION SECTION:
;.				IN	NS

;; ANSWER SECTION:
.			60407	IN	NS	a.root-servers.net.
.			60407	IN	NS	b.root-servers.net.
.			60407	IN	NS	c.root-servers.net.
.			60407	IN	NS	d.root-servers.net.
.			60407	IN	NS	e.root-servers.net.
.			60407	IN	NS	f.root-servers.net.
.			60407	IN	NS	g.root-servers.net.
.			60407	IN	NS	h.root-servers.net.
.			60407	IN	NS	i.root-servers.net.
.			60407	IN	NS	j.root-servers.net.
.			60407	IN	NS	k.root-servers.net.
.			60407	IN	NS	l.root-servers.net.
.			60407	IN	NS	m.root-servers.net.

;; Query time: 264 msec
;; SERVER: 8.8.8.8#53(8.8.8.8)
;; WHEN: Sat Oct 19 15:52:22 2019
;; MSG SIZE  rcvd: 228
\end{lstlisting}

\newpage

So when we are digging information about the librone.csail.mit.edu(.), we would get the following:

\begin{lstlisting}[numberstyle=\tiny\monaco,
        basicstyle=\small\monaco]
; <<>> DiG 9.8.3-P1 <<>> lirone.csail.mit.edu
;; global options: +cmd
;; Got answer:
;; ->>HEADER<<- opcode: QUERY, status: NOERROR, id: 22227
;; flags: qr rd ra; QUERY: 1, ANSWER: 1, AUTHORITY: 0, ADDITIONAL: 0

;; QUESTION SECTION:
;lirone.csail.mit.edu.		IN	A

;; ANSWER SECTION:
lirone.csail.mit.edu.	1799	IN	A	128.52.129.186

;; Query time: 476 msec
;; SERVER: 8.8.8.8#53(8.8.8.8)
;; WHEN: Sat Oct 19 15:51:56 2019
;; MSG SIZE  rcvd: 54

\end{lstlisting}

Under the current circumstance, lirone.csail.mit.edu. are mapped directly to 128.52.129.186. But when we see it with +trace command, here's what we've gotten:

\begin{lstlisting}[numberstyle=\tiny\monaco,
        basicstyle=\small\monaco]
; <<>> DiG 9.8.3-P1 <<>> lirone.csail.mit.edu. +trace
;; global options: +cmd
.			60099	IN	NS	m.root-servers.net.
.			60099	IN	NS	b.root-servers.net.
.			60099	IN	NS	c.root-servers.net.
.			60099	IN	NS	d.root-servers.net.
.			60099	IN	NS	e.root-servers.net.
.			60099	IN	NS	f.root-servers.net.
.			60099	IN	NS	g.root-servers.net.
.			60099	IN	NS	h.root-servers.net.
.			60099	IN	NS	a.root-servers.net.
.			60099	IN	NS	i.root-servers.net.
.			60099	IN	NS	j.root-servers.net.
.			60099	IN	NS	k.root-servers.net.
.			60099	IN	NS	l.root-servers.net.
;; Received 228 bytes from 8.8.8.8#53(8.8.8.8) in 3759 ms

edu.			172800	IN	NS	e.edu-servers.net.
edu.			172800	IN	NS	i.edu-servers.net.
edu.			172800	IN	NS	l.edu-servers.net.
edu.			172800	IN	NS	k.edu-servers.net.
edu.			172800	IN	NS	j.edu-servers.net.
edu.			172800	IN	NS	b.edu-servers.net.
edu.			172800	IN	NS	d.edu-servers.net.
edu.			172800	IN	NS	a.edu-servers.net.
edu.			172800	IN	NS	m.edu-servers.net.
edu.			172800	IN	NS	f.edu-servers.net.
edu.			172800	IN	NS	c.edu-servers.net.
edu.			172800	IN	NS	g.edu-servers.net.
edu.			172800	IN	NS	h.edu-servers.net.
;; Received 497 bytes from 199.7.83.42#53(199.7.83.42) in 4095 ms

mit.edu.		172800	IN	NS	usw2.akam.net.
mit.edu.		172800	IN	NS	asia1.akam.net.
mit.edu.		172800	IN	NS	asia2.akam.net.
mit.edu.		172800	IN	NS	use2.akam.net.
mit.edu.		172800	IN	NS	ns1-37.akam.net.
mit.edu.		172800	IN	NS	ns1-173.akam.net.
mit.edu.		172800	IN	NS	eur5.akam.net.
mit.edu.		172800	IN	NS	use5.akam.net.
;; Received 205 bytes from 192.12.94.30#53(192.12.94.30) in 2969 ms

csail.mit.edu.		1800	IN	NS	auth-ns2.csail.mit.edu.
csail.mit.edu.		1800	IN	NS	auth-ns0.csail.mit.edu.
csail.mit.edu.		1800	IN	NS	auth-ns3.csail.mit.edu.
csail.mit.edu.		1800	IN	NS	auth-ns1.csail.mit.edu.
;; Received 222 bytes from 184.26.161.64#53(184.26.161.64) in 1842 ms

lirone.csail.mit.edu.	1800	IN	A	128.52.129.186
;; Received 54 bytes from 128.30.2.123#53(128.30.2.123) in 394 ms
\end{lstlisting}

That's a perfect hierarchy! It starts from 13 root DNSs, and then the edu. (ruled by the MoE or something), mit.edu. (ruled by the MIT), csail.mit.edu. (ruled by the MIT Computer Science \& Artificial Intelligence Lab), and finally the domain lirone.csail.mit.edu. (ruled by The Programming Methodology Group, CSAIL, MIT.)

\subsection*{Question 5 \& 6}

Differences between Twitter and Baidu's treatment?

\subsubsection*{Twitter}

~ dig www.twitter.com +trace

\begin{lstlisting}[numberstyle=\tiny\monaco,
        basicstyle=\small\monaco]
; <<>> DiG 9.8.3-P1 <<>> www.twitter.com +trace
;; global options: +cmd
.			59784	IN	NS	i.root-servers.net.
.			59784	IN	NS	k.root-servers.net.
.			59784	IN	NS	e.root-servers.net.
.			59784	IN	NS	g.root-servers.net.
.			59784	IN	NS	d.root-servers.net.
.			59784	IN	NS	c.root-servers.net.
.			59784	IN	NS	h.root-servers.net.
.			59784	IN	NS	l.root-servers.net.
.			59784	IN	NS	a.root-servers.net.
.			59784	IN	NS	m.root-servers.net.
.			59784	IN	NS	b.root-servers.net.
.			59784	IN	NS	j.root-servers.net.
.			59784	IN	NS	f.root-servers.net.
;; Received 228 bytes from 8.8.8.8#53(8.8.8.8) in 317 ms

com.			172800	IN	NS	d.gtld-servers.net.
com.			172800	IN	NS	k.gtld-servers.net.
com.			172800	IN	NS	h.gtld-servers.net.
com.			172800	IN	NS	g.gtld-servers.net.
com.			172800	IN	NS	c.gtld-servers.net.
com.			172800	IN	NS	b.gtld-servers.net.
com.			172800	IN	NS	a.gtld-servers.net.
com.			172800	IN	NS	m.gtld-servers.net.
com.			172800	IN	NS	f.gtld-servers.net.
com.			172800	IN	NS	i.gtld-servers.net.
com.			172800	IN	NS	j.gtld-servers.net.
com.			172800	IN	NS	l.gtld-servers.net.
com.			172800	IN	NS	e.gtld-servers.net.
;; Received 493 bytes from 192.33.4.12#53(192.33.4.12) in 26 ms

www.twitter.com.	190	IN	A	69.171.232.21
;; Received 49 bytes from 192.54.112.30#53(192.54.112.30) in 28 ms

\end{lstlisting}

\subsubsection*{Baidu}

~ dig www.baidu.com +trace

\begin{lstlisting}[numberstyle=\tiny\monaco,
        basicstyle=\small\monaco]
; <<>> DiG 9.8.3-P1 <<>> www.baidu.com +trace
;; global options: +cmd
.			59782	IN	NS	a.root-servers.net.
.			59782	IN	NS	b.root-servers.net.
.			59782	IN	NS	c.root-servers.net.
.			59782	IN	NS	d.root-servers.net.
.			59782	IN	NS	e.root-servers.net.
.			59782	IN	NS	f.root-servers.net.
.			59782	IN	NS	g.root-servers.net.
.			59782	IN	NS	h.root-servers.net.
.			59782	IN	NS	i.root-servers.net.
.			59782	IN	NS	j.root-servers.net.
.			59782	IN	NS	k.root-servers.net.
.			59782	IN	NS	l.root-servers.net.
.			59782	IN	NS	m.root-servers.net.
;; Received 228 bytes from 8.8.8.8#53(8.8.8.8) in 248 ms

com.			172800	IN	NS	c.gtld-servers.net.
com.			172800	IN	NS	f.gtld-servers.net.
com.			172800	IN	NS	i.gtld-servers.net.
com.			172800	IN	NS	j.gtld-servers.net.
com.			172800	IN	NS	b.gtld-servers.net.
com.			172800	IN	NS	g.gtld-servers.net.
com.			172800	IN	NS	d.gtld-servers.net.
com.			172800	IN	NS	l.gtld-servers.net.
com.			172800	IN	NS	k.gtld-servers.net.
com.			172800	IN	NS	a.gtld-servers.net.
com.			172800	IN	NS	h.gtld-servers.net.
com.			172800	IN	NS	m.gtld-servers.net.
com.			172800	IN	NS	e.gtld-servers.net.
;; Received 491 bytes from 192.5.5.241#53(192.5.5.241) in 24 ms

baidu.com.		172800	IN	NS	ns2.baidu.com.
baidu.com.		172800	IN	NS	ns3.baidu.com.
baidu.com.		172800	IN	NS	ns4.baidu.com.
baidu.com.		172800	IN	NS	ns1.baidu.com.
baidu.com.		172800	IN	NS	ns7.baidu.com.
;; Received 201 bytes from 192.5.6.30#53(192.5.6.30) in 399 ms

www.baidu.com.		1200	IN	CNAME	www.a.shifen.com.
a.shifen.com.		1200	IN	NS	ns5.a.shifen.com.
a.shifen.com.		1200	IN	NS	ns2.a.shifen.com.
a.shifen.com.		1200	IN	NS	ns3.a.shifen.com.
a.shifen.com.		1200	IN	NS	ns4.a.shifen.com.
a.shifen.com.		1200	IN	NS	ns1.a.shifen.com.
;; Received 228 bytes from 180.76.76.92#53(180.76.76.92) in 41 ms
\end{lstlisting}

Baidu has a quite normal hierarchy: The domain name resolve process starts from 13 root DNSs, and then some Generic Top Level Domains (those gtld-servers.net.), and then Baidu's internal DNSs; finally, the shifen.com is Baidu's advertisement managements system. Finally we got a normal IP as 180.76.76.92.

What about Twitter? we may see that after the DNSs and GTLDs, no requests were posted to Twitter's internal network. And the IP we got is also very weird: 69.171.232.21 seems belonging to facebook.com! Why?

Well, it's a very practical technique that resolve domain names to wrong IPs could make some domain ``unreachable". And what if we resolve those requests to those IP that I want to ban too, we can shoot two birds down in one fell swoop! Great.

Here's the full list that normal requests to the DNS would be incorrectly resolves to, most of them belongs to FaceBook and Google:

\begin{lstlisting}[numberstyle=\tiny\monaco,
        basicstyle=\small\monaco]
8.7.198.45
31.13.64.1
31.13.64.33
31.13.64.49
31.13.65.1
31.13.65.17
31.13.65.18
31.13.66.1
31.13.66.6
31.13.66.23
31.13.68.1
31.13.68.22
31.13.69.33
31.13.69.86
31.13.69.129
31.13.69.160
31.13.70.1
31.13.70.20
31.13.71.7
31.13.71.23
31.13.72.1
31.13.72.17
31.13.72.23
31.13.72.34
31.13.72.54
31.13.73.1
31.13.73.17
31.13.73.23
31.13.74.1
31.13.74.17
31.13.75.17
31.13.75.18
31.13.76.8
31.13.76.16
31.13.77.33
31.13.77.55
31.13.78.65
31.13.78.66
31.13.79.1
31.13.79.17
31.13.80.1
31.13.80.17
31.13.81.1
31.13.81.17
31.13.82.1
31.13.82.17
31.13.82.23
31.13.83.1
31.13.83.8
31.13.83.16
31.13.84.1
31.13.84.8
31.13.84.16
31.13.85.1
31.13.85.8
31.13.85.16
31.13.86.1
31.13.86.8
31.13.86.16
31.13.97.245
31.13.97.248
46.82.174.68
59.24.3.173
64.13.192.74
64.13.192.76
64.13.232.149
66.220.146.94
66.220.147.11
66.220.147.44
66.220.147.47
66.220.149.18
66.220.149.32
66.220.149.99
66.220.151.20
66.220.152.17
66.220.152.28
66.220.155.12
66.220.155.14
66.220.158.32
67.15.100.252
67.15.129.210
67.228.37.26
67.228.74.123
67.228.102.32
67.228.126.62
67.228.221.221
67.228.235.91
67.228.235.93
69.63.176.15
69.63.176.59
69.63.176.143
69.63.178.13
69.63.180.173
69.63.181.11
69.63.181.12
69.63.184.14
69.63.184.30
69.63.184.142
69.63.186.30
69.63.186.31
69.63.187.12
69.63.189.16
69.63.190.26
69.171.224.12
69.171.224.40
69.171.224.85
69.171.225.13
69.171.227.37
69.171.228.20
69.171.228.74
69.171.229.11
69.171.229.28
69.171.229.73
69.171.230.18
69.171.232.21
69.171.233.24
69.171.233.33
69.171.233.37
69.171.234.18
69.171.234.29
69.171.234.48
69.171.235.16
69.171.235.64
69.171.235.101
69.171.237.16
69.171.237.26
69.171.239.11
69.171.240.27
69.171.242.11
69.171.242.30
69.171.244.11
69.171.244.12
69.171.244.15
69.171.245.49
69.171.245.53
69.171.245.84
69.171.246.9
69.171.247.20
69.171.247.32
69.171.247.71
69.171.248.65
69.171.248.112
69.171.248.128
74.86.3.208
74.86.12.172
74.86.12.173
74.86.17.48
74.86.118.24
74.86.142.55
74.86.151.162
74.86.151.167
74.86.226.234
74.86.228.110
74.86.235.236
75.126.2.43
75.126.33.156
75.126.115.192
75.126.124.162
75.126.135.131
75.126.150.210
75.126.164.178
75.126.215.88
78.16.49.15
88.191.249.182
88.191.249.183
88.191.253.157
93.46.8.89
173.252.73.48
173.252.100.21
173.252.100.32
173.252.102.16
173.252.102.241
173.252.103.64
173.252.110.21
174.36.196.242
174.36.228.136
174.37.54.20
174.37.154.236
174.37.175.229
199.16.156.7
199.16.156.40
199.16.158.190
199.59.148.14
199.59.148.97
199.59.148.140
199.59.148.209
199.59.149.136
199.59.149.244
199.59.150.11
199.59.150.49
205.186.152.122
208.43.170.231
208.43.237.140
208.101.21.43
208.101.48.171
208.101.60.87
243.185.187.39
\end{lstlisting}

May you rest in peace.

\end{document} 