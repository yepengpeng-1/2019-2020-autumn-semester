% Options for packages loaded elsewhere
\PassOptionsToPackage{unicode}{hyperref}
\PassOptionsToPackage{hyphens}{url}
%
\documentclass[
]{article}
\usepackage{lmodern}
\usepackage{amssymb,amsmath}
\usepackage{ifxetex,ifluatex}
\ifnum 0\ifxetex 1\fi\ifluatex 1\fi=0 % if pdftex
  \usepackage[T1]{fontenc}
  \usepackage[utf8]{inputenc}
  \usepackage{textcomp} % provide euro and other symbols
\else % if luatex or xetex
  \usepackage{unicode-math}
  \defaultfontfeatures{Scale=MatchLowercase}
  \defaultfontfeatures[\rmfamily]{Ligatures=TeX,Scale=1}
\fi
% Use upquote if available, for straight quotes in verbatim environments
\IfFileExists{upquote.sty}{\usepackage{upquote}}{}
\IfFileExists{microtype.sty}{% use microtype if available
  \usepackage[]{microtype}
  \UseMicrotypeSet[protrusion]{basicmath} % disable protrusion for tt fonts
}{}
\makeatletter
\@ifundefined{KOMAClassName}{% if non-KOMA class
  \IfFileExists{parskip.sty}{%
    \usepackage{parskip}
  }{% else
    \setlength{\parindent}{0pt}
    \setlength{\parskip}{6pt plus 2pt minus 1pt}}
}{% if KOMA class
  \KOMAoptions{parskip=half}}
\makeatother
\usepackage{xcolor}
\IfFileExists{xurl.sty}{\usepackage{xurl}}{} % add URL line breaks if available
\IfFileExists{bookmark.sty}{\usepackage{bookmark}}{\usepackage{hyperref}}
\hypersetup{
  hidelinks,
  pdfcreator={LaTeX via pandoc}}
\urlstyle{same} % disable monospaced font for URLs
\setlength{\emergencystretch}{3em} % prevent overfull lines
\providecommand{\tightlist}{%
  \setlength{\itemsep}{0pt}\setlength{\parskip}{0pt}}
\setcounter{secnumdepth}{-\maxdimen} % remove section numbering

\date{}

\begin{document}

\hypertarget{header-n0}{%
\section{Nov 28 Thu}\label{header-n0}}

\hypertarget{header-n2}{%
\subsection{SE-344::CG}\label{header-n2}}

\hypertarget{header-n3}{%
\subsubsection{擬合的方式}\label{header-n3}}

\hypertarget{header-n4}{%
\paragraph{插值}\label{header-n4}}

上回講過的,多種多樣的插值方法。還有兩種 N 階的方法。

\hypertarget{header-n6}{%
\paragraph{逼近樣條}\label{header-n6}}

我不祈求所有的點都能經過;只要最後的 Bias 和最小就行了。

\begin{quote}
放棄局部最優,選擇全局最優。
\end{quote}

如果強行要求經過每個點,或許會造成「扭曲」的情形,導致光順性受到影響。

這種時候,光順性比起局部最優性更重要。

\hypertarget{header-n12}{%
\subsubsection{樣條曲線的表示}\label{header-n12}}

怎麼表示⋯⋯一條參數曲線?

三種\textbf{等價}的方式。

\hypertarget{header-n15}{%
\paragraph{給定一組邊界條件}\label{header-n15}}

\hypertarget{header-n16}{%
\paragraph{列出邊界條件的矩陣表示}\label{header-n16}}

\hypertarget{header-n17}{%
\paragraph{列出一組擬合函數}\label{header-n17}}

以 Hermit 三次樣條曲線為例。

\hypertarget{header-n19}{%
\subsubsection{Bezier 曲線}\label{header-n19}}

Bezier
曲線非常著名!一種以逼近為基礎的參數曲線。(採用的是上面的逼近樣條。)

\hypertarget{header-n21}{%
\paragraph{定義}\label{header-n21}}

\({\mathbf  {B}}(t)=\sum _{{i=0}}^{n}{\mathbf  {P}}_{i}{\mathbf  {b}}_{{i,n}}(t),\quad t\in [0,1]\)

即多項式

\({\mathbf  {b}}_{{i,n}}(t)={n \choose i}t^{i}(1-t)^{{n-i}},\quad i=0,\ldots n\)

又稱作\emph{n}階的\href{https://zh.wikipedia.org/w/index.php?title=伯恩斯坦多項式\&action=edit\&redlink=1}{伯恩斯坦基底多項式},同時追加定義
0\textsuperscript{0} = 1。

伯恩斯坦的調和函數:天才!\(B~i, n~(t)\) 具有非常對稱的形式。

點\textbf{P}\textsubscript{i}稱作貝茲曲線的\textbf{控制點}。\href{https://zh.wikipedia.org/wiki/多邊形}{多邊形}以帶有\href{https://zh.wikipedia.org/wiki/線}{線}的貝茲點連接而成,起始於\textbf{P}0並以\textbf{P*}n*終止,稱作\textbf{貝茲多邊形}(或\textbf{控制多邊形})。貝茲多邊形的\href{https://zh.wikipedia.org/wiki/凸包}{凸包}(convex
hull)包含有 Bezier 曲線。

\hypertarget{header-n28}{%
\paragraph{特性}\label{header-n28}}

本身曲線的波動幾乎總是小於邊界多邊形的波動。

也就是說,曲線的起伏總是要比控制多邊形小。

這個特點非常好,確保了它的形狀可控,光順性可以得到保障。

\hypertarget{header-n32}{%
\paragraph{升級}\label{header-n32}}

本來是 n 階的 Bezier 曲線,原地不動、形狀不變,突然變成了 n + 1 階?

這可能嗎?当然可能了。

通過增加控制節點,我們可以在不改變當前曲線形狀的情況下,實現對曲線更靈活的控制。

亦即,在原有的包裹多邊形的一條邊上,加入一個控制點(使得他們三點共線)。

這樣,即可以保證圖形結構不改變,同時還能更精確地控制圖形(因為多一個控制點能帶來更多的變化,這是自然的。)

\hypertarget{header-n38}{%
\paragraph{缺點}\label{header-n38}}

這個控制點一多,就非常難算⋯

況且,在控制點多時,單個控制點對整體的影響力非常之小。

\begin{quote}
人類也很難直觀想像出調整控制點所帶來的影響。
\end{quote}

\hypertarget{header-n43}{%
\subsubsection{B 樣條曲線}\label{header-n43}}

Bezier 曲線的問題就是指定點有多少,就會有多少控制點。

這樣,做局部針對性調整就很困難。

B 樣條曲線的改進就在於:提供的點數目跟實際的控制點數目------可以不同。

跟 Bezier
唯一不一樣的地方就是可以隨意選擇控制點數目,公式表示都是类似的。

\end{document}
