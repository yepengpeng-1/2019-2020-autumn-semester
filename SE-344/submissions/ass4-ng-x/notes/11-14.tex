% Options for packages loaded elsewhere
\PassOptionsToPackage{unicode}{hyperref}
\PassOptionsToPackage{hyphens}{url}
%
\documentclass[
]{article}
\usepackage{lmodern}
\usepackage{amssymb,amsmath}
\usepackage{ifxetex,ifluatex}
\ifnum 0\ifxetex 1\fi\ifluatex 1\fi=0 % if pdftex
  \usepackage[T1]{fontenc}
  \usepackage[utf8]{inputenc}
  \usepackage{textcomp} % provide euro and other symbols
\else % if luatex or xetex
  \usepackage{unicode-math}
  \defaultfontfeatures{Scale=MatchLowercase}
  \defaultfontfeatures[\rmfamily]{Ligatures=TeX,Scale=1}
\fi
% Use upquote if available, for straight quotes in verbatim environments
\IfFileExists{upquote.sty}{\usepackage{upquote}}{}
\IfFileExists{microtype.sty}{% use microtype if available
  \usepackage[]{microtype}
  \UseMicrotypeSet[protrusion]{basicmath} % disable protrusion for tt fonts
}{}
\makeatletter
\@ifundefined{KOMAClassName}{% if non-KOMA class
  \IfFileExists{parskip.sty}{%
    \usepackage{parskip}
  }{% else
    \setlength{\parindent}{0pt}
    \setlength{\parskip}{6pt plus 2pt minus 1pt}}
}{% if KOMA class
  \KOMAoptions{parskip=half}}
\makeatother
\usepackage{xcolor}
\IfFileExists{xurl.sty}{\usepackage{xurl}}{} % add URL line breaks if available
\IfFileExists{bookmark.sty}{\usepackage{bookmark}}{\usepackage{hyperref}}
\hypersetup{
  hidelinks,
  pdfcreator={LaTeX via pandoc}}
\urlstyle{same} % disable monospaced font for URLs
\setlength{\emergencystretch}{3em} % prevent overfull lines
\providecommand{\tightlist}{%
  \setlength{\itemsep}{0pt}\setlength{\parskip}{0pt}}
\setcounter{secnumdepth}{-\maxdimen} % remove section numbering

\date{}

\begin{document}

\hypertarget{header-n0}{%
\section{Nov 14 Thu}\label{header-n0}}

\hypertarget{header-n2}{%
\subsection{SE-344::CG}\label{header-n2}}

\hypertarget{header-n3}{%
\subsubsection{復習}\label{header-n3}}

上節課我們提到了⋯⋯多邊形化的方法。

希望盡可能快(高效率)。

用盡可能少的頂點數。

且最多地保留原有形狀(保形性)。

滿足上面三條的算法:自適應型多邊形化。這個有用的。

\hypertarget{header-n9}{%
\subsubsection{合併與條帶化}\label{header-n9}}

考慮到如果我們分別獨立存儲每個三角形,就會造成同樣的頂點被多次存儲。同樣的,被三角形共享的邊也會被多次存儲,不夠經濟實惠。

如果我們抽離出那些被共享的頂點和邊,然後據此來構建條帶化的三角形,就能節約空間,提升效能。

「合併與條帶化」做的事情就是這些了。

\hypertarget{header-n13}{%
\subsubsection{網格簡化與 LOD}\label{header-n13}}

網格簡化的目的:在圖形外觀(基本)不變的情況下,盡可能減少模型的頂點數量。

實際最好的情況是可以動態地調整目標頂點數量。

\hypertarget{header-n16}{%
\paragraph{LOD}\label{header-n16}}

多分辨率模型:預先計算好多種分辨率的模型。

\begin{itemize}
\item
  Remeshing
\item
  Parametric Surfaces
\item
  Subdivision Surfaces
\end{itemize}

\hypertarget{header-n25}{%
\paragraph{Vertex Clustering}\label{header-n25}}

頂點聚合。{[}Rossignac \& Borrel 93{]}

這個問題就在於,沒辦法保證下一次做聚合時,哪些點會遭到聚合,哪些又不會。

並且,無法控制頂點聚合之後,結果的細粒度是多少。

再並且,這個操作是不可逆的。聚合了就再也分不開。

\hypertarget{header-n30}{%
\paragraph{Edge Collapse}\label{header-n30}}

邊塌陷法。所謂邊塌陷,相當於兩點合併為一點。減少了一個點,減少了兩個三角形,減少了三條邊。

好處就在於,我們做一次邊塌陷,減少的點、邊、三角形數目都是固定的。

可以精細地控制最終剩餘多少點、邊、三角形。

而且,我們只要記錄下來我們把哪兩個點合併了,就還能進行
Undo。是個可逆操作。

\hypertarget{header-n35}{%
\subsubsection{動態網格簡化}\label{header-n35}}

\hypertarget{header-n36}{%
\paragraph{Hoppe's 漸進網格法}\label{header-n36}}

\hypertarget{header-n37}{%
\paragraph{Garland 和 Heckbert 的方法}\label{header-n37}}

利用「代價函數」來改進。代價函數定義為平面與該頂點新位置之間距離的平方和。

通過定量的代價函數,我們就可以嚴格地進行動態網格簡化。

\hypertarget{header-n40}{%
\paragraph{基於保持表面特徵不變的代價函數}\label{header-n40}}

人類的觀察是很複雜的。要保持其表面特徵,很難量化地表述。

Linston \& Turk
提出了一種可以量化「相似性」的算法,可以定量地衡量走樣的程度。

\hypertarget{header-n43}{%
\subsubsection{Transformation}\label{header-n43}}

I love transformation!

OpenGL has provided much APIs, and it could be implemented quite easily.

However, what does the transformation APIs do indeed? What should we do
if we're going to implement it by ourselves?

\hypertarget{header-n47}{%
\paragraph{Geometric Transformation}\label{header-n47}}

Changing Shapes' position, size, rotation to implement animation,
design, and layout.

\hypertarget{header-n49}{%
\paragraph{Animations}\label{header-n49}}

As human eye are not functionally good, it can't separatedly tell the
segmentation of frames when the FPS (Frame per Second) is about over
than 24fps. Human eyes sucked at all.

So we can trick our eyes, constructing animation at low costs.

\hypertarget{header-n52}{%
\paragraph{Coordinate}\label{header-n52}}

Coordinate: 所謂坐標系。Commonly accepted coordinate includes:

\begin{itemize}
\item
  \textbf{Modeling Coordinate}, created by the main object (model) as
  the central of the object. 
\item
  \textbf{World Coordinate}, created when the world was created.
\item
  \textbf{Vision Coordinate}, transformed from the world coordinate.
  That could be used for the viewpoint's generation.
\end{itemize}

\hypertarget{header-n61}{%
\paragraph{Rotation}\label{header-n61}}

That's simple. Give me the base point \((x_0, y_0)\) and the rotation
angle \(\theta\).

The transformed point would be \((x_0\cos\theta, y_0\cos\theta)\)。

It's very easy to conclude under the polar coordinate.

If the rotation operation was referenced as another point, we just need
to move the coordinate to the reference point, do the rotation, and then
move back the coordinate.

\hypertarget{header-n66}{%
\paragraph{Scaling}\label{header-n66}}

Under 2D coordinate, the transformed position can be calculated by:

\(x' = x \times s_x\), \(y' = y \times s_y\).

If we want to scale by another base point, just do as above!

\begin{itemize}
\item
  Move the object to another base point
\item
  Scale
\item
  Move it back
\end{itemize}

That's it.

\hypertarget{header-n78}{%
\paragraph{2D to 3D}\label{header-n78}}

It's getting more complex. Why?

Moving and Scaling are nothing different. However, here's the problem:

The rotation stuff 需要指定在哪個平面上進行旋轉。

或者说,绕着那根轴进行旋转。

二维的旋转只要认定按照自心坐标系的原点转,就一定只有一种旋转方法。你总是转不出纸面外面。

很遗憾,三维下不是这样了。

\hypertarget{header-n85}{%
\subsection{Inconsistency}\label{header-n85}}

在笛卡兒坐標系下,矩陣變換的平移、縮放和旋轉的形式不統一。

因此我们希望能够实现统一化,这会让我们在实现的过程中少很多麻烦事。

俗话说得好,低维下不自然、不一致,那就升维啊!

平移用三维矩阵不好描述,那就直接引入第四维「平移维」啊!

当然,假如我们要做的事情和平移没有关系,那么我们直接忽略第四维,将其退化到我们之前的三维情况就好了。

Easy。

\end{document}
