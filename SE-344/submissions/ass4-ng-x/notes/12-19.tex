% Options for packages loaded elsewhere
\PassOptionsToPackage{unicode}{hyperref}
\PassOptionsToPackage{hyphens}{url}
%
\documentclass[
]{article}
\usepackage{lmodern}
\usepackage{amssymb,amsmath}
\usepackage{ifxetex,ifluatex}
\ifnum 0\ifxetex 1\fi\ifluatex 1\fi=0 % if pdftex
  \usepackage[T1]{fontenc}
  \usepackage[utf8]{inputenc}
  \usepackage{textcomp} % provide euro and other symbols
\else % if luatex or xetex
  \usepackage{unicode-math}
  \defaultfontfeatures{Scale=MatchLowercase}
  \defaultfontfeatures[\rmfamily]{Ligatures=TeX,Scale=1}
\fi
% Use upquote if available, for straight quotes in verbatim environments
\IfFileExists{upquote.sty}{\usepackage{upquote}}{}
\IfFileExists{microtype.sty}{% use microtype if available
  \usepackage[]{microtype}
  \UseMicrotypeSet[protrusion]{basicmath} % disable protrusion for tt fonts
}{}
\makeatletter
\@ifundefined{KOMAClassName}{% if non-KOMA class
  \IfFileExists{parskip.sty}{%
    \usepackage{parskip}
  }{% else
    \setlength{\parindent}{0pt}
    \setlength{\parskip}{6pt plus 2pt minus 1pt}}
}{% if KOMA class
  \KOMAoptions{parskip=half}}
\makeatother
\usepackage{xcolor}
\IfFileExists{xurl.sty}{\usepackage{xurl}}{} % add URL line breaks if available
\IfFileExists{bookmark.sty}{\usepackage{bookmark}}{\usepackage{hyperref}}
\hypersetup{
  hidelinks,
  pdfcreator={LaTeX via pandoc}}
\urlstyle{same} % disable monospaced font for URLs
\setlength{\emergencystretch}{3em} % prevent overfull lines
\providecommand{\tightlist}{%
  \setlength{\itemsep}{0pt}\setlength{\parskip}{0pt}}
\setcounter{secnumdepth}{-\maxdimen} % remove section numbering

\date{}

\begin{document}

\hypertarget{header-n0}{%
\section{Dec 19 Thu}\label{header-n0}}

\hypertarget{header-n2}{%
\subsection{SE-344::CG}\label{header-n2}}

\hypertarget{header-n3}{%
\subsubsection{真实}\label{header-n3}}

什么是真实?什么可以怀疑?

你不可以怀疑你在怀疑这件事情本身。------XYB \& DKE

\begin{quote}
DKE := 笛卡尔
\end{quote}

为了使得我们的渲染结果更为真实\ldots 我们引入了光和反射。这两件事能使得我们的结果显得相当真实。

\begin{center}\rule{0.5\linewidth}{\linethickness}\end{center}

\hypertarget{header-n10}{%
\subsubsection{反射}\label{header-n10}}

\hypertarget{header-n11}{%
\paragraph{BRDF 反射模型}\label{header-n11}}

什么是 BRDF 啊?

\begin{quote}
\textbf{雙向反射分布函數}(\textbf{bidirectional reflectance
distribution function}、\textbf{BRDF})。
\end{quote}

BRDF
是一個定義\href{https://zh.wikipedia.org/wiki/光}{光線}在不透明表反射的四次元\href{https://zh.wikipedia.org/wiki/函數}{函數},基本式為:\({\displaystyle {f_{r}(\omega _{i},\omega _{r})\ }}\),在這裡
\({\displaystyle \omega _{i}\ }\)
是指\href{https://zh.wikipedia.org/wiki/光線}{光線}的反方向,另外
\({\displaystyle \omega _{r}\ }\)
是指光線\href{https://zh.wikipedia.org/wiki/反射}{反射}的方向,除此之外,還有一個
\({\displaystyle \mathbf {n} }\)
代表\href{https://zh.wikipedia.org/wiki/法线}{法線},這個值的意義是在
\({\displaystyle \omega _{\text{r}}}\)方向的反射光線的
\href{https://zh.wikipedia.org/wiki/辐射率}{輻射率} 和同一點上 從
\({\displaystyle \omega _{\text{i}}}\)
方向射入的光線的\href{https://zh.wikipedia.org/wiki/辐射率}{輻射率}的比值。每一個
\(\omega \)
方向可以被\href{https://zh.wikipedia.org/w/index.php?title=参数化\&action=edit\&redlink=1}{參數化}
為 \href{https://zh.wikipedia.org/wiki/方位角}{方位角} \(\phi\) 和
\href{https://zh.wikipedia.org/wiki/天頂角}{天頂角}
\(\theta\)。因此BRDF是一個四維函數。 BRDF的單位是
sr\textsuperscript{−1}, 其中 (sr) 是
\href{https://zh.wikipedia.org/wiki/球面度}{球面度}的單位.

假如我们不考虑「实际上的光线可能来自多个方向」,那么我们就可以去掉这个二维积分,仅考虑一个方向上的入射光。这样
BRDF 的方程就简化为一般的反射方向,加上一个四次参数。

确定了这个参数,就知道了 BRDF 参数。

\begin{center}\rule{0.5\linewidth}{\linethickness}\end{center}

\hypertarget{header-n19}{%
\subparagraph{测量}\label{header-n19}}

我们可以通过简单测量的办法来确认不同材料表面的 BRDF 参数。

\hypertarget{header-n21}{%
\subparagraph{理论模型}\label{header-n21}}

通过物理计算来确定这四个参数\ldots(逃

主要的纠结点有:镜面叶的方向、汇聚系数、考虑菲涅耳现象与否。

\hypertarget{header-n24}{%
\subparagraph{基于图像}\label{header-n24}}

通过我们真实世界拍摄的图像来「学习」参数\ldots{}

\hypertarget{header-n26}{%
\subparagraph{做伸手党}\label{header-n26}}

抄现成的参数(

\hypertarget{header-n28}{%
\subsubsection{折射}\label{header-n28}}

考虑一下:在我们的光经过一个非全不透明的物体的时候,我们的光不仅会按照上面的规律反射,还可能会按照折射定律
(\(m sin\alpha = n sin\beta\))产生折射\ldots{}

\hypertarget{header-n30}{%
\subsubsection{光影}\label{header-n30}}

这里的部分都已经是非常细枝末节的部分了。

阴影是个很要紧的事情:在出现阴影的时候,我们就能更直观地了解到物体在空间中的位置关系,也是增强真实感的一点细节吧。

\hypertarget{header-n33}{%
\paragraph{反射}\label{header-n33}}

反射应该是最容易实现的了:直接对着世界坐标系做变换,然后直接对着平面做一个反射就好了。

\hypertarget{header-n35}{%
\paragraph{预防反射失真}\label{header-n35}}

过于「完美」的反射会适得其反:倒影怎么会跟原像一模一样呢?难道没有反射亮度衰减和微小的形变吗?

所以我们对着平面增加了一个
Mask:这个遮罩会阻碍(部分)反射,模拟出影影绰绰的感觉。

\hypertarget{header-n38}{%
\paragraph{硬阴影 \& 软阴影}\label{header-n38}}

一个有一定宽度的光源和有一定宽度的遮挡物,一定存在一个类似于「单缝干涉」的亮度波形样。

我们称:从该定长宽度光源的任何一处发出的光都无法到达的位置称为「遮挡物」的本影区,这部分就称为硬阴影。而那些只有一些部分发出的光能到达,有一些部分被遮挡的范围称为「软阴影」。总是可达的位置则不称为阴影。

\hypertarget{header-n41}{%
\subsubsection{纹理}\label{header-n41}}

为了用最最低的成本来实现某种大量规律性的、背景性的画面效果\ldots{}

\ldots 就直接贴图好了。效果也差不多嘛(×)

材质细节、光照细节、几何细节全部都可以直接贴图解决了。

但是,要贴得好看、贴得漂亮,我们还是需要费一番功夫的。

\hypertarget{header-n46}{%
\paragraph{原理}\label{header-n46}}

将纹理模式映射到物体模型表面,以便模拟物体表面材质细节、光照细节和几何细节的这个过程就称为「纹理映射」(Texture
Mapping)。

\begin{quote}
纹理并不一定是二维的。待会可以看到也存在一维和三维的纹理。
\end{quote}

但我们先以(最常用的)二维纹理空间为例。

这种纹理一般定义在单位正方形域中。

\hypertarget{header-n52}{%
\paragraph{描述}\label{header-n52}}

纹理可以使用函数来描述:类似于 \(g(s, t)\) 这样的双参数函数。

当然,一张位图也可以作为纹理使用。

\hypertarget{header-n55}{%
\paragraph{映射}\label{header-n55}}

我们现在有了纹理表示了;该怎么映射到物体上面呢?

\begin{itemize}
\item
  纹理扫描

  直接将纹理模式「糊」到物体表面,然后交给後面的干活。

  但是这样耗费高,而且可能在后面进行投影变换後,我们的纹理就变形了。
\item
  像素次序扫描

  先将投影平面的像素区域映射到物体表面(做逆向投影变换),再对这些像素进行扫描。

  这就可以保证我们的纹理不会再最终结果中产生变形,且能保证最终的纹理结果均匀。

  这个方法,好!我们要多用啊。
\end{itemize}

\begin{center}\rule{0.5\linewidth}{\linethickness}\end{center}

重大问题:你这个,「逆向投影变换」是咋做的啊?

我们还得先看投影函数是怎么做的。

\(P(x, y, z)\) =\textgreater{} \((s, t)\)。但这怎么才能反向确定呢?

\begin{itemize}
\item
  方法一:在建模的时候就提供一个不参与投影的「贴图坐标」,这样就能保留一一对应信息。

  \begin{itemize}
  \item
    问题:消耗太大。我们用不起。
  \end{itemize}
\item
  方法二:结合简单片元形状反推。

  \begin{itemize}
  \item
    这种问题就在于不能自由调整贴图的方向(只能沿着 \((u, v)\)
    的方向来),而且计算量大。
  \end{itemize}
\item
  方法三:两步法。

  \begin{itemize}
  \item
    基本思想:先把纹理给贴到简单的 3D
    表面上(类似于平面、球面、圆柱面、立方体面等等),称之为中介面。中介面应该完全在目的物体外围。
  \item
    这个计算相对方便(因为物体简单啊)。这个过程称为 S Mapping。
  \item
    然后,再把纹理从中介面上再贴到实际的物体上,称为 O Mapping。
  \end{itemize}
\end{itemize}

\end{document}
