% Options for packages loaded elsewhere
\PassOptionsToPackage{unicode}{hyperref}
\PassOptionsToPackage{hyphens}{url}
%
\documentclass[
]{article}
\usepackage{lmodern}
\usepackage{amssymb,amsmath}
\usepackage{ifxetex,ifluatex}
\ifnum 0\ifxetex 1\fi\ifluatex 1\fi=0 % if pdftex
  \usepackage[T1]{fontenc}
  \usepackage[utf8]{inputenc}
  \usepackage{textcomp} % provide euro and other symbols
\else % if luatex or xetex
  \usepackage{unicode-math}
  \defaultfontfeatures{Scale=MatchLowercase}
  \defaultfontfeatures[\rmfamily]{Ligatures=TeX,Scale=1}
\fi
% Use upquote if available, for straight quotes in verbatim environments
\IfFileExists{upquote.sty}{\usepackage{upquote}}{}
\IfFileExists{microtype.sty}{% use microtype if available
  \usepackage[]{microtype}
  \UseMicrotypeSet[protrusion]{basicmath} % disable protrusion for tt fonts
}{}
\makeatletter
\@ifundefined{KOMAClassName}{% if non-KOMA class
  \IfFileExists{parskip.sty}{%
    \usepackage{parskip}
  }{% else
    \setlength{\parindent}{0pt}
    \setlength{\parskip}{6pt plus 2pt minus 1pt}}
}{% if KOMA class
  \KOMAoptions{parskip=half}}
\makeatother
\usepackage{xcolor}
\IfFileExists{xurl.sty}{\usepackage{xurl}}{} % add URL line breaks if available
\IfFileExists{bookmark.sty}{\usepackage{bookmark}}{\usepackage{hyperref}}
\hypersetup{
  hidelinks,
  pdfcreator={LaTeX via pandoc}}
\urlstyle{same} % disable monospaced font for URLs
\setlength{\emergencystretch}{3em} % prevent overfull lines
\providecommand{\tightlist}{%
  \setlength{\itemsep}{0pt}\setlength{\parskip}{0pt}}
\setcounter{secnumdepth}{-\maxdimen} % remove section numbering

\date{}

\begin{document}

\hypertarget{header-n0}{%
\section{Dec 12 Thu}\label{header-n0}}

\hypertarget{header-n2}{%
\subsection{SE-344::CG}\label{header-n2}}

\hypertarget{header-n3}{%
\subsubsection{光照的计算}\label{header-n3}}

光照计算,有什么意义吗?

真实的世界里,光线是可以藉由整个世界的物体反射、晕染,等等。

如果我们的光不能产生反射,就无法真实地模拟出真实世界的复杂光线条件。

\hypertarget{header-n7}{%
\subsubsection{光的种类}\label{header-n7}}

\hypertarget{header-n8}{%
\paragraph{环境光}\label{header-n8}}

何为环境光?

是指在每个方面上的反射光线为常数,而与观察方向无关。

\hypertarget{header-n11}{%
\paragraph{漫反射光}\label{header-n11}}

强度公式:\(I = kI_t\cos \theta\)。

这就是说漫反射光的强度根观察角度跟入射角度的夹角有关。

\hypertarget{header-n14}{%
\paragraph{镜面反射(Phong 模型)}\label{header-n14}}

镜面反射下的光滑物体的大部分发射光都仅仅反射到了某一个观察方向,而仅有少部分到了其他方向上。这个唤作镜面反射。

\begin{quote}
理想的镜面反射是会将所有的光线都按照反射定律反射出去,而完全不产生散射的。

然而实际上并不存在完美光滑的表面,因此这样的镜面反射也是不存在的。
\end{quote}

Phong
模型说:反射的光线会被限定在一个光锥里,光锥整体指向理想反射的方向。

越光滑的表面,光锥半径越小,反射光线越集中;越粗糙的表面,光锥半径越大,反射光线越松散。

\(I_{spec} = I(\theta) \cos(n\phi)\)

\hypertarget{header-n22}{%
\paragraph{菲涅尔现象}\label{header-n22}}

在接近 90 度时,反射光强度会突然增大。

(想象直视太阳的感觉。)

有一些优秀的反射模型会考虑这种情形。

\hypertarget{header-n26}{%
\paragraph{多个光源}\label{header-n26}}

多个光源出现的情况下,每个点的光线强度是每个光线强度分量的矢量加和。

\hypertarget{header-n28}{%
\subsubsection{光的衰减}\label{header-n28}}

最简单的衰减函数:一个二次多项式的倒数。

据说这样的衰减函数效果已经很好。

经过的光程越多,衰减越大。

\begin{quote}
物理原理:在大气之中,经过的距离越长,空气中微尘造成的耗散越多。
\end{quote}

\hypertarget{header-n34}{%
\subsubsection{透明度}\label{header-n34}}

考虑一个透明的物体。

我们利用一个「透明度」\(\alpha\)
来衡量一个物体对光线的透射程度------介于 \([0, 1]\) 之间的值。

利用 \(\alpha\)
来将入射光分为两部分:一部分投射、一部分反射。两部分分别进行光线运算。

\hypertarget{header-n38}{%
\subsubsection{阴影}\label{header-n38}}

光的遮挡造成了阴影。阴影跟物体的形状、光源的方向都有关系。

但是,假如在某一个场景中,物体和光源都是静止的,那么我们实际上可以预先渲染出「阴影」这个物体,到时候直接贴上去就好了。这样能省下很多运算。

\hypertarget{header-n41}{%
\subsubsection{繪製陰影}\label{header-n41}}

\hypertarget{header-n42}{%
\paragraph{粗暴法}\label{header-n42}}

恒定光强度敏感处理。超级快

一个多边形上的点都用相同的光强度来绘制。

基于下面的假设:

\begin{itemize}
\item
  物体是一个多面体
\item
  光源离物体足够远,使得物体表面各处对该光线的衰减程度几乎类似
\item
  视点离物体足够远,使得 V × R 对于物体表面各点几乎类似
\end{itemize}

在这几条不满足的情况下,就拆分多面体成多个多边形,事情就简单了。

为了将一整个多边形压平成一个平面,我们计算一个多边形的平均法矢量来作为「压平」后平面的法矢量。

\hypertarget{header-n55}{%
\paragraph{马赫带效应(Mach-band Effect)}\label{header-n55}}

表面上的高光有时候会出现异常的情况;现行光线差值会造成表面过亮或过暗的条纹。

这是人类视觉的错觉。

\hypertarget{header-n58}{%
\subsubsection{OpenGL APIs}\label{header-n58}}

\hypertarget{header-n59}{%
\paragraph{定义法矢量}\label{header-n59}}

为了画出好看的阴影效果,在我们调用 \texttt{glVertex} 之前一定要先调用
\texttt{glNormal} 来指定顶点的法矢量,这样光线才能正确计算反射和阴影。

假如我们按照之前的路数,不提供法矢量方向,那就会直接糊成一团。

\hypertarget{header-n62}{%
\paragraph{创建光源}\label{header-n62}}

用 \texttt{glLightfv} 来指定光源的参数。

可以指定环境光颜色、漫反射光颜色、镜面光颜色、光的位置。

在此之後,我们用 \texttt{glEnable} 打开这个光源。

这还不够,还需要用 \texttt{glEnable(LIGHTINGS)} 打开光线渲染才行。

\hypertarget{header-n67}{%
\paragraph{聚光源}\label{header-n67}}

给光线设定 \texttt{GL\_SPOT\_EXPONENT}
参数,这个参数会使得光线不会平均向各个方向发光,而是聚集在目标方向上的一定范围内。

调整这个参数的值,就能实现聚光的效果了。

\hypertarget{header-n70}{%
\subparagraph{模型属性}\label{header-n70}}

通过 \texttt{glColorMaterial}
可以调整场景物体的材质,使得他们在不同的光线种类下产生不同的响应。

\end{document}
