% Options for packages loaded elsewhere
\PassOptionsToPackage{unicode}{hyperref}
\PassOptionsToPackage{hyphens}{url}
%
\documentclass[
]{article}
\usepackage{lmodern}
\usepackage{amssymb,amsmath}
\usepackage{ifxetex,ifluatex}
\ifnum 0\ifxetex 1\fi\ifluatex 1\fi=0 % if pdftex
  \usepackage[T1]{fontenc}
  \usepackage[utf8]{inputenc}
  \usepackage{textcomp} % provide euro and other symbols
\else % if luatex or xetex
  \usepackage{unicode-math}
  \defaultfontfeatures{Scale=MatchLowercase}
  \defaultfontfeatures[\rmfamily]{Ligatures=TeX,Scale=1}
\fi
% Use upquote if available, for straight quotes in verbatim environments
\IfFileExists{upquote.sty}{\usepackage{upquote}}{}
\IfFileExists{microtype.sty}{% use microtype if available
  \usepackage[]{microtype}
  \UseMicrotypeSet[protrusion]{basicmath} % disable protrusion for tt fonts
}{}
\makeatletter
\@ifundefined{KOMAClassName}{% if non-KOMA class
  \IfFileExists{parskip.sty}{%
    \usepackage{parskip}
  }{% else
    \setlength{\parindent}{0pt}
    \setlength{\parskip}{6pt plus 2pt minus 1pt}}
}{% if KOMA class
  \KOMAoptions{parskip=half}}
\makeatother
\usepackage{xcolor}
\IfFileExists{xurl.sty}{\usepackage{xurl}}{} % add URL line breaks if available
\IfFileExists{bookmark.sty}{\usepackage{bookmark}}{\usepackage{hyperref}}
\hypersetup{
  hidelinks,
  pdfcreator={LaTeX via pandoc}}
\urlstyle{same} % disable monospaced font for URLs
\setlength{\emergencystretch}{3em} % prevent overfull lines
\providecommand{\tightlist}{%
  \setlength{\itemsep}{0pt}\setlength{\parskip}{0pt}}
\setcounter{secnumdepth}{-\maxdimen} % remove section numbering

\date{}

\begin{document}

\hypertarget{header-n0}{%
\section{Oct 29 Tue}\label{header-n0}}

\hypertarget{header-n2}{%
\subsection{SE-302::Compilers}\label{header-n2}}

继续 Translation(AST =\textgreater{} IR)。

\hypertarget{header-n4}{%
\subsubsection{Arithmetics}\label{header-n4}}

算术操作的翻译,相对比较容易。

\hypertarget{header-n6}{%
\subsubsection{Conditional}\label{header-n6}}

条件表达式\ldots 留意到我们需要支持 Shortcut(短路求值)。

\hypertarget{header-n8}{%
\paragraph{\& \textbar{}}\label{header-n8}}

处理 OR 跟 AND 的情况下,我们将其展开为 IF - THEN - ELSE
语句来实现短路操作。

在抽象语法分析阶段已经被拆成 IF - THEN - ELSE 了。因此也不用过于担心。

\hypertarget{header-n11}{%
\paragraph{If Expression}\label{header-n11}}

IF 来了!怎么处理?

\texttt{if\ e1\ then\ e2\ else\ e3}

最简单的思路就是把 \(e1\) 视为 Cx,e2 和 e3 都视为 Ex 表达式。即使 e2 跟
e3 都是不返回值的表达式(Nx),也会对其做好强制类型转换。

只不过这样效率较低(需要进行多次 unEx 和强制类型转换);

况且在 e2 和 e3 为 Cx 的情况下,进行 unEx
可能会导致转移和标号产生混乱。这是很不好的。

\hypertarget{header-n17}{%
\paragraph{If Expression Fix}\label{header-n17}}

特别识别出 e2 跟 e3 的表达式类型(Cx?
Ex?)再将其表示为对应的表达式类型。

如果是 Nx,那也不用取出返回值了(没有用的)\ldots\ldots{}

\hypertarget{header-n20}{%
\paragraph{strcmp}\label{header-n20}}

String comparison 可不能直接把头指针放到 CJUMP 里进行比较\ldots{}

我们需要逐个地比较 bytes 才能得到结果。

最终我们不去实现这个函数。调用 Runtime System Library 函数
\texttt{stringEqual} 就行了\ldots{}

P.S. string unequal 的实现得调用一回
\texttt{stringEqual},然后自己把结果反一反就得了\ldots{}

\hypertarget{header-n25}{%
\subsubsection{String Literal}\label{header-n25}}

\hypertarget{header-n26}{%
\paragraph{Assembly}\label{header-n26}}

\begin{verbatim}
.section	.data

string:
	.ascii	"hello, world\n"
string_end:
	.equ	len, string_end - string
	
.section	.text
.global		main

main:
	...
\end{verbatim}

在一般的汇编之中,字符串常量会被放在一处内存 Segment 之中;

每碰到一个 String,就增加一个 Fragment。

系统记录了每一串字符串的内容和长度。

为了索引(找到)这个字符串,我们一般使用一个
Label(标签🏷️)作为寻找字符串的指针。

\hypertarget{header-n32}{%
\paragraph{F\_string}\label{header-n32}}

\texttt{F\_string(lab,\ lit)} 可以把字符串字面量 lit (String Literal)
给放到 Data Segment 中,并且赋予其标签 lab (Label)。

Pascal 里头,字符串是固定长度的。不够长度的部分用 0 填充(这没什么用)

C 的实现是:字符串不定长,可以无限长,仅仅以 \textbackslash0
作为字符串的结束。(这个实现不错,只是 \textbackslash0
无法在字符串本体中出现,这是个问题。)

Tiger
的实现是:字符串由两部分构成;第一部分是字符串的长度;第二部分是对应长度的字节(即字符串内容)。

\hypertarget{header-n37}{%
\subsubsection{Record Creation}\label{header-n37}}

上面提到过,Record 的引用都只是用他的所谓「指针」。

事实上 Record 本体是存在 Heap 上的,而且从不释放它们。

因此不像字符串无法动态生成,Record 是需要动态生成并且在 Heap
上开辟空间的。

\hypertarget{header-n41}{%
\paragraph{Function Call}\label{header-n41}}

首先,计算出 Record 所需要占用的空间;然后开辟出这么多的内存。

\hypertarget{header-n43}{%
\paragraph{Initialization}\label{header-n43}}

随后,我们需要对此类 Record 的每一个 Entry
进行初始化。是的,没有办法进行一次完成;必须一个个写。

\hypertarget{header-n45}{%
\subsubsection{Array Creation}\label{header-n45}}

Array 的创建跟 Record 的创建绝对类似。

同样需要调用外部的内存分配函数(因为 Tiger 做不了这件事情);

同样需要一个个地对每个参数进行初始化。(难受)

\hypertarget{header-n49}{%
\subsubsection{While / For Loop}\label{header-n49}}

怎么翻译循环呢?

\hypertarget{header-n51}{%
\paragraph{While 循环}\label{header-n51}}

基本语法:

\begin{verbatim}
while (condition)
	do body
\end{verbatim}

翻译一下,是这样的:

\begin{verbatim}
test:
	if not(condition) goto done
	body
	goto test
done:
\end{verbatim}

\hypertarget{header-n56}{%
\paragraph{Break 中断}\label{header-n56}}

但是留意一下有一个特别的东西需要特别处理:\texttt{break}。

为什么就一个呢?因为\ldots Tiger 没有 \texttt{continue}。

\texttt{break} 的意义是直接跳脱循环,直接进入离他最近的那个
\texttt{done}。

只不过因为可能存在多个递归的循环,我们要很小心不要跳错了。

因此 \texttt{transExp} 需要获取一个额外的参数:done 标号,确定了如果有
\texttt{break} 我该跳到哪里。

\hypertarget{header-n62}{%
\paragraph{For 循环}\label{header-n62}}

\begin{verbatim}
for i := lo to hi
	do body
\end{verbatim}

for 循环的基本语法是这样的。

翻译的话,可以翻译成 while 来做(本身也就是一层语法糖衣)

\begin{verbatim}
let var i := lo
	var limit := hi
in while i <= limit
	do (body; i := i + 1)
end
\end{verbatim}

翻译成这样,可以吗?

在 lo \textgreater= hi 的时候就直接炸了。在 hi = INTMAX 的时候也炸了。

解决方案也很简单;把 \textless= 改用 \textless{}
来判断,并且先进行测试再做 \texttt{i\ :=\ i\ +\ 1}。

不过既然咱们不在意这些 Corner Cases,我们也就不管了。

\hypertarget{header-n71}{%
\subsubsection{Function Call}\label{header-n71}}

\texttt{CALL(NAME\ lf,\ {[}sl,\ e1,\ e2,\ ...{]})}。就是这样的。

sl 是静态链。

\hypertarget{header-n74}{%
\subsubsection{Declaration}\label{header-n74}}

声明?分为很多种。Variable Decl, Function Decl, Type Decl...
总归很麻烦。

\hypertarget{header-n76}{%
\paragraph{Function Declaration}\label{header-n76}}

麻烦了。这样我们需要在 \texttt{.text} 中增加一部分新的汇编代码。

汇编分为好几部分:

\hypertarget{header-n79}{%
\subparagraph{Prologue}\label{header-n79}}

标准化的入口处理代码。

首先要用

\begin{verbatim}
.text

	...
	
.type	incr, @function
\end{verbatim}

来指明这是一个名为 \texttt{incr} 的函数正在被定义。

随后,我们需要调整 Stack 指针(IA32
需要同时调整栈顶指针和栈底指针;AMD64 只需要调整栈底),

以便存放我们开在栈上的临时变量。

之后,还需要分别处理 \texttt{@escaping} 和 \texttt{non-escaping}
的参数,将其分别放到栈帧里和寄存器里。

最后,我们只需要保存 Callee-saved 寄存器的内容入栈就好了。

\hypertarget{header-n88}{%
\subparagraph{Body}\label{header-n88}}

Body 就是 Compiler 按照函数体内的代码翻译出来的。

\hypertarget{header-n90}{%
\subparagraph{Epilogue}\label{header-n90}}

出口处理代码也是很标准化的,主要包括如下部分:

\begin{itemize}
\item
  将返回值放到返回结果寄存器(\%rax)的指令。
\item
  恢复 Callee-saved 寄存器的内容(之前是入栈了的,现在弹出来)。
\item
  恢复栈指针(跟上面一样反着操作)。
\item
  一条 return 指令(JUMP 到该返回的地址)。
\item
  汇编里面一条特殊指令(Psuedo Command),标识函数的结束。
\end{itemize}

\hypertarget{header-n103}{%
\subparagraph{Fragment}\label{header-n103}}

跟每遇到一个 String 都要做一个 Fragment 一样,函数也会生成一个 Function
Fragment List。

\hypertarget{header-n105}{%
\paragraph{Variable Declaration}\label{header-n105}}

\hypertarget{header-n106}{%
\subsubsection{Basic Blocks \& Traces}\label{header-n106}}

\begin{quote}
基本块与轨迹
\end{quote}

Tree 和机器语言也不可以直接转换,也存在很多 Conflict 的情况。

比如,CJUMP 存在 True/False
的两种跳转分支,而实际的机器是跳/不跳,转下一条指令的区分。

CALL 中 CALL 会带来问题。在 Tree 语言里仍然保留着 f(g(x))
形式的代码;但是在实际机器码中这肯定是不被允许的。

ESEQ 也会带来问题;计算顺序会带来不确定的结果。

为了去除这些问题,我们会引入到一个新的 Tree:Canonical Tree。

\hypertarget{header-n114}{%
\subsubsection{Canonical Tree}\label{header-n114}}

P.S. Canonical Tree 跟原来的 TreeLang
的结构完全一致,只不过她会做这样的修正:⌫

\begin{enumerate}
\def\labelenumi{\arabic{enumi}.}
\item
  Lineralize:线性化。去除所有的 SEQ 和
  ESEQ(确保跟汇编代码中的顺序执行样式相同);
\item
  组合成不包含转移和标号的基本块(Basic Blocks)集合。
\item
  将基本块集合组合成一组轨迹(Trace)。
\end{enumerate}

\end{document}
