% Options for packages loaded elsewhere
\PassOptionsToPackage{unicode}{hyperref}
\PassOptionsToPackage{hyphens}{url}
%
\documentclass[
]{article}
\usepackage{lmodern}
\usepackage{amssymb,amsmath}
\usepackage{ifxetex,ifluatex}
\ifnum 0\ifxetex 1\fi\ifluatex 1\fi=0 % if pdftex
  \usepackage[T1]{fontenc}
  \usepackage[utf8]{inputenc}
  \usepackage{textcomp} % provide euro and other symbols
\else % if luatex or xetex
  \usepackage{unicode-math}
  \defaultfontfeatures{Scale=MatchLowercase}
  \defaultfontfeatures[\rmfamily]{Ligatures=TeX,Scale=1}
\fi
% Use upquote if available, for straight quotes in verbatim environments
\IfFileExists{upquote.sty}{\usepackage{upquote}}{}
\IfFileExists{microtype.sty}{% use microtype if available
  \usepackage[]{microtype}
  \UseMicrotypeSet[protrusion]{basicmath} % disable protrusion for tt fonts
}{}
\makeatletter
\@ifundefined{KOMAClassName}{% if non-KOMA class
  \IfFileExists{parskip.sty}{%
    \usepackage{parskip}
  }{% else
    \setlength{\parindent}{0pt}
    \setlength{\parskip}{6pt plus 2pt minus 1pt}}
}{% if KOMA class
  \KOMAoptions{parskip=half}}
\makeatother
\usepackage{xcolor}
\IfFileExists{xurl.sty}{\usepackage{xurl}}{} % add URL line breaks if available
\IfFileExists{bookmark.sty}{\usepackage{bookmark}}{\usepackage{hyperref}}
\hypersetup{
  hidelinks,
  pdfcreator={LaTeX via pandoc}}
\urlstyle{same} % disable monospaced font for URLs
\usepackage{color}
\usepackage{fancyvrb}
\newcommand{\VerbBar}{|}
\newcommand{\VERB}{\Verb[commandchars=\\\{\}]}
\DefineVerbatimEnvironment{Highlighting}{Verbatim}{commandchars=\\\{\}}
% Add ',fontsize=\small' for more characters per line
\newenvironment{Shaded}{}{}
\newcommand{\AlertTok}[1]{\textcolor[rgb]{1.00,0.00,0.00}{\textbf{#1}}}
\newcommand{\AnnotationTok}[1]{\textcolor[rgb]{0.38,0.63,0.69}{\textbf{\textit{#1}}}}
\newcommand{\AttributeTok}[1]{\textcolor[rgb]{0.49,0.56,0.16}{#1}}
\newcommand{\BaseNTok}[1]{\textcolor[rgb]{0.25,0.63,0.44}{#1}}
\newcommand{\BuiltInTok}[1]{#1}
\newcommand{\CharTok}[1]{\textcolor[rgb]{0.25,0.44,0.63}{#1}}
\newcommand{\CommentTok}[1]{\textcolor[rgb]{0.38,0.63,0.69}{\textit{#1}}}
\newcommand{\CommentVarTok}[1]{\textcolor[rgb]{0.38,0.63,0.69}{\textbf{\textit{#1}}}}
\newcommand{\ConstantTok}[1]{\textcolor[rgb]{0.53,0.00,0.00}{#1}}
\newcommand{\ControlFlowTok}[1]{\textcolor[rgb]{0.00,0.44,0.13}{\textbf{#1}}}
\newcommand{\DataTypeTok}[1]{\textcolor[rgb]{0.56,0.13,0.00}{#1}}
\newcommand{\DecValTok}[1]{\textcolor[rgb]{0.25,0.63,0.44}{#1}}
\newcommand{\DocumentationTok}[1]{\textcolor[rgb]{0.73,0.13,0.13}{\textit{#1}}}
\newcommand{\ErrorTok}[1]{\textcolor[rgb]{1.00,0.00,0.00}{\textbf{#1}}}
\newcommand{\ExtensionTok}[1]{#1}
\newcommand{\FloatTok}[1]{\textcolor[rgb]{0.25,0.63,0.44}{#1}}
\newcommand{\FunctionTok}[1]{\textcolor[rgb]{0.02,0.16,0.49}{#1}}
\newcommand{\ImportTok}[1]{#1}
\newcommand{\InformationTok}[1]{\textcolor[rgb]{0.38,0.63,0.69}{\textbf{\textit{#1}}}}
\newcommand{\KeywordTok}[1]{\textcolor[rgb]{0.00,0.44,0.13}{\textbf{#1}}}
\newcommand{\NormalTok}[1]{#1}
\newcommand{\OperatorTok}[1]{\textcolor[rgb]{0.40,0.40,0.40}{#1}}
\newcommand{\OtherTok}[1]{\textcolor[rgb]{0.00,0.44,0.13}{#1}}
\newcommand{\PreprocessorTok}[1]{\textcolor[rgb]{0.74,0.48,0.00}{#1}}
\newcommand{\RegionMarkerTok}[1]{#1}
\newcommand{\SpecialCharTok}[1]{\textcolor[rgb]{0.25,0.44,0.63}{#1}}
\newcommand{\SpecialStringTok}[1]{\textcolor[rgb]{0.73,0.40,0.53}{#1}}
\newcommand{\StringTok}[1]{\textcolor[rgb]{0.25,0.44,0.63}{#1}}
\newcommand{\VariableTok}[1]{\textcolor[rgb]{0.10,0.09,0.49}{#1}}
\newcommand{\VerbatimStringTok}[1]{\textcolor[rgb]{0.25,0.44,0.63}{#1}}
\newcommand{\WarningTok}[1]{\textcolor[rgb]{0.38,0.63,0.69}{\textbf{\textit{#1}}}}
\usepackage[normalem]{ulem}
% Avoid problems with \sout in headers with hyperref
\pdfstringdefDisableCommands{\renewcommand{\sout}{}}
\setlength{\emergencystretch}{3em} % prevent overfull lines
\providecommand{\tightlist}{%
  \setlength{\itemsep}{0pt}\setlength{\parskip}{0pt}}
\setcounter{secnumdepth}{-\maxdimen} % remove section numbering

\date{}

\begin{document}

\hypertarget{header-n0}{%
\section{Oct 15 Tue}\label{header-n0}}

\hypertarget{header-n2}{%
\subsection{SE-302::Compilers}\label{header-n2}}

今天我们要讲的是 Type Checking。

回忆一下我们上回讲的 Symbol Table。这个过程会生成两张符号表:venv
(variable environment) 和 tenv (type environment)。

Type Checking
要做的就是:检查表达式中出现的操作符能否结合,并计算出整个表达式的类型。(为了进行下一步计算。)

\hypertarget{header-n6}{%
\subsubsection{Tiger 里的类型}\label{header-n6}}

\begin{itemize}
\item
  \texttt{Ty\_record}
\end{itemize}

类似于 \texttt{struct}。

\begin{itemize}
\item
  \texttt{Ty\_nil}
\item
  \texttt{Ty\_int}
\end{itemize}

整型!

\begin{itemize}
\item
  \texttt{Ty\_string}
\end{itemize}

字符串!

\begin{itemize}
\item
  \texttt{Ty\_array}
\end{itemize}

数组!(P.S. Array
可能会装填着不同类型的元素,因此还需要一个额外参数来指定 Element
Type。)

\begin{itemize}
\item
  \texttt{Ty\_name}
\end{itemize}

Name 就是 Symbol 的名字。

\begin{itemize}
\item
  \texttt{Ty\_void}
\end{itemize}

大概只会在函数返回值里用。

\hypertarget{header-n33}{%
\subsubsection{Type Checking Expressions}\label{header-n33}}

Type Checking 要做的事情:

\begin{itemize}
\item
  扫描一遍代码,得到一个 Abstract Syntax Tree。
\end{itemize}

Type Checking 用黑话来说,就叫 Translation。所以下面的缩写都是从他来的。

\hypertarget{header-n39}{%
\paragraph{检查表达式}\label{header-n39}}

\begin{Shaded}
\begin{Highlighting}[]
\KeywordTok{struct}\NormalTok{ expty transExp(S_table venv, S_table tenv, A_exp a) \{}
    \ControlFlowTok{switch}\NormalTok{(a->kind) \{}
        \ControlFlowTok{case}\NormalTok{ A_opExp:}
            \ControlFlowTok{if}\NormalTok{ (}\CommentTok{/* 合法 */}\NormalTok{) \{}
                \CommentTok{// 合法的时候,构造出一个类型代表表达式返回的类型}
                \ControlFlowTok{return}\NormalTok{ expTy(NULL, Ty_Int());}
\NormalTok{            \} }\ControlFlowTok{else}\NormalTok{ \{}
                \CommentTok{// 不合法的时候,返回错误值}
\NormalTok{            \}}
            \ControlFlowTok{break}\NormalTok{;}
        \ControlFlowTok{case}\NormalTok{ A_letExp:}
            \CommentTok{// let}
            \CommentTok{//  ... decs}
            \CommentTok{// in}
            \CommentTok{//  ... body}
            \CommentTok{// end}
            
            \CommentTok{/* let 操作符需要做的事情是}
\CommentTok{               创建一个新的 environment,然后用它来}
\CommentTok{               递归地去遍历 body 内内容。}
\CommentTok{            */}
            \ControlFlowTok{break}\NormalTok{;}
        \ControlFlowTok{case} \CommentTok{/* other ops */}\NormalTok{:}
            \ControlFlowTok{break}\NormalTok{;}
\NormalTok{    \}}
\NormalTok{\}}
\end{Highlighting}
\end{Shaded}

\hypertarget{header-n41}{%
\paragraph{检查类型}\label{header-n41}}

Tiger Language 里面,有三种不同的 Variable:

\begin{itemize}
\item
  Simple Variable
\end{itemize}

简单变量就是个符号。里面又包括 varEntry(一般变量)和
funEntry(一般函数)。

\begin{itemize}
\item
  Field Variable
\end{itemize}

实参列表里的变量。

\begin{itemize}
\item
  Subscript Variable
\end{itemize}

Subscript 用的 Variable,形如 \texttt{a.i} 或是 \texttt{a{[}i{]}} 的变量
\texttt{i} 就叫做 subscript variable。

\begin{Shaded}
\begin{Highlighting}[]
\KeywordTok{struct}\NormalTok{ expty transVar(S_table venv, S_table tenv, A_var a) \{}
    \ControlFlowTok{switch}\NormalTok{ (v->kind) \{}
        \ControlFlowTok{case}\NormalTok{ A_SimpleVar : \{}
            \CommentTok{// 简单变量,先从符号表里找这个}
            \KeywordTok{auto}\NormalTok{ x = S_look(venv, v->u.simple);}
            \ControlFlowTok{if}\NormalTok{ (x && x->kind == E_varEntry) \{}
                \CommentTok{// 如果我们拿到的是个名字类型,}
                \CommentTok{// 比如是 let type := int 这种一层包装的类型,}
                \CommentTok{// 就追根究底找出最基本的类型。}
                \CommentTok{// 所以有那么一句 actual_ty。}
                \ControlFlowTok{return}\NormalTok{ ExpTy(NULL, actual_ty(x->u.var.ty));}
\NormalTok{            \} }\ControlFlowTok{else}\NormalTok{ \{}
\NormalTok{                EM_Error(}\StringTok{"哈哈哈哈哈哈哈哈错误消息}\SpecialCharTok{\textbackslash{}n}\StringTok{"}\NormalTok{);}
\NormalTok{            \}}
\NormalTok{        \}}
\NormalTok{    \}}
\NormalTok{\}}
\end{Highlighting}
\end{Shaded}

\hypertarget{header-n56}{%
\paragraph{检查声明}\label{header-n56}}

\hypertarget{header-n57}{%
\subparagraph{变量声明}\label{header-n57}}

处理一个无类型约束的变量声明:

\begin{verbatim}
var x := exp
\end{verbatim}

很简单,因为我们了解赋值操作右边 \texttt{exp} 的唯一类型,把它当作
\texttt{x} 的类型就可以了。

但也有这种写法:

\begin{verbatim}
var x: Type := exp
\end{verbatim}

同时给出了可用于类型推断的 \texttt{exp} 和声明好的
\texttt{Type}。这就是说我们需要考虑推断出来的隐含 Type 跟显式说明的 Type
是否兼容。

\hypertarget{header-n64}{%
\subparagraph{Ms.Nil}\label{header-n64}}

Ty\_Nil 是空指针类型,它可同任何类型兼容。她的实例 Nil
值为空,用于初始化各种类型的对象。

\hypertarget{header-n66}{%
\subparagraph{函数声明}\label{header-n66}}

函数的声明方法\ldots\ldots{}

\begin{verbatim}
function f(a: ta, b: tb): rt = body
\end{verbatim}

我们用到 \texttt{transDec} 函数。

\begin{Shaded}
\begin{Highlighting}[]
\DataTypeTok{void}\NormalTok{ transDec(S_table venv, S_table tenv, A_dec d) \{}
    \ControlFlowTok{switch}\NormalTok{ (d->kind) \{}
        \CommentTok{// ...}
        \ControlFlowTok{case}\NormalTok{ A_functionDec: \{}
            \CommentTok{/* 啊,是个函数声明! */}
\NormalTok{            A_fundec f = d->u.function->head;}
            
\NormalTok{            Ty_ty resultTy = S_look(tenv, f->result);}
            \CommentTok{// 确认函数的返回值。}
            
\NormalTok{            Ty_tyList formalTys = makeFormalTyList(tenv, f->params);}
            \CommentTok{// 构造出形参类型列表。我们不关心他们的名字。}
            
            \CommentTok{// 到此为止我们有足够的信息来构造这个 funEntry 并放入值环境中 `venv` 了。}
            \CommentTok{// 注意,到此为止我们还没有开始分析 body 的内容,}
            \CommentTok{// 也就是说在 body 被分析之前,我们已经把函数 funEntry 放到环境里了。}
            \CommentTok{// 这也就是为什么递归可以被实现。}
            \CommentTok{// 因为在函数体被分析之前,函数 Entry 本身已经被放到环境里了。}
            
\NormalTok{            S_beginScope(venv);}
            
            \ControlFlowTok{for}\NormalTok{ (}\KeywordTok{auto}\NormalTok{ &i : params) \{}
\NormalTok{                S_enter(venv, l->head->name, E_VarEntry(t->head));}
                \CommentTok{// 把形参的名字压栈}
\NormalTok{            \}}
            
            \CommentTok{// 检查 body 的类型}
\NormalTok{            transExp(venv, tenv, d->u.function->body);}
            
            \CommentTok{// Scope 包裹的内容,在离开这个 Body 的时候清除形参符号。}
            \CommentTok{// 不过注意,Function 的符号不会被清除。}
\NormalTok{            S_endScope(venv);}
\NormalTok{        \}}
\NormalTok{    \}}
\NormalTok{\}}
\end{Highlighting}
\end{Shaded}

\hypertarget{header-n71}{%
\paragraph{递归}\label{header-n71}}

TigerLang 里面,所有的类型递归都基于
Pointer,而非真的嵌入一份自己。否则递归定义自己也是无话可说了。

因此递归定义类型这一点还是比较难办\ldots{}

Tiger 里面有一个说法:只有「挨着」(Literally 挨着)的声明才可以递归。

也就是说,可能需要在定义完成某条语句之后,用新的 Environment
回溯一下上一个声明(如果上一个声明失败的话)。

也算是\ldots 给了一点简化吧。

\hypertarget{header-n77}{%
\paragraph{Type Equivalent}\label{header-n77}}

就是那个

\begin{verbatim}
let type x := y
\end{verbatim}

的表达式。

留意到在 Tiger 里,

\begin{verbatim}
let type a = {x: int, y: int}
let type b = {x: int, y: int}

/* 这里我们说 a 跟 b 是结构等价的。 */
\end{verbatim}

a 跟 b 是不一样的两种类型。(Type Inequivalent)

而

\begin{verbatim}
let type a = {x: int, y: int}
let type b = a

/* 这里 a 跟 b 是名字等价的 */
\end{verbatim}

中,a 跟 b 是同样的类型(Type Equivalent)。

\hypertarget{header-n87}{%
\subsection{\texorpdfstring{SE-227::\sout{File System}
CSE}{SE-227::File System CSE}}\label{header-n87}}

\hypertarget{header-n88}{%
\subsubsection{Design DNS}\label{header-n88}}

之前的 GFS 云云,最终都没有跳脱开 Posix 的文件系统,依然还是个 Unix-like
文件系统。

那么有什么 Unix-unlike 的系统呢?

\hypertarget{header-n91}{%
\paragraph{Unix-unlike}\label{header-n91}}

我们来考虑一个 URI:

\begin{verbatim}
www.example.com/favicon.ico
\end{verbatim}

这是否也很类似于一个文件路径呢?

但这个路径经历的并不简单;

在这个请求被发送出去时,可能会借由 CDN
直接抄近路,也可能前往数据中心,还可能根本拿不到结果。

\hypertarget{header-n97}{%
\paragraph{Domain Name and IP}\label{header-n97}}

DNS 要做的就是把我们的一个 Domain Name 给翻译成 IP 位址。

因为类似于 \texttt{www.example.com}
的域名不是语义化的。不能直接发一个请求给「\texttt{www.foobar.com}」,而必须通过
IP 位址来发送请求。

\begin{quote}
为啥不直接用 IP 呢?

难记难用难路由。
\end{quote}

翻译有这些要求:

\begin{itemize}
\item
  快。
\end{itemize}

浏览个网页不能太慢\ldots{}

\begin{itemize}
\item
  动态。
\end{itemize}

随时在更新的域名\ldots{}

\begin{itemize}
\item
  高效。
\end{itemize}

目前域名的数量过大,以至于我们必须很高效地完成翻译过程。

\hypertarget{header-n116}{%
\subparagraph{Questions}\label{header-n116}}

Q: 一个 Domain Name 可以有多个 IP 位址吗?

A: 是的。可以有。因此有的时候 DNS 会在一次查询中返回多个 IP 位址。

Q: 一个 IP 可以有多个 Domain Name 吗?

A: Rui Ren:
可以。甚至讲过用这种方法同时接两份私活,却只花一份服务器的钱,而且两位甲方都没发现的故事。

\hypertarget{header-n121}{%
\paragraph{DNS}\label{header-n121}}

如果一个 Domain Name \textless=\textgreater{} IP address
的关系需要更新了,如何去告诉 DNS 呢?

最早 DNS 是写在像是电话本的书里的。后来写在本地的 \texttt{HOSTS.TXT}
里。但这肯定不靠谱。

后来 1984 年出了个 BIND: Berkeley Internet Name Domain。一台 Name Server
机器存不下了,我们总得分布式地存储他们。

那如何分布呢?规模上去了之后就得\emph{分层 + 结构化}。

首先先分成 \texttt{.com}、\texttt{.gov}、\texttt{.net} 之类的 Zone。

但这还不够,还需要一个绝对 Root 来告诉这些 Zone 到底在哪。

现在这个绝对 Root 是由 \textbf{ICANN} 管理的。

\begin{quote}
ICANN: 〖 Internet Corporation for Assigned Names and Numbers
〗インターネットで使用されるドメイン名IP
アドレスプロトコルなどの管理を行う非営利公益法人の国際機関。世界各地から集まった理事によって運営される。1998
年に設立。本部はアメリカのカリフォルニア。
\end{quote}

\texttt{.com} 就是 for Commercial 的,一般是由 VeriSign 提供的。

\texttt{.sjtu.edu.cn} 就是 for Educational 的。

所以很奇怪的,在 Domain Name
里面,底层的根节点在后头,而越是表层的越在前头。

每一个节点都可以随意分配自己的子域名。但是对自己上层的结构就完全不知情。

能力的划分也是责任的划分。IP 也是一层一层往里找的。

\hypertarget{header-n136}{%
\paragraph{Secret}\label{header-n136}}

其实所有的 Domain Name 最后都有一个点。

类似于文件系统路径开始于 \texttt{/},域名也都结束于一个 \texttt{.}。

Host Name
事实上也跟网站本身没关系,本质上就跟文件名之于文件一样。只是锦上添花,蛋糕上的一层糖霜,奶糖外面的可食用包装纸

\hypertarget{header-n140}{%
\paragraph{Fault Tolerant}\label{header-n140}}

一般来说一个节点炸了,会影响到的就是它自己对应的域名及其子域名的解析。

所以为了应对 Fault,一般一个节点的解析服务器不止一台。

\hypertarget{header-n143}{%
\paragraph{Recursion}\label{header-n143}}

在你想解析一个 Domain Name 的时候,可以自己来慢慢解析:

以 \texttt{www.sjtu.edu.cn.} 为例:

可以先自己去爬 \texttt{.} =\textgreater{} \texttt{.cn.} =\textgreater{}
\texttt{.edu.cn.} =\textgreater{} \texttt{sjtu.edu.cn.} =\textgreater{}
\texttt{www.sjtu.edu.cn.}。

自己一个办公室一个办公室跑,

也可以直接去问 \texttt{www.sjtu.edu.cn},让服务器来帮我找这一路。

\hypertarget{header-n149}{%
\paragraph{Caching}\label{header-n149}}

不光是我本地有 Cache,路上所有的服务器也都会给你做 Cache。

但是 Cache 总归会遇到一个问题:Inconsistency.

映射改了之后,全球的 DNS Server 并不能做到即时改动。

解决方案?

\hypertarget{header-n154}{%
\paragraph{TTL}\label{header-n154}}

约定俗成,一个 Cache 的寿命就是 24 小时。

又约定俗成,在你改换 DN =\textgreater{} IP 对应关系的时候,会给出一个 24
小时的 Cooldown 时间,在这段时间照顾那些还没来得及刷新的
Cache。在那之后,就直接 Reuse 了,且不会产生问题。

\hypertarget{header-n157}{%
\paragraph{Replica}\label{header-n157}}

所有的数据都应该做好
Replica,且分散在世界各地,以防发生核战争之后大家没网上。

SJTU
也在学校各地各校区放了许多个备份,且在学校外也至少提供了一个服务器(以便提供一些公开的服务)。\sout{但总归要断网一起断}

\hypertarget{header-n160}{%
\paragraph{The Hidden Master}\label{header-n160}}

\ldots 但是 DNS 总该有个头啊。是谁给的呢?

在连 Wi-Fi 的时候,会使用 DHCP 技术来分配给你一套 IP、Mask、Router
之类的数据。可选地,也会给你一个 DNS Address。

\begin{verbatim}
“Wi-Fi”已连接至“SJTU”,其 IP 地址为 10.162.204.224。

已通过“EAP-PEAP”鉴定 (MSCHAPv2)
连接时间:03:23:38
\end{verbatim}

\hypertarget{header-n164}{%
\paragraph{Bad Points}\label{header-n164}}

\hypertarget{header-n165}{%
\subparagraph{Policy}\label{header-n165}}

目前 DNS 的政策仍然由 ICANN 决定。也有一定的能力做坏事。

\hypertarget{header-n167}{%
\subparagraph{DNS as a Weapon}\label{header-n167}}

如果有个网站我想灭掉它,就把 google 的 DNS 重定向给他。大家都上
Google,上 Google 就会去查 Google 的 IP。于是这个网站服务器就爆了。

\hypertarget{header-n169}{%
\subparagraph{Cache Miss Attack}\label{header-n169}}

如果所有的 Cache 都 miss,这些压力都会一直向上走,增大 Root
根结点的压力。

\hypertarget{header-n171}{%
\subparagraph{Security}\label{header-n171}}

我们可以信任 VeriSign/CNNIC 吗?

2015 年,因 CNNIC 发行的一个中级CA被发现发行了 Google
域名的假证书,许多用户选择不信任 CNNIC 颁发的数字证书。并引起对 CNNIC
滥用证书颁发权力的担忧。

2015 年 4 月 2 日,Google 宣布不再承认 CNNIC 所颁发的电子证书。4 月 4
日,继 Google 之后,Mozilla 也宣布不再承认 CNNIC 所颁发的电子证书。2016
年 8 月,CNNIC 官方网站已放弃自行发行的根证书,改用由 DigiCert
颁发的证书。

\hypertarget{header-n175}{%
\subsubsection{Introduction to Network}\label{header-n175}}

网络是怎么实现的?

\hypertarget{header-n177}{%
\paragraph{OSI}\label{header-n177}}

\hypertarget{header-n178}{%
\paragraph{TCP/UDP}\label{header-n178}}

TCP 握手三次。UDP 不握手,上来就发。

\hypertarget{header-n180}{%
\paragraph{Protocol Stack}\label{header-n180}}

\hypertarget{header-n181}{%
\paragraph{CSE 3 Layers}\label{header-n181}}

Layer 1: 两台直接相连(通过网线)的计算机可以交流信息。

Layer 2: 物理上多台计算机连接在一起,任意两台相连的计算机可以交换信息。

Layer 3: 性能、安全性、容错。

\end{document}
