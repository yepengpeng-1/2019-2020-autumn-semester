% Options for packages loaded elsewhere
\PassOptionsToPackage{unicode}{hyperref}
\PassOptionsToPackage{hyphens}{url}
%
\documentclass[
]{article}
\usepackage{lmodern}
\usepackage{amssymb,amsmath}
\usepackage{ifxetex,ifluatex}
\ifnum 0\ifxetex 1\fi\ifluatex 1\fi=0 % if pdftex
  \usepackage[T1]{fontenc}
  \usepackage[utf8]{inputenc}
  \usepackage{textcomp} % provide euro and other symbols
\else % if luatex or xetex
  \usepackage{unicode-math}
  \defaultfontfeatures{Scale=MatchLowercase}
  \defaultfontfeatures[\rmfamily]{Ligatures=TeX,Scale=1}
\fi
% Use upquote if available, for straight quotes in verbatim environments
\IfFileExists{upquote.sty}{\usepackage{upquote}}{}
\IfFileExists{microtype.sty}{% use microtype if available
  \usepackage[]{microtype}
  \UseMicrotypeSet[protrusion]{basicmath} % disable protrusion for tt fonts
}{}
\makeatletter
\@ifundefined{KOMAClassName}{% if non-KOMA class
  \IfFileExists{parskip.sty}{%
    \usepackage{parskip}
  }{% else
    \setlength{\parindent}{0pt}
    \setlength{\parskip}{6pt plus 2pt minus 1pt}}
}{% if KOMA class
  \KOMAoptions{parskip=half}}
\makeatother
\usepackage{xcolor}
\IfFileExists{xurl.sty}{\usepackage{xurl}}{} % add URL line breaks if available
\IfFileExists{bookmark.sty}{\usepackage{bookmark}}{\usepackage{hyperref}}
\hypersetup{
  hidelinks,
  pdfcreator={LaTeX via pandoc}}
\urlstyle{same} % disable monospaced font for URLs
\usepackage{graphicx,grffile}
\makeatletter
\def\maxwidth{\ifdim\Gin@nat@width>\linewidth\linewidth\else\Gin@nat@width\fi}
\def\maxheight{\ifdim\Gin@nat@height>\textheight\textheight\else\Gin@nat@height\fi}
\makeatother
% Scale images if necessary, so that they will not overflow the page
% margins by default, and it is still possible to overwrite the defaults
% using explicit options in \includegraphics[width, height, ...]{}
\setkeys{Gin}{width=\maxwidth,height=\maxheight,keepaspectratio}
% Set default figure placement to htbp
\makeatletter
\def\fps@figure{htbp}
\makeatother
\setlength{\emergencystretch}{3em} % prevent overfull lines
\providecommand{\tightlist}{%
  \setlength{\itemsep}{0pt}\setlength{\parskip}{0pt}}
\setcounter{secnumdepth}{-\maxdimen} % remove section numbering

\date{}

\begin{document}

\hypertarget{header-n0}{%
\section{Oct 28 Mon}\label{header-n0}}

\hypertarget{header-n2}{%
\subsection{SE-342::CV}\label{header-n2}}

\hypertarget{header-n3}{%
\subsubsection{Remember}\label{header-n3}}

回忆上节课所讲的:结构元、膨胀、腐蚀、开操作、闭操作\ldots\ldots{}

Hit/Miss. 定义两个无交集的结构元:一个击中前景,一个击中背景。

\hypertarget{header-n6}{%
\subsubsection{New Stuff}\label{header-n6}}

\hypertarget{header-n7}{%
\paragraph{Thinning}\label{header-n7}}

所谓「细化」。

对前景的细化就是对背景的粗化。

\hypertarget{header-n10}{%
\paragraph{Thickening}\label{header-n10}}

所以我们直接对此对象求补集(Complementary),然后进行细化再补回来。

就成了。因此也没人去做粗化。

\hypertarget{header-n13}{%
\paragraph{Shape Inspection}\label{header-n13}}

如何区分不同形状的物体,并对其进行分离呢?

看你具体想要区分的形状类型?

最好确定的应该就是 Circle 了(没有任何旋转问题)

其他的可能都得考虑旋转造成的非一致性问题。

\hypertarget{header-n18}{%
\subsubsection{Grayscale Dilation \& Erosion}\label{header-n18}}

在我们从二值化的图形走向灰阶图形的的时候,该怎么处理?

Dilation: 输出的是结构元指定区域的邻域内的最大值,

Erosion: 输出的是结构元指定区域的邻域内的最小值。

目前结构元不能再使用 非 False 即 True 的思路了。

\hypertarget{header-n23}{%
\paragraph{Dilation}\label{header-n23}}

In \href{https://en.wikipedia.org/wiki/Grayscale}{grayscale} morphology,
images are
\href{https://en.wikipedia.org/wiki/Function_(mathematics)}{functions}
mapping a \href{https://en.wikipedia.org/wiki/Euclidean_space}{Euclidean
space} or \href{https://en.wikipedia.org/wiki/Lattice_graph}{grid}
\emph{E} into
R∪\{∞,−∞\}\includegraphics{https://wikimedia.org/api/rest_v1/media/math/render/svg/f9cee5555d47e06ecbd7cd3c6c937351743f5142},
where
\includegraphics{https://wikimedia.org/api/rest_v1/media/math/render/svg/786849c765da7a84dbc3cce43e96aad58a5868dc}
is the set of \href{https://en.wikipedia.org/wiki/Real_numbers}{reals},
∞\includegraphics{https://wikimedia.org/api/rest_v1/media/math/render/svg/c26c105004f30c27aa7c2a9c601550a4183b1f21}
is an element greater than any real number, and −∞ is an element less
than any real number.

Grayscale structuring elements are also functions of the same format,
called "structuring functions".

Denoting an image by \emph{f}(\emph{x}) and the structuring function by
\emph{b}(\emph{x}), the grayscale dilation of \emph{f} by \emph{b} is
given by

\((f\oplus b)(x)=\sup _{{y\in E}}[f(y)+b(x-y)]\),

where "sup" denotes the
\href{https://en.wikipedia.org/wiki/Supremum}{supremum}.

\href{https://en.wikipedia.org/wiki/File:Grayscale_Morphological_Dilation.gif}{\includegraphics{https:////upload.wikimedia.org/wikipedia/commons/thumb/7/70/Grayscale_Morphological_Dilation.gif/220px-Grayscale_Morphological_Dilation.gif}}

Example of dilation on a grayscale image using a 5x5 flat structuring
element. The top figure demonstrates the application of the structuring
element window to the individual pixels of the original image. The
bottom figure shows the resulting dilated image.

It is common to use flat structuring elements in morphological
applications. Flat structuring functions are functions
\emph{b}(\emph{x}) in the form

\(b(x)=\left\{{\begin{array}{ll}0,&x\in B,\\-\infty ,&{\text{otherwise}},\end{array}}\right.\)

where
\includegraphics{https://wikimedia.org/api/rest_v1/media/math/render/svg/2c752e3a4ab7676572b6f4ab3f038aebaab3e4f5}.

In this case, the dilation is greatly simplified, and given by

\((f\oplus b)(x)=\sup _{{y\in E}}[f(y)+b(x-y)]=\sup _{{z\in E}}[f(x-z)+b(z)]=\sup _{{z\in B}}[f(x-z)].\)

(Suppose \emph{x} = (\emph{px}, \emph{qx}), \emph{z} = (\emph{pz},
\emph{qz}), then \emph{x} − \emph{z} = (\emph{px} − \emph{pz}, \emph{qx}
− \emph{qz}).)

In the bounded, discrete case (\emph{E} is a grid and \emph{B} is
bounded), the \href{https://en.wikipedia.org/wiki/Supremum}{supremum}
operator can be replaced by the
\href{https://en.wikipedia.org/wiki/Maximum}{maximum}. Thus, dilation is
a particular case of
\href{https://en.wikipedia.org/wiki/Order_statistics}{order statistics}
filters, returning the maximum value within a moving window (the
symmetric of the structuring function support \emph{B})

简单说,定义如此。按照这样的办法就好了。

\begin{figure}
\centering
\includegraphics{https://raw.githubusercontent.com/yuetsin/private-image-repo/master/2019/10/28/{[}779EEDC8-536D-42A5-B746-FF712BDA227C{]} IMG_1258.jpeg}
\caption{}
\end{figure}

\hypertarget{header-n40}{%
\paragraph{Erosion}\label{header-n40}}

\hypertarget{header-n41}{%
\subsection{\texorpdfstring{Grayscale
erosion{[}\href{https://en.wikipedia.org/w/index.php?title=Erosion_(morphology)\&action=edit\&section=4}{edit}{]}}{Grayscale erosion{[}edit{]}}}\label{header-n41}}

\href{https://en.wikipedia.org/wiki/File:Grayscale_Morphological_Erosion.gif}{\includegraphics{https:////upload.wikimedia.org/wikipedia/commons/thumb/b/b9/Grayscale_Morphological_Erosion.gif/220px-Grayscale_Morphological_Erosion.gif}}

Example of erosion on a grayscale image using a 5x5 flat structuring
element. The top figure demonstrates the application of the structuring
element window to the individual pixels of the original image. The
bottom figure shows the resulting eroded image.

In \href{https://en.wikipedia.org/wiki/Grayscale}{grayscale} morphology,
images are
\href{https://en.wikipedia.org/wiki/Function_(mathematics)}{functions}
mapping a \href{https://en.wikipedia.org/wiki/Euclidean_space}{Euclidean
space} or \href{https://en.wikipedia.org/wiki/Lattice_graph}{grid}
\emph{E} into
\includegraphics{https://wikimedia.org/api/rest_v1/media/math/render/svg/f9cee5555d47e06ecbd7cd3c6c937351743f5142},
where
\includegraphics{https://wikimedia.org/api/rest_v1/media/math/render/svg/786849c765da7a84dbc3cce43e96aad58a5868dc}
is the set of \href{https://en.wikipedia.org/wiki/Real_numbers}{reals},
\includegraphics{https://wikimedia.org/api/rest_v1/media/math/render/svg/c26c105004f30c27aa7c2a9c601550a4183b1f21}
is an element larger than any real number, and
\includegraphics{https://wikimedia.org/api/rest_v1/media/math/render/svg/ca2608c4b5fd3bffc73585f8c67e379b4e99b6f1}
is an element smaller than any real number.

Denoting an image by \emph{f(x)} and the grayscale structuring element
by \emph{b(x)}, where B is the space that b(x) is defined, the grayscale
erosion of \emph{f} by \emph{b} is given by

\((f\ominus b)(x)=\inf _{{y\in B}}[f(x+y)-b(y)]\),

where "inf" denotes the
\href{https://en.wikipedia.org/wiki/Infimum}{infimum}.

In other words the erosion of a point is the minimum of the points in
its neighborhood, with that neighborhood defined by the structuring
element. In this way it is similar to many other kinds of image filters
like the \href{https://en.wikipedia.org/wiki/Median_filter}{median
filter} and the
\href{https://en.wikipedia.org/wiki/Gaussian_filter}{gaussian filter}.

\begin{figure}
\centering
\includegraphics{https://raw.githubusercontent.com/yuetsin/private-image-repo/master/2019/10/28/{[}B32D85DB-AE9E-49C0-8270-0A48E553326C{]} IMG_1259.jpeg}
\caption{}
\end{figure}

\hypertarget{header-n50}{%
\subsubsection{Edge Detection}\label{header-n50}}

\hypertarget{header-n51}{%
\paragraph{Inner Edge}\label{header-n51}}

原图像---腐蚀结果

\hypertarget{header-n53}{%
\paragraph{Outer Edge}\label{header-n53}}

膨胀结果---原图像

\hypertarget{header-n55}{%
\paragraph{Standard Edge}\label{header-n55}}

膨胀结果---腐蚀结果

\end{document}
