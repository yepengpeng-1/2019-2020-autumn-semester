% Options for packages loaded elsewhere
\PassOptionsToPackage{unicode}{hyperref}
\PassOptionsToPackage{hyphens}{url}
%
\documentclass[
]{article}
\usepackage{lmodern}
\usepackage{amssymb,amsmath}
\usepackage{ifxetex,ifluatex}
\ifnum 0\ifxetex 1\fi\ifluatex 1\fi=0 % if pdftex
  \usepackage[T1]{fontenc}
  \usepackage[utf8]{inputenc}
  \usepackage{textcomp} % provide euro and other symbols
\else % if luatex or xetex
  \usepackage{unicode-math}
  \defaultfontfeatures{Scale=MatchLowercase}
  \defaultfontfeatures[\rmfamily]{Ligatures=TeX,Scale=1}
\fi
% Use upquote if available, for straight quotes in verbatim environments
\IfFileExists{upquote.sty}{\usepackage{upquote}}{}
\IfFileExists{microtype.sty}{% use microtype if available
  \usepackage[]{microtype}
  \UseMicrotypeSet[protrusion]{basicmath} % disable protrusion for tt fonts
}{}
\makeatletter
\@ifundefined{KOMAClassName}{% if non-KOMA class
  \IfFileExists{parskip.sty}{%
    \usepackage{parskip}
  }{% else
    \setlength{\parindent}{0pt}
    \setlength{\parskip}{6pt plus 2pt minus 1pt}}
}{% if KOMA class
  \KOMAoptions{parskip=half}}
\makeatother
\usepackage{xcolor}
\IfFileExists{xurl.sty}{\usepackage{xurl}}{} % add URL line breaks if available
\IfFileExists{bookmark.sty}{\usepackage{bookmark}}{\usepackage{hyperref}}
\hypersetup{
  hidelinks,
  pdfcreator={LaTeX via pandoc}}
\urlstyle{same} % disable monospaced font for URLs
\usepackage{color}
\usepackage{fancyvrb}
\newcommand{\VerbBar}{|}
\newcommand{\VERB}{\Verb[commandchars=\\\{\}]}
\DefineVerbatimEnvironment{Highlighting}{Verbatim}{commandchars=\\\{\}}
% Add ',fontsize=\small' for more characters per line
\newenvironment{Shaded}{}{}
\newcommand{\AlertTok}[1]{\textcolor[rgb]{1.00,0.00,0.00}{\textbf{#1}}}
\newcommand{\AnnotationTok}[1]{\textcolor[rgb]{0.38,0.63,0.69}{\textbf{\textit{#1}}}}
\newcommand{\AttributeTok}[1]{\textcolor[rgb]{0.49,0.56,0.16}{#1}}
\newcommand{\BaseNTok}[1]{\textcolor[rgb]{0.25,0.63,0.44}{#1}}
\newcommand{\BuiltInTok}[1]{#1}
\newcommand{\CharTok}[1]{\textcolor[rgb]{0.25,0.44,0.63}{#1}}
\newcommand{\CommentTok}[1]{\textcolor[rgb]{0.38,0.63,0.69}{\textit{#1}}}
\newcommand{\CommentVarTok}[1]{\textcolor[rgb]{0.38,0.63,0.69}{\textbf{\textit{#1}}}}
\newcommand{\ConstantTok}[1]{\textcolor[rgb]{0.53,0.00,0.00}{#1}}
\newcommand{\ControlFlowTok}[1]{\textcolor[rgb]{0.00,0.44,0.13}{\textbf{#1}}}
\newcommand{\DataTypeTok}[1]{\textcolor[rgb]{0.56,0.13,0.00}{#1}}
\newcommand{\DecValTok}[1]{\textcolor[rgb]{0.25,0.63,0.44}{#1}}
\newcommand{\DocumentationTok}[1]{\textcolor[rgb]{0.73,0.13,0.13}{\textit{#1}}}
\newcommand{\ErrorTok}[1]{\textcolor[rgb]{1.00,0.00,0.00}{\textbf{#1}}}
\newcommand{\ExtensionTok}[1]{#1}
\newcommand{\FloatTok}[1]{\textcolor[rgb]{0.25,0.63,0.44}{#1}}
\newcommand{\FunctionTok}[1]{\textcolor[rgb]{0.02,0.16,0.49}{#1}}
\newcommand{\ImportTok}[1]{#1}
\newcommand{\InformationTok}[1]{\textcolor[rgb]{0.38,0.63,0.69}{\textbf{\textit{#1}}}}
\newcommand{\KeywordTok}[1]{\textcolor[rgb]{0.00,0.44,0.13}{\textbf{#1}}}
\newcommand{\NormalTok}[1]{#1}
\newcommand{\OperatorTok}[1]{\textcolor[rgb]{0.40,0.40,0.40}{#1}}
\newcommand{\OtherTok}[1]{\textcolor[rgb]{0.00,0.44,0.13}{#1}}
\newcommand{\PreprocessorTok}[1]{\textcolor[rgb]{0.74,0.48,0.00}{#1}}
\newcommand{\RegionMarkerTok}[1]{#1}
\newcommand{\SpecialCharTok}[1]{\textcolor[rgb]{0.25,0.44,0.63}{#1}}
\newcommand{\SpecialStringTok}[1]{\textcolor[rgb]{0.73,0.40,0.53}{#1}}
\newcommand{\StringTok}[1]{\textcolor[rgb]{0.25,0.44,0.63}{#1}}
\newcommand{\VariableTok}[1]{\textcolor[rgb]{0.10,0.09,0.49}{#1}}
\newcommand{\VerbatimStringTok}[1]{\textcolor[rgb]{0.25,0.44,0.63}{#1}}
\newcommand{\WarningTok}[1]{\textcolor[rgb]{0.38,0.63,0.69}{\textbf{\textit{#1}}}}
\setlength{\emergencystretch}{3em} % prevent overfull lines
\providecommand{\tightlist}{%
  \setlength{\itemsep}{0pt}\setlength{\parskip}{0pt}}
\setcounter{secnumdepth}{-\maxdimen} % remove section numbering

\date{}

\begin{document}

\hypertarget{header-n0}{%
\section{Oct 8 2019}\label{header-n0}}

\hypertarget{header-n2}{%
\subsection{SE-302}\label{header-n2}}

Error Recovery\ldots 今天要讲的是错误恢复。

在我们 Parsing 的时候遇到看起来不对的情况的时候,该怎么办呢?

\begin{itemize}
\item
  直接停下来,爆出错误
\item
  记录一个错误,但还是试着向下做 Parse
\end{itemize}

方法 II
有什么好处呢?可以尽量一次性发现更多的错误,减少编译次数,提高编译效率。

几乎所有的 Parser 方式(无论是 LL、LR 还是其他的)都可以适用。

不光是对 Parser 是这样,Lex 跟 Semantic Check 也差不多是这么回事。

\hypertarget{header-n13}{%
\subsubsection{如何做 Recovery?}\label{header-n13}}

我们有两种还不错的策略:局部的,全局的。

\hypertarget{header-n15}{%
\paragraph{Local Strategy}\label{header-n15}}

本地策略,即直接在语法规则里面加入一些跟「ERROR」有关的规则,使得在一些语句块里面的
ERROR 不会影响到全局的 Parsing。

举例:

\texttt{;}之后,以及\texttt{(}、\texttt{)}之间是可以看成一个相对独立的
Scope 的(不会对全局造成影响)。

因此如果我们在 \texttt{;} 之前和 \texttt{()} 之间发现了一个
ERROR,就直接把这一块抛掉,但不影响后续的 Parsing。

表达式写作 \texttt{exp\ -\textgreater{}\ (ERROR)},即括号中包含一个
ERROR 时会被直接忽略掉,并继续分析后续的内容。

这个唤作 Local Strategy。

不过在语法规则比较复杂时,ERROR 规则可能有点复杂。

\hypertarget{header-n23}{%
\paragraph{Global Strategy}\label{header-n23}}

全局策略?那是啥?

Global
会提前扫描一遍整个程序并发现问题,并且决定对程序如何做出\textbf{最小改动}来使得程序变得无误。

这个好处是什么呢?相比 Local Strategy,GS 可以尽量保留程序原意,而不是像
LS 直接抛弃掉了有问题的这一块。

但是,如何决定作出那个「最小改动」,也是 GS 的难点了。

\hypertarget{header-n28}{%
\paragraph{Burke-Fisher Error Repair}\label{header-n28}}

BF 错误修复法:简单暴力(x)

对一段程序中的 k 长度的 Token
序列在每一个位置都尝试一次插入、删除、替换不同 Token 的操作。

并每次都做一次 Parsing 尝试,哪一种能够进行最长时间的 Parsing
(能走到最远的地方)就选择哪一种。

\begin{quote}
随便想一下都知道这种方法的效率有多低了\ldots 但凡 k 稍微取大一点,Token
的数量稍微多一点就爆炸了。
\end{quote}

\begin{quote}
作者也说了:Use this merely in uncommon case.
\end{quote}

\hypertarget{header-n36}{%
\subsubsection{Examples}\label{header-n36}}

像是 \texttt{C++} 这样的语言,在发现缺少类型标识符的时候,会默认将其定为
\texttt{int} (\texttt{C} 的历史遗留)来分析后面的词法。

(当然 \texttt{C++} 在这里是会报告错误的)

\hypertarget{header-n39}{%
\subsubsection{Parser Generator}\label{header-n39}}

生成分析器,咱们得用 \texttt{Yacc}。

这位爷的名字意思是 Yet Another Compiler Compiler。哈。

\hypertarget{header-n42}{%
\subsubsection{Yacc Syntax}\label{header-n42}}

\hypertarget{header-n43}{%
\paragraph{Example}\label{header-n43}}

\begin{Shaded}
\begin{Highlighting}[]
\BaseNTok{%\{}\NormalTok{Supplementary Code%\}}

\NormalTok{%definition}
\NormalTok{……}
\NormalTok{%definition}

\NormalTok

\NormalTok{Rules}
\end{Highlighting}
\end{Shaded}

\hypertarget{header-n45}{%
\paragraph{Supplementary Code}\label{header-n45}}

绝类 Lex,这么写:

\begin{Shaded}
\begin{Highlighting}[]
\BaseNTok{%\{}
 \DataTypeTok{int}\NormalTok{ yylex(}\DataTypeTok{void}\NormalTok{);}
 \DataTypeTok{void}\NormalTok{ yyerror(}\DataTypeTok{char}\NormalTok{ *s) \{ EM_error(EM_tokPos, “%s”, s)\}  }
\BaseNTok{%\}}
\end{Highlighting}
\end{Shaded}

声明一些常用的 Helper 函数。

\hypertarget{header-n49}{%
\paragraph{\texorpdfstring{\texttt{\%definition}}{\%definition}}\label{header-n49}}

\begin{itemize}
\item
  \texttt{\%token}
\end{itemize}

这是 Terminal。

\begin{itemize}
\item
  \texttt{\%start}
\end{itemize}

这是 Non-Terminal。

\begin{itemize}
\item
  \texttt{\%left} / \texttt{\%right} / \texttt{\%nonassoc}
\end{itemize}

分别指左结合、右结合、不结合。

用它们可以声明运算符的优先级、结合律。如:

\begin{Shaded}
\begin{Highlighting}[]
\KeywordTok{%left}\NormalTok{ PLUS MINUS}
\KeywordTok{%left}\NormalTok{ TIMES}
\KeywordTok{%left}\NormalTok{ UMINUS}
\end{Highlighting}
\end{Shaded}

说明了 PLUS、MINUS、TIMES 和 UMINUS 都是左结合的;而且,UMINUS
的优先级高于 TIMES,TIMES 的优先级又高于 PLUS 和 MINUS。

\hypertarget{header-n65}{%
\paragraph{\texorpdfstring{\texttt{\%union}}{\%union}}\label{header-n65}}

\begin{Shaded}
\begin{Highlighting}[]
\KeywordTok{%union}\NormalTok{ \{}
	\DataTypeTok{int}\NormalTok{ num;}
\NormalTok{	string id;}
\NormalTok{\}}
\end{Highlighting}
\end{Shaded}

生命了一个联合,于是我们就可以在 \texttt{\%token} 和 \texttt{\%type}
的时候使用 \textless\textgreater{} 来取用 Union 中的其中一枚。

语法写作 \texttt{\%token\textless{}num\textgreater{}} 或是
\texttt{\%type\textless{}int\textgreater{}}。

\hypertarget{header-n69}{%
\paragraph{Rules}\label{header-n69}}

\texttt{\%\%} 后面的就是规则了。咋写?

\begin{Shaded}
\begin{Highlighting}[]

\BaseNTok
\StringTok{stm}\NormalTok{:	ID ASSIGN ae}
\NormalTok{	  |	ID ASSIGN be}

\StringTok{be}\NormalTok{:	be OR be}
\NormalTok{	|	be AND be}
\NormalTok{	|	ae EQUAL ae}
\NormalTok{    |	ID}

\StringTok{ae}\NormalTok{:	ae PLUS ae}
\NormalTok{    |	ID}
\end{Highlighting}
\end{Shaded}

\hypertarget{header-n72}{%
\paragraph{操作}\label{header-n72}}

\begin{Shaded}
\begin{Highlighting}[]
\NormalTok{exp	:	INT			\{$$ = $1;\}}
\end{Highlighting}
\end{Shaded}

从上向下优先级从高到低;最左边的代名词为 \texttt{\$\$},:
右边的符号一个个命名为 \texttt{\$1、\$2、\$3...}。

\{\} 里面是在条件满足的时候执行的代码。你可以嵌入各种 \texttt{C++}
代码。

\begin{Shaded}
\begin{Highlighting}[]
\NormalTok{exp	:	UMINUS exp	  	}\KeywordTok{%prec}\NormalTok{ UMINUS  \{$$ = -$2; \}}
\end{Highlighting}
\end{Shaded}

\begin{quote}
\texttt{\%prec} 代表这条规则的优先级和 \texttt{UMINUS} 一样。
\end{quote}

\begin{quote}
这就属于手动指定优先级。
\end{quote}

\hypertarget{header-n81}{%
\paragraph{Burke-Fisher 法支持}\label{header-n81}}

在我们插入的时候,当然不可能随便找 token 或者是 int 去插入。

(token 那么多\ldots\ldots)

我们通过

\begin{Shaded}
\begin{Highlighting}[]
\KeywordTok{%value}\NormalTok{ INT\{0\}}

\KeywordTok{%value}\NormalTok{ STRING \{}\StringTok{""}\NormalTok{\}}

\KeywordTok{%value}\NormalTok{ ID \{make_id }\StringTok{"bogus"}\NormalTok{\}}
\end{Highlighting}
\end{Shaded}

来提供一些各种 Token 类型的默认值供 Burke-Fisher 法使用。

\hypertarget{header-n87}{%
\subsubsection{不只是 True / False}\label{header-n87}}

在 Parsing 步骤结束之后,我们还需要进行后续的 Semantic
Check。因此如果我们在 Parse 的时候仅仅告诉结果 True /
False,那是不够的。(你这样后面的各位怎么开展工作)

我们需要把我们 Parse
出来的结果变成一些有效的数据,提供给后面的工位让他们干活。

这个唤作「模块化」。

\hypertarget{header-n91}{%
\subsubsection{Intermediate Representation}\label{header-n91}}

中间表达式是用来在 Compiler 内部各个模块之间的通信交流所用到的数据格式。

主要由下面两部分构成:

\hypertarget{header-n94}{%
\paragraph{抽象语法树}\label{header-n94}}

Abstract Syntax Tree.

\hypertarget{header-n96}{%
\paragraph{符号表(环境相关)}\label{header-n96}}

Environment Table.

主要包含 Variables, Functions, User-defined Types 等基础信息。

\hypertarget{header-n99}{%
\subsubsection{注意}\label{header-n99}}

有三个事实需要您注意。

\begin{enumerate}
\def\labelenumi{\arabic{enumi}.}
\item
  存在递归定义的 Type。
\item
  CFG, Control Flow Graph,控制流图用来定义程序的构成。
\item
  Parse tree can be used to represent the internal structures of the
  program.
\end{enumerate}

所以说了这些,树用来表示这种内部语法是非常合适的。因此我们就选用了 Tree
这种数据结构来记录我们的抽象语法。

\hypertarget{header-n109}{%
\subsubsection{Semantic Rules}\label{header-n109}}

事实上满足一个语义规则时我们所做的事情可以是:

给一棵树添枝加叶;

更新一个全局值;

打印一个值,等等。

\begin{quote}
Warning: 更新全局值、打印值等等这些操作都是有「Side
Effect」的。如在遇到错误、需要回滚时,或是做 Burke-Fisher
暴力法时,可能会多次分析同一段
Token。如果说没有处理好,可能会导致全局变量混乱,或者重复打印值。
\end{quote}

\hypertarget{header-n116}{%
\subsubsection{Syntax Tree Structures}\label{header-n116}}

事实上这棵语法树和我们在 \texttt{Lab\ 1} 中做的 Straightline 很类似啊。

这里有一些我们可以 call 的函数:

\begin{Shaded}
\begin{Highlighting}[]
\CommentTok{// Define operations on the data structures}

\NormalTok{mknode(op, left, right);}

\NormalTok{mkleaf(id, entry);}

\NormalTok{mkleaf(num, val);}
\end{Highlighting}
\end{Shaded}

\hypertarget{header-n120}{%
\subsubsection{Tiger 的语法树结构}\label{header-n120}}

\begin{Shaded}
\begin{Highlighting}[]
 \KeywordTok{typedef} \KeywordTok{struct}\NormalTok{ A_var_ *A_var ;}
 \KeywordTok{struct}\NormalTok{ A_var_ \{}
	\KeywordTok{enum}\NormalTok{ \{ A_simpleVar, A_fieldVar, A_subscriptVar \} kind ;}
\NormalTok{	A_pos pos ;}
	\KeywordTok{union}\NormalTok{ \{}
\NormalTok{		S_symbol simple ;}
		\KeywordTok{struct}\NormalTok{ \{ A_var var ; S_symbol sym ; \} field ;}
		\KeywordTok{struct}\NormalTok{ \{ A_var var ; A_exp exp ; \} subscript ;}
\NormalTok{	\} u ;}
\NormalTok{ \}}
 
\NormalTok{A_var A_SimpleVar(A_pos, S_symbol);}
\NormalTok{A_var A_FieldVar(A_pos, A_var, S_symbol);}
\NormalTok{A_var A_SubscriptVar(A_pos, A_var, A_exp);}

\NormalTok{A_exp A_VarExp(A_pos, A_var);}
\NormalTok{A_exp A_NilExp(A_pos);}
\NormalTok{A_exp A_IntExp(A_pos, }\DataTypeTok{int}\NormalTok{);}
\NormalTok{A_exp A_StringExp(A_pos, string);}
\NormalTok{A_exp A_CallExp(A_pos, S_symbol, A_expList);}
\NormalTok{A_exp A_OpExp(A_pos, A_oper, A_exp, A_exp);}
\NormalTok{A_exp A_RecordExp(A_pos, S_symbol, A_efieldlist);}
\NormalTok{A_exp A_SeqExp(A_pos, A_expList);}

\NormalTok{A_exp A_AssignExp(A_pos, A_var, A_exp);}
\NormalTok{A_exp A_IfExp(A_pos, A_exp, A_exp, A_exp);}
\NormalTok{A_exp A_WhileExp(A_pos, A_exp, A_exp);}
\NormalTok{A_exp A_BreakExp(A_pos);}
\NormalTok{A_exp A_ForExp(A_pos, S_symbol, A_exp, A_exp, A_exp);}
\NormalTok{A_exp A_LetExp(A_pos, A_decList, A_exp);}
\NormalTok{A_exp A_ArrayExp(A_pos, S_symbol, A_exp, A_exp);}

\NormalTok{A_SeqExp ( A_pos pos, A_expList exp ) ; }

\NormalTok{A_dec A_FunctionDec(A_pos, A_funcdecList);}
\NormalTok{A_dec A_VarDec(A_pos, S_symbol, S_symbol, A_exp);}
\NormalTok{A_dec A_TypeDec(A_pos, A_nametyList);}

\NormalTok{A_ty A_NameTy(A_pos, S_symbol);}
\NormalTok{A_ty A_RecordTy(A_pos, A_fieldList);}
\NormalTok{A_ty A_ArrayTy(A_pos, S_symbol);}

\NormalTok{A_field A_Field(A_pos, S_symbol, S_symbol);}
\NormalTok{A_fieldList A_FieldList(A_field, A_fieldList);}
\NormalTok{A_expList A_ExpList(A_exp, A_expList);}
\NormalTok{A_fundec A_Fundec(A_pos, S_symbol, A_fieldList, S_symbol, A_exp);}
\NormalTok{A_fundecList A_FundecList(A_fundec, A_fundecList);}
\NormalTok{A_decList A_DecList(A_dec, A_decList);}
\NormalTok{A_namety A_Namety(S_symbol, A_ty ty);}
\NormalTok{A_nametyList A_NametyList(A_namety, A_nametyList);}
\NormalTok{A_efield A_Field(S_symbol, A_exp);}
\NormalTok{A_efieldList A_FieldList(A_efield, A_efieldList);}

\KeywordTok{typedef} \KeywordTok{enum}\NormalTok{ \{A_plusOp, A_minusOp, A_timeOp, A_divideOp, A_eqOp, A_neqOp, A_ltOp, A_leOp, A_gtOp, A_geOp\} A_oper;}
\end{Highlighting}
\end{Shaded}

\hypertarget{header-n122}{%
\subsubsection{Position}\label{header-n122}}

当出现剖析错误时\ldots\ldots 能告诉我 ERROR 在哪儿吗?

\begin{verbatim}
int x == 3;
1,5:  ^^ ERROR: blablabla
\end{verbatim}

比如像这样和谐美妙的错误提示\ldots 写 Lab 经常见的错误提示\ldots(x)

Bison 可以提供错误位置的提示。然而 Yacc 做不到。

所以我们还是用 Bison 好啦。

\hypertarget{header-n128}{%
\subsection{SE-227}\label{header-n128}}

CSE,CSE,LaLaLaLaLa

这门课还不如改名叫做「Introduction to the File Systems」

\hypertarget{header-n131}{%
\subsubsection{Hints}\label{header-n131}}

\begin{itemize}
\item
  Soft Link 和 Hard Link
\item
  File Read \& Timeline
\end{itemize}

\hypertarget{header-n137}{%
\subsubsection{One Layer Deeper}\label{header-n137}}

再往下看一层,我们可以看出来 PIO 是最接近磁盘底层的抽象。

Load/Store。无所谓 Block 代表的是数据,是 META
data。总归都成了一堆数据。

这里有一个关键字 \texttt{volatile},指明了这种从 Disk 到 Register
的映射是虚假的。要告诉 Compilers 和 OS,这不是真的 DRAM,这是 Disk 啊。

\hypertarget{header-n141}{%
\subsubsection{Bus}\label{header-n141}}

总线?

啥是 Bus 呢?在所有的设备之间连起来的一条车道,每个
Component(小组件)都可以往里面丢信息,从里面吃信息。

每一次 Bus 的「开行」会带来一系列的比特位。为了不造成混淆,每个
Component 都分到其中特定的一些位段。

其中,绝大部分都被 Memory 吃掉了。(啧

事实上,总线上面跑的绝大部分就是 Virtual Memory Address,虚拟内存地址。

但事实上,其中有(很少)一部分被空出来了,还需要给
Network、Keyboard、Disk、Monitor 用。

在 ICS 里面简化起见,就说这里面的内容全部都是
VMA。(事实上还有一些跟着在总线里跑的东西。)

\hypertarget{header-n149}{%
\paragraph{Broadcasting}\label{header-n149}}

广播。

总线上的两个组件通信时,总线上所有人都可以了解到其中内容,是谓广播。

同时任何一个总线组件也都可以截获任何总线 Transaction。

简单方便,也不安全。

\hypertarget{header-n154}{%
\paragraph{Bus Judgement}\label{header-n154}}

总线仲裁。确认每一时刻总线归谁使用,避免出现中间 Spy。

\hypertarget{header-n156}{%
\subsubsection{Transaction}\label{header-n156}}

咱们来试着来一回 Transaction 吧。

使用的是一种 READY Status 的方式。

总归,总线传递的是一串很长的数字,也可以把它看作是一串 1/0
二进制位的组合。

事先,每个总线上的 Component
都已经了解到了自己所占有的总线位范围,知道了自己应该去 Observe 哪些位。

\begin{quote}
BIOS 会感知你插入了多大块的内存,并首先给 Memory 分配合适的二进制位。

基础的一些设备,如键盘/IDE 都是写死的(IBM 干的),因此始终都需要
Special Fix(特别处理)。

其他的设备,如 Plug-in and Use 型设备都是由 OS 动态分配的。
\end{quote}

\hypertarget{header-n165}{%
\paragraph{Examples}\label{header-n165}}

在 CPU 发出一条指令,如

\begin{verbatim}
LOAD 1742, R1
\end{verbatim}

的时候,出了 CPU 就将其编码为对应的总线请求,Memory
发现这跟我有关了就去读取内存,读出来之后又通过总线还给 CPU,CPU
再将其放在自己的 Register 里。

\hypertarget{header-n169}{%
\subsubsection{NoC}\label{header-n169}}

现代的 Bus 替代品,Network on
Chip,芯片载网络。不仅是一条直路,而包含了更多的支路、甚至路由;形成了一整个网络。

每个请求都包含了一个「地址」+「命令」。比起传统的总线来说已经快得多了。

\hypertarget{header-n172}{%
\subsubsection{Summary}\label{header-n172}}

Processor、Memory、I/O Device

之间是一个三角关系。Processor 和 IO Device,Processor 和 Memory,Memory
和 I/O Device 都是可以互相通信的,via 总线。

\hypertarget{header-n175}{%
\subsubsection{Crash Consistency}\label{header-n175}}

崩溃发生后,能保持一致吗?

如下策略:

\hypertarget{header-n178}{%
\paragraph{\texorpdfstring{ALL\emph{OR}NOTHING}{ALLORNOTHING}}\label{header-n178}}

要么全有要么全完蛋

只要一次在三个地方写,就可以保证恢复的时候能做到
\texttt{ALL\_OR\_NOTHING}。要么全写好要么全恢复。

NEW/OLD is OK。Bad Inconsistency isn't OK。

\hypertarget{header-n182}{%
\paragraph{Recovering from Crash}\label{header-n182}}

上面的这些所有都是文件系统的抽象想象,Disk 本身不知道你在搞些什么的。

\begin{itemize}
\item
  Synchronous meta-data update + fsck
\end{itemize}

Meta-Data 很重要,因此保证写穿!其他的有问题咱们再 fscheck 吧(

\begin{itemize}
\item
  Soft update
\end{itemize}

不展开说。

\begin{itemize}
\item
  Logging/Journaling
\end{itemize}

记日志。目前的 \texttt{Linux} \texttt{ext3/ext4} 都是支持 Logging
的,这种机制可以保证 Consistency。

每次做任何的 Write 操作,都会在日志里面留下特别的印记。

顺序:

\begin{enumerate}
\def\labelenumi{\arabic{enumi}.}
\item
  (D)先把数据块写入 Disk;
\end{enumerate}

\begin{quote}
(如果此时断电,只是白写了一堆 Disk,但是没有 metadata。没关系)
\end{quote}

\begin{enumerate}
\def\labelenumi{\arabic{enumi}.}
\item
  (Jm)再把 MetaData 连着当前时间等等信息写入 Journal 区;
\end{enumerate}

\begin{quote}
(如果此时断电,还是 Nothing,因为原有的 Disk 里面没有被写入任何
MetaData。只是在日志里。)
\end{quote}

\begin{enumerate}
\def\labelenumi{\arabic{enumi}.}
\item
  (Jc)最后进行 Commit。非常非常细小的操作。
\end{enumerate}

\texttt{/*\ Critical.\ 这个操作一定不能出问题。\ */}

\begin{enumerate}
\def\labelenumi{\arabic{enumi}.}
\item
  (M)最后将 MetaData 写入 File System,此时就不需要 Flush
  了。因为就算断电数据丢失,Journal 也可以保证我能恢复这个 MetaData 了。
\end{enumerate}

假如我们直接往 Disk 里写
MetaData,很可能在开始写的时候破坏了原有的数据,但在中途断电的时候无法回去。

\begin{quote}
Ext 文件系统之中有个模式叫做 Full Journal
Mode(连同数据也一起写在日志里。您想想这多慢呢\ldots)

还有个模式叫做 Ordered Journal Mode。数据不进日志,但是 MetaData
进日志。

这是默认模式。因为 FJM 太慢了\ldots{}
\end{quote}

\hypertarget{header-n221}{%
\subsubsection{Improvements: Journaling without
Ordering}\label{header-n221}}

加一个 Checksum 校验和,将 Data 和 MetaData 共同计算出一个校验和放置在
Commit 末尾。

\hypertarget{header-n223}{%
\section{CSE 期中考试悲报}\label{header-n223}}

10 月 29 日星期二

\end{document}
