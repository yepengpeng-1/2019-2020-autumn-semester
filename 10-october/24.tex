% Options for packages loaded elsewhere
\PassOptionsToPackage{unicode}{hyperref}
\PassOptionsToPackage{hyphens}{url}
%
\documentclass[
]{article}
\usepackage{lmodern}
\usepackage{amssymb,amsmath}
\usepackage{ifxetex,ifluatex}
\ifnum 0\ifxetex 1\fi\ifluatex 1\fi=0 % if pdftex
  \usepackage[T1]{fontenc}
  \usepackage[utf8]{inputenc}
  \usepackage{textcomp} % provide euro and other symbols
\else % if luatex or xetex
  \usepackage{unicode-math}
  \defaultfontfeatures{Scale=MatchLowercase}
  \defaultfontfeatures[\rmfamily]{Ligatures=TeX,Scale=1}
\fi
% Use upquote if available, for straight quotes in verbatim environments
\IfFileExists{upquote.sty}{\usepackage{upquote}}{}
\IfFileExists{microtype.sty}{% use microtype if available
  \usepackage[]{microtype}
  \UseMicrotypeSet[protrusion]{basicmath} % disable protrusion for tt fonts
}{}
\makeatletter
\@ifundefined{KOMAClassName}{% if non-KOMA class
  \IfFileExists{parskip.sty}{%
    \usepackage{parskip}
  }{% else
    \setlength{\parindent}{0pt}
    \setlength{\parskip}{6pt plus 2pt minus 1pt}}
}{% if KOMA class
  \KOMAoptions{parskip=half}}
\makeatother
\usepackage{xcolor}
\IfFileExists{xurl.sty}{\usepackage{xurl}}{} % add URL line breaks if available
\IfFileExists{bookmark.sty}{\usepackage{bookmark}}{\usepackage{hyperref}}
\hypersetup{
  hidelinks,
  pdfcreator={LaTeX via pandoc}}
\urlstyle{same} % disable monospaced font for URLs
\usepackage{color}
\usepackage{fancyvrb}
\newcommand{\VerbBar}{|}
\newcommand{\VERB}{\Verb[commandchars=\\\{\}]}
\DefineVerbatimEnvironment{Highlighting}{Verbatim}{commandchars=\\\{\}}
% Add ',fontsize=\small' for more characters per line
\newenvironment{Shaded}{}{}
\newcommand{\AlertTok}[1]{\textcolor[rgb]{1.00,0.00,0.00}{\textbf{#1}}}
\newcommand{\AnnotationTok}[1]{\textcolor[rgb]{0.38,0.63,0.69}{\textbf{\textit{#1}}}}
\newcommand{\AttributeTok}[1]{\textcolor[rgb]{0.49,0.56,0.16}{#1}}
\newcommand{\BaseNTok}[1]{\textcolor[rgb]{0.25,0.63,0.44}{#1}}
\newcommand{\BuiltInTok}[1]{#1}
\newcommand{\CharTok}[1]{\textcolor[rgb]{0.25,0.44,0.63}{#1}}
\newcommand{\CommentTok}[1]{\textcolor[rgb]{0.38,0.63,0.69}{\textit{#1}}}
\newcommand{\CommentVarTok}[1]{\textcolor[rgb]{0.38,0.63,0.69}{\textbf{\textit{#1}}}}
\newcommand{\ConstantTok}[1]{\textcolor[rgb]{0.53,0.00,0.00}{#1}}
\newcommand{\ControlFlowTok}[1]{\textcolor[rgb]{0.00,0.44,0.13}{\textbf{#1}}}
\newcommand{\DataTypeTok}[1]{\textcolor[rgb]{0.56,0.13,0.00}{#1}}
\newcommand{\DecValTok}[1]{\textcolor[rgb]{0.25,0.63,0.44}{#1}}
\newcommand{\DocumentationTok}[1]{\textcolor[rgb]{0.73,0.13,0.13}{\textit{#1}}}
\newcommand{\ErrorTok}[1]{\textcolor[rgb]{1.00,0.00,0.00}{\textbf{#1}}}
\newcommand{\ExtensionTok}[1]{#1}
\newcommand{\FloatTok}[1]{\textcolor[rgb]{0.25,0.63,0.44}{#1}}
\newcommand{\FunctionTok}[1]{\textcolor[rgb]{0.02,0.16,0.49}{#1}}
\newcommand{\ImportTok}[1]{#1}
\newcommand{\InformationTok}[1]{\textcolor[rgb]{0.38,0.63,0.69}{\textbf{\textit{#1}}}}
\newcommand{\KeywordTok}[1]{\textcolor[rgb]{0.00,0.44,0.13}{\textbf{#1}}}
\newcommand{\NormalTok}[1]{#1}
\newcommand{\OperatorTok}[1]{\textcolor[rgb]{0.40,0.40,0.40}{#1}}
\newcommand{\OtherTok}[1]{\textcolor[rgb]{0.00,0.44,0.13}{#1}}
\newcommand{\PreprocessorTok}[1]{\textcolor[rgb]{0.74,0.48,0.00}{#1}}
\newcommand{\RegionMarkerTok}[1]{#1}
\newcommand{\SpecialCharTok}[1]{\textcolor[rgb]{0.25,0.44,0.63}{#1}}
\newcommand{\SpecialStringTok}[1]{\textcolor[rgb]{0.73,0.40,0.53}{#1}}
\newcommand{\StringTok}[1]{\textcolor[rgb]{0.25,0.44,0.63}{#1}}
\newcommand{\VariableTok}[1]{\textcolor[rgb]{0.10,0.09,0.49}{#1}}
\newcommand{\VerbatimStringTok}[1]{\textcolor[rgb]{0.25,0.44,0.63}{#1}}
\newcommand{\WarningTok}[1]{\textcolor[rgb]{0.38,0.63,0.69}{\textbf{\textit{#1}}}}
\usepackage[normalem]{ulem}
% Avoid problems with \sout in headers with hyperref
\pdfstringdefDisableCommands{\renewcommand{\sout}{}}
\setlength{\emergencystretch}{3em} % prevent overfull lines
\providecommand{\tightlist}{%
  \setlength{\itemsep}{0pt}\setlength{\parskip}{0pt}}
\setcounter{secnumdepth}{-\maxdimen} % remove section numbering

\date{}

\begin{document}

\hypertarget{header-n0}{%
\section{Oct 24 Thu}\label{header-n0}}

\hypertarget{header-n2}{%
\subsection{SE-302::Compilers}\label{header-n2}}

Today Widgets

一点小补充。对应书本 6.2.7:

Keeping Track of Level.

Translate
负责整理静态链相关的问题。考虑到静态链相关的事情都是「语言相关」的东西,所以这些东西不该交给下面的后端,而应该在编译成中间语言之前就解决掉。

留意到 main
函数是个特别的函数。这是一个不位于任何函数内的函数;但应该有权了解到所有位置的变量。

因此在「没有父级的函数」里 call \texttt{Tr\_newlevel} 时,将其定义为
\texttt{Tr\_outermost} 的返回值;这是专为最外层级设计的一个标识。

\hypertarget{header-n9}{%
\subsubsection{Trees Language}\label{header-n9}}

何为树语言?

是一种中间表示(IR, Intermediate Representation)型语言。IR
的目的是在需要将多种语言编译到多种 Target Architecture
的时候,我们可以通过 IR
来做一个中介;以便减少我们犯错的几率,节省工作量。

IR 需要具有这样的条件:

\begin{itemize}
\item
  和具体的语言无关
\item
  足够简单(不能太难翻译和表示)
\item
  功能强大(以便表示各种高级语言)
\item
  和平台无关,能编译到多种 Target Arch。
\end{itemize}

因此我们推出了这一种语言:TreesLang。

TreesLang 已经非常接近底层了;除了在 TreesLang
中有无限的虚拟寄存器可用。

\hypertarget{header-n24}{%
\subsubsection{Expression}\label{header-n24}}

在 Tiger 中,所有的语句都可以作为 Expression。(想想 Lab 3 里的 xxExp)

但是到了 TreesLang 中之后,我们不这么认为了。

主要分为下面 7 类:

\hypertarget{header-n28}{%
\paragraph{Constant}\label{header-n28}}

\texttt{CONST(i)}:整型常数。

\begin{Shaded}
\begin{Highlighting}[]
\NormalTok{T_Const(i);}
\end{Highlighting}
\end{Shaded}

\hypertarget{header-n31}{%
\paragraph{Name}\label{header-n31}}

(事实上起的是 Symbolic Constant 的作用)

\texttt{NAME(n)}:符号常数 n,代表了以 n 为名的符号。

\begin{Shaded}
\begin{Highlighting}[]
\NormalTok{T_Name(n);}
\end{Highlighting}
\end{Shaded}

\hypertarget{header-n35}{%
\paragraph{Temporary}\label{header-n35}}

(就是上面提到的,无限多的 Virtual Registers)

\texttt{TEMP(t)}:临时变量,或者说虚拟寄存器。

\hypertarget{header-n38}{%
\paragraph{Binary Operation}\label{header-n38}}

(二元操作符,如加减乘除和移位操作等。)

\texttt{BINOP(o,\ e1,\ e2)}:对 e1 和 e2 两个操作数施加 o 操作。

\hypertarget{header-n41}{%
\paragraph{Memory Access}\label{header-n41}}

(访存操作。)

\texttt{MEM(e)},默认拿出 e 内存地址处的 wordSize 个字节的内容。

\hypertarget{header-n44}{%
\paragraph{Expression Sequence}\label{header-n44}}

(一串儿表达式。)

\texttt{ESEQ(s,\ e)}:考虑到表达式的副作用问题,s 会被先执行,而 e
会被后执行,然后把 e 的返回值作为整个表达式串的返回值返回。

也是为了 Tiger
强加的一个语法(因此要花整个第八章来处理)。正常人类也不用这个。

\hypertarget{header-n48}{%
\paragraph{Procedure Call}\label{header-n48}}

(对函数(或者过程)的调用。)

\texttt{CALL(f,\ l)}:表示过程调用。用参数列表 l 来调用函数 f;但 f
作为函数指针也可能是个表达式。

因此会首先计算 f 内容(如果是一个表达式产生的
Function),再从左到右地计算参数列表;

最后施加调用。

\hypertarget{header-n53}{%
\paragraph{Statement Classifies}\label{header-n53}}

(Statement 是不需要计算出 Value 的表达式,但会产生副作用。)

(如果连副作用都没有,那他有什么意义呢)

\hypertarget{header-n56}{%
\subparagraph{Move}\label{header-n56}}

\texttt{MOVE(TEMP\ t,\ e)}:把 expression 的计算值赋给虚拟寄存器 t。

\texttt{MOVE(MEM(e1),\ e2)}:先计算
e1(Expression),找出指定的内存地址;然后计算
e2,把计算结果赋给该内存地址。

\hypertarget{header-n59}{%
\subparagraph{Expression}\label{header-n59}}

\texttt{EXP(e)}:计算表达式 e
但忽略其结果。(只想要他的副作用。(那这还叫副作用吗?))

\hypertarget{header-n61}{%
\subparagraph{Jump!}\label{header-n61}}

\texttt{JUMP(e,\ labs)}:把控制权交给 e 对应的代码段处。

e 可以是一个 Label;也可以是一个内存地址;也可以是一个
labs(跳转表)的索引。

\texttt{CJUMP(o,\ e1,\ e2,\ t,\ f)}:条件跳转。依次计算出 e1 和 e2
的结果,用 o 对其进行比较;成立时跳转到 t 对应的代码段处;不成立时跳转到
f 对应的代码段处。

\hypertarget{header-n65}{%
\subparagraph{Label}\label{header-n65}}

\texttt{LABEL(n)}:将 n 的值设定为当前机器代码(PC)的地址。

\hypertarget{header-n67}{%
\subsubsection{AST Expressions}\label{header-n67}}

\hypertarget{header-n68}{%
\paragraph{归类}\label{header-n68}}

\begin{itemize}
\item
  Ex::AST Exp that compute values
\end{itemize}

表示为 \texttt{T\_exp}。

\begin{itemize}
\item
  Nx::AST Exp that returns no value
\end{itemize}

表示为 \texttt{T\_stm}。Nx 的意思是:无结果语句。

\begin{itemize}
\item
  Cx::AST Exp with Boolean values
\end{itemize}

仅在 Conditional Jump 时使用。Cx 的意思是:条件语句。

因为我们需要进行 Shortcut
Judgement(短路求值),因此所有的布尔都可以翻译为 Conditional Jump。

\hypertarget{header-n82}{%
\paragraph{问题}\label{header-n82}}

说到短路求值,我们来看个例子。

对于 Tiger 表达式
\texttt{a\ \textgreater{}\ b\ \textbar{}\ c\ \textless{}\ d},我们按道理应该把它翻成一个
Cx。

但是问题是,在我们翻译完成整条语句之前,我们并不能知道 t
的时候该去哪里、f 的时候该去哪里。

具体的指令位置还远远没定下来。

因此在不了解情况的时候,我们首先将其使用 NULL(or nullptr, if
std\textgreater=c++11。)来填充位置。

等我们知道了之后(这真的在很久之后,至少要生成了汇编代码才知道),再去回填这些数据。

\hypertarget{header-n89}{%
\paragraph{True Patch List \& False Counterpart}\label{header-n89}}

为了保证上面我们的「回填数据法」可以行得通,我们需要维护两张表:

一张「真值标号回填表」;另一张「假值标号回填表」。

表中记录那些「t 已知的时候,应该去填充 NULL 的位置」;

以及「f 已知的时候,应该去填充 NULL 的位置」。

\hypertarget{header-n94}{%
\paragraph{Name := Cx?}\label{header-n94}}

在我们的 Conditional Expression 被拿来当作表达式来赋值的时候,怎么处理?

这相当于是把一个 Cx 当作 Ex 用。

这里有一些 Utility Functions 来做简单表达式类型的转换。

\begin{Shaded}
\begin{Highlighting}[]
\AttributeTok{static}\NormalTok{ T_exp unEx(Tr_exp e);}
\AttributeTok{static}\NormalTok{ T_stm unNx(Tr_exp e);}
\AttributeTok{static} \KeywordTok{struct}\NormalTok{ Cx unCx(Tr_exp e);}
\end{Highlighting}
\end{Shaded}

留意到无论传入的 Tr\_exp 本来为什么类型,三个函数都要能正确处理;

并正确剥离出 T\emph{exp、T}stm 或 struct Cx 的类型。

另外,Cx 的参数为 0 或 1(Constant)时的 trivial
情况应该特别处理;因为可以很容易地减少额外开销。

试图对一个 Tr\_nx 的 exp 进行 unCx 是不合理的。程序应该能处理这种异常。

最简单的办法应该就是:挑一个虚拟寄存器来作为中转\sout{(反正 Virtual
Registers 不要钱)}

然后简单进行一层 Wrapping 就好。

\hypertarget{header-n105}{%
\subsubsection{Translation}\label{header-n105}}

现在我们要真的开始 Translation 了。

\hypertarget{header-n107}{%
\paragraph{Simple Variables}\label{header-n107}}

简单变量的翻译?简单如其名。

\hypertarget{header-n109}{%
\subparagraph{Access Variables}\label{header-n109}}

Tr\emph{Exp Tr}SimpleVar(Tr\emph{Access, Tr}level);

比如,我们这里需要访问一个变量,就只需要把它 Move 到一个 Memory
地址里或虚拟寄存器里就可以了。

为此我们有一个新鲜的 struct:

\begin{Shaded}
\begin{Highlighting}[]
\KeywordTok{struct}\NormalTok{ Tr_access_ \{}
\NormalTok{	Tr_level level;}
\NormalTok{    F_access access;}
\NormalTok{\};}
\end{Highlighting}
\end{Shaded}

Tr\_level 是从 environment 中拿到的信息;标识了访问变量的上下文。

F\_access
则是告诉了我们最终应该从哪儿去拿这个变量;寄存器里?内存里?它告诉我们。

\hypertarget{header-n116}{%
\subparagraph{Rules}\label{header-n116}}

尽量不要污染 Semantic
部分。考虑一下我们现在正在做的事情。将生成的语法树变换成中间语言的形式。

这个过程不应该依赖于任何特定语言的 Semantic Analysis,而应该尽可能独立。

不要为了图方便而去用 Semantic 里面开的后门。

\hypertarget{header-n120}{%
\subparagraph{Static Links}\label{header-n120}}

鉴于上一节我们讲过的静态链操作,我们在访问非本级别 Scope
中定义的变量的时候,需要沿着静态链爬 n 步才能到达实际的 Variable 位置。

\hypertarget{header-n122}{%
\paragraph{Subscription \& Field Selection}\label{header-n122}}

访问数组比起访问简单变量来说,要稍微麻烦一些。

对 Tiger 而言,数组访问比起其他语言要更特别一些。

\texttt{a{[}i{]}} =\textgreater{} MEM(+(MEM(e), CONST offset f))。

不像 C 那样,Tiger 的数组名称并不是「指向数组头元素的指针」。

C 中,你不能把一个数组作为 \texttt{:=} 左边的值来赋值。而 Tiger
可以这么做。

数组符号本身是在栈(Stack)上的。然而数组的内容则是分配在堆(Heap)上面的。

在 Scope 退出、数组符号本身被退栈之后,Heap 中的内容会交由 GC
机制来回收(See Chapter 13)。

访问一个 Record 的成员变量,跟 Subscription 也很类似。

只是 offset 的计算方式有别;本质上并无区别。

\texttt{A.f} =\textgreater{} MEM(+(MEM(e), CONST offset f))。

\hypertarget{header-n133}{%
\paragraph{lvalue?}\label{header-n133}}

「Structured Left-Value」::结构化的左值

留意到简单变量名、Subscription 的结果、以及 Field Selection
的结果都是「可被赋值的」。

也就是,可以出现在赋值符号 \texttt{:=} 左边的名字。

同时,类似于 \texttt{42},\texttt{a\ +\ 2}
之类的值就不是可被赋值的名字。他们不是左值。

虽说叫做左值,但是其实也可以出现在 \texttt{:=}
右边,此时隐含地(Implicitly)作访问其值的意思讲。

技术上来说,左值应该被作为一个 Memory Address 来存储;

出现在赋值号左边时,应该将其作为 Mem. Addr. 来解读;

出现在赋值号右边时,应该隐式地将其解读为这个内存地址里存放的值来解读。

\hypertarget{header-n142}{%
\subsubsection{Criticizing}\label{header-n142}}

\begin{quote}
作者要开始批判一番了
\end{quote}

\hypertarget{header-n145}{%
\paragraph{Bound Checking}\label{header-n145}}

Java 是会对数组的访问去做 Bound Checking(边界检查)的;Java
也根本不提供裸指针用。

然而在编译阶段的时候,在检查通过之后,会消去这部分冗余代码。

\hypertarget{header-n148}{%
\paragraph{Null Checking}\label{header-n148}}

空指针检查:这也是应该执行的检查。

\hypertarget{header-n150}{%
\subsubsection{What does the Tiger say?}\label{header-n150}}

Tiger 里不提供 Unary(原生的负值支持,类似于 -1)。

不提供 Floating Point 的支持。

不提供一元的位操作(如取反)。但却提供了 XoR(跟自己 XoR
也能实现取反操作)

\end{document}
