% Options for packages loaded elsewhere
\PassOptionsToPackage{unicode}{hyperref}
\PassOptionsToPackage{hyphens}{url}
%
\documentclass[
]{article}
\usepackage{lmodern}
\usepackage{amssymb,amsmath}
\usepackage{ifxetex,ifluatex}
\ifnum 0\ifxetex 1\fi\ifluatex 1\fi=0 % if pdftex
  \usepackage[T1]{fontenc}
  \usepackage[utf8]{inputenc}
  \usepackage{textcomp} % provide euro and other symbols
\else % if luatex or xetex
  \usepackage{unicode-math}
  \defaultfontfeatures{Scale=MatchLowercase}
  \defaultfontfeatures[\rmfamily]{Ligatures=TeX,Scale=1}
\fi
% Use upquote if available, for straight quotes in verbatim environments
\IfFileExists{upquote.sty}{\usepackage{upquote}}{}
\IfFileExists{microtype.sty}{% use microtype if available
  \usepackage[]{microtype}
  \UseMicrotypeSet[protrusion]{basicmath} % disable protrusion for tt fonts
}{}
\makeatletter
\@ifundefined{KOMAClassName}{% if non-KOMA class
  \IfFileExists{parskip.sty}{%
    \usepackage{parskip}
  }{% else
    \setlength{\parindent}{0pt}
    \setlength{\parskip}{6pt plus 2pt minus 1pt}}
}{% if KOMA class
  \KOMAoptions{parskip=half}}
\makeatother
\usepackage{xcolor}
\IfFileExists{xurl.sty}{\usepackage{xurl}}{} % add URL line breaks if available
\IfFileExists{bookmark.sty}{\usepackage{bookmark}}{\usepackage{hyperref}}
\hypersetup{
  hidelinks,
  pdfcreator={LaTeX via pandoc}}
\urlstyle{same} % disable monospaced font for URLs
\usepackage{color}
\usepackage{fancyvrb}
\newcommand{\VerbBar}{|}
\newcommand{\VERB}{\Verb[commandchars=\\\{\}]}
\DefineVerbatimEnvironment{Highlighting}{Verbatim}{commandchars=\\\{\}}
% Add ',fontsize=\small' for more characters per line
\newenvironment{Shaded}{}{}
\newcommand{\AlertTok}[1]{\textcolor[rgb]{1.00,0.00,0.00}{\textbf{#1}}}
\newcommand{\AnnotationTok}[1]{\textcolor[rgb]{0.38,0.63,0.69}{\textbf{\textit{#1}}}}
\newcommand{\AttributeTok}[1]{\textcolor[rgb]{0.49,0.56,0.16}{#1}}
\newcommand{\BaseNTok}[1]{\textcolor[rgb]{0.25,0.63,0.44}{#1}}
\newcommand{\BuiltInTok}[1]{#1}
\newcommand{\CharTok}[1]{\textcolor[rgb]{0.25,0.44,0.63}{#1}}
\newcommand{\CommentTok}[1]{\textcolor[rgb]{0.38,0.63,0.69}{\textit{#1}}}
\newcommand{\CommentVarTok}[1]{\textcolor[rgb]{0.38,0.63,0.69}{\textbf{\textit{#1}}}}
\newcommand{\ConstantTok}[1]{\textcolor[rgb]{0.53,0.00,0.00}{#1}}
\newcommand{\ControlFlowTok}[1]{\textcolor[rgb]{0.00,0.44,0.13}{\textbf{#1}}}
\newcommand{\DataTypeTok}[1]{\textcolor[rgb]{0.56,0.13,0.00}{#1}}
\newcommand{\DecValTok}[1]{\textcolor[rgb]{0.25,0.63,0.44}{#1}}
\newcommand{\DocumentationTok}[1]{\textcolor[rgb]{0.73,0.13,0.13}{\textit{#1}}}
\newcommand{\ErrorTok}[1]{\textcolor[rgb]{1.00,0.00,0.00}{\textbf{#1}}}
\newcommand{\ExtensionTok}[1]{#1}
\newcommand{\FloatTok}[1]{\textcolor[rgb]{0.25,0.63,0.44}{#1}}
\newcommand{\FunctionTok}[1]{\textcolor[rgb]{0.02,0.16,0.49}{#1}}
\newcommand{\ImportTok}[1]{#1}
\newcommand{\InformationTok}[1]{\textcolor[rgb]{0.38,0.63,0.69}{\textbf{\textit{#1}}}}
\newcommand{\KeywordTok}[1]{\textcolor[rgb]{0.00,0.44,0.13}{\textbf{#1}}}
\newcommand{\NormalTok}[1]{#1}
\newcommand{\OperatorTok}[1]{\textcolor[rgb]{0.40,0.40,0.40}{#1}}
\newcommand{\OtherTok}[1]{\textcolor[rgb]{0.00,0.44,0.13}{#1}}
\newcommand{\PreprocessorTok}[1]{\textcolor[rgb]{0.74,0.48,0.00}{#1}}
\newcommand{\RegionMarkerTok}[1]{#1}
\newcommand{\SpecialCharTok}[1]{\textcolor[rgb]{0.25,0.44,0.63}{#1}}
\newcommand{\SpecialStringTok}[1]{\textcolor[rgb]{0.73,0.40,0.53}{#1}}
\newcommand{\StringTok}[1]{\textcolor[rgb]{0.25,0.44,0.63}{#1}}
\newcommand{\VariableTok}[1]{\textcolor[rgb]{0.10,0.09,0.49}{#1}}
\newcommand{\VerbatimStringTok}[1]{\textcolor[rgb]{0.25,0.44,0.63}{#1}}
\newcommand{\WarningTok}[1]{\textcolor[rgb]{0.38,0.63,0.69}{\textbf{\textit{#1}}}}
\setlength{\emergencystretch}{3em} % prevent overfull lines
\providecommand{\tightlist}{%
  \setlength{\itemsep}{0pt}\setlength{\parskip}{0pt}}
\setcounter{secnumdepth}{-\maxdimen} % remove section numbering

\date{}

\begin{document}

\hypertarget{header-n0}{%
\section{Oct 10 Thu}\label{header-n0}}

\hypertarget{header-n2}{%
\subsection{SE-227}\label{header-n2}}

今天我们要讲的是从头到底的一个服务器搭建。

在服务器所需要提供的服务量越来越大的时候,我们该怎么应对?

1) 买更强的大型机器。

2) 买更多便宜的小机器。

方法 1
很好懂,总归就是烧钱。但总归技术发展是有瓶颈的,单个服务器的能力总归是有限的。

便宜的小机器的限制就小很多。总归小机器很便宜,大不了多买几个就行了。

但问题就来了:如何才能让这么多小机器互相协作呢?

而且,最好是我们不需要改已有的现有代码。

\hypertarget{header-n11}{%
\subsubsection{RPC:Remote Procedure Call}\label{header-n11}}

像读写本地文件系统一样,直接读写网络的文件,可以支持多台机器同时连接到网络,并且远程机器会自动加好锁并处理好资源竞争问题。本地只管读,好像这块盘就是本地的一样。

\hypertarget{header-n13}{%
\subsubsection{如何移植到 RPC?}\label{header-n13}}

大概是这样:

Client 只需要这么处理发送给 Server 的请求:

\begin{Shaded}
\begin{Highlighting}[]
\NormalTok{client->call(protocol::method_name, parameter_1, parameter_2, ...);}
\end{Highlighting}
\end{Shaded}

等价于直接调用

\begin{Shaded}
\begin{Highlighting}[]
\NormalTok{method_name(parameter_1, parameter_2, ...);}
\end{Highlighting}
\end{Shaded}

但如果说我们不希望修改我们原有的那些代码(也就是不想动那些祖传代码),该怎么办呢?

Stub。打桩。自动生成一些无聊的代码。

截获那些直接的函数调用,把 \texttt{method\_name} 内部改成
\texttt{client-\textgreater{}call},并且处理状态码。

这样就可以保证用户态代码不做任何多余改动,直接就能实现 RPC。

当然 Server 代码也需要对应地修改,只是工作量要小很多。

\hypertarget{header-n24}{%
\subsubsection{Marshalling / Unmarshalling}\label{header-n24}}

Marshall 的意思:本意是马歇尔将军(\textbf{George Catlett
Marshall})的名字,后来被引申到将军、列队、集结、结集、编码、编组、编集、安整、数据打包、列集等稀奇古怪的意思。

因为我们 Client 跟 Server
的函数调用都是美好的、基于函数调用的抽象。然而这种抽象是没办法(直接)在网路上传输并用于
RPC,因为 网络上传输的只能是字节流。

因此为了传输,我们得把这种函数调用给他拍扁了。

C++ 中的参数传递有两种方式:传值和传引用。

但是呢?Client 跟 Server 根本不是一台机器。你传一个 Pointer 或者说是
Reference(本质上也就是一个给小孩子用的 Pointer)都没有意义。

因为你拿着这个 Client 给的指针去 Server,也找不到那个 Object。

所以我们必须把这个 Pointer 指向的 Object 也得拷贝并发送过去。

如果这个 Object 里还有 Pointer\ldots 也得一层层 Copy 过去。

深深地拷贝(Deep
Copy)这些东西,并打成一个数据包之后,才能通过网络来发送。

这个过程叫做 Marshalling。

反过来,在服务器端要把收到的这些数据包反过来构造成 Object
并重建指针关系的这个过程就叫做 Unmarshalling。

\begin{verbatim}
Wikipedia

Marshal 和 Serialize 的比较

"marshal"一个对象意味着记录下它的状态与codebase(s)在这种方式,以便当这个marshal对象在被"unmarshalled"时,可以获得最初代码的一个副本,可能会自动装入这个对象的类定义。可以marshal任何能被序列化或远程(即实现java.rmi.Remote接口)。Marshalling类似序列化,除了marshalling也记录codebases。Marshalling不同于序列化是marshalling特别处理远程对象。
\end{verbatim}

看这种机器翻译,就跟看陈昊鹏翻的 Java 书一样。就是说还不如去看英文原版。

\hypertarget{header-n39}{%
\subsubsection{意义}\label{header-n39}}

咱们费了这么大劲,把我们原有的函數調用給改成了「又是發送給遠端服務器,又是列隊又是解散」的,好不容易能用了。那你能不能告訴我這有什麼好處呢?

這都是為了分佈式啊!現在所有的請求全部都變成了異步的請求。這樣就可以讓所有需要等待的事情(如
I/O 等)都拋給服務器來處理,其他的所有東西我們都不用再關心。

\hypertarget{header-n42}{%
\subsubsection{NFS}\label{header-n42}}

Network File System。那是啥?

一个分布式的文件系统。

\hypertarget{header-n45}{%
\subsubsection{File Handler}\label{header-n45}}

为了处理每一个文件的先后关系,我们总得给他上锁。

假如说我们每次有一个读都去给整个文件系统上锁\ldots 那这个效率也太低了。

我们最好单独地给文件上锁,有必要的时候锁上整个目录。

仅仅使用 inode 不能完全确定这个文件。考虑这种情况:

A 来读这个 inode,B 把这个 inode 对应的文件删掉了,C
又创建了一个新文件,刚好又分配到了这个 inode。

那么对于 A 来说,这就是失败了。不正确了。

因此除了 inode,我们还要加一个 Generation Number(代际数),inode
每被分配一次,都会增加一个 Generation Number。

只有 inode Number + Generation Number 才能唯一确认一把锁。

主要原因还是 NFS 是 Stateless
的。无状态确定了它必须要额外的东西(就是这个 Generation
Number)来确认状态,就像 HTTP 里的小饼干。

本机是不可能出现这种问题的。 File Description 跟 iNode 是 OS
内记住了的。OS 有记忆。

Stateless NFS 没有。需要 G.N. 帮助。

在我们发现 inode 一样但是 Generation Number
不一样的时候,就返回一个错误「Stale inode Number」,就是说您的 inode
number 不新鲜了。

\hypertarget{header-n58}{%
\subsubsection{CAP Theorem}\label{header-n58}}

在理论计算机科学中,CAP定理(CAP theorem),又被称作布鲁尔定理(Brewer's
theorem),它指出对于一个分布式计算系统来说,不可能同时满足以下三点:

\begin{itemize}
\item
  一致性(Consistency) (等同于所有节点访问同一份最新的数据副本)
\item
  可用性(Availability)(每次请求都能获取到非错的响应------但是不保证获取的数据为最新数据)
\item
  分区容错性(Partition
  tolerance)(以实际效果而言,分区相当于对通信的时限要求。系统如果不能在时限内达成数据一致性,就意味着发生了分区的情况,必须就当前操作在C和A之间做出选择。)
\end{itemize}

\hypertarget{header-n67}{%
\subsubsection{幂等操作}\label{header-n67}}

一定要保证幂等操作:做 0 次就是做 0 次(没做就是没做),做 N 次(N
\textgreater= 1)等于做 1 次。

因为事实上互联网的发送包是很不靠谱的。你可能会丢包。

丢包了你就要重发。

但是你认为丢了的包也不一定丢了。

\begin{quote}
或许是对方的 Response 包丢了。
\end{quote}

\begin{quote}
或许是对方的 Response 来得比你的 Threshold 晚了一点。
\end{quote}

\begin{quote}
或许是对方 Response
要来了的时候,你的机器断电了。在重启之后包已经不在了,你以为没发到其实已经发到了。
\end{quote}

就是说你可能失手发了多个一样的包。而且对方也都收到了。

如果不是幂等操作,这种事情就会导致你发出去的包意义变了。

\hypertarget{header-n80}{%
\subsubsection{Eternal Question}\label{header-n80}}

永远不知道是对方没收到我的包

还是对方收到了我的包但我却没收到他的回复包。

\hypertarget{header-n83}{%
\subsection{SE-344::Computer Graphics}\label{header-n83}}

不论你要 Build 出如何真实、如何帅气、如何有意思的图形,

最终你还是要把它给 Render, Flatten 成一个个无趣的像素点的。

这个过程就叫做「光栅化技术」。

\begin{center}\rule{0.5\linewidth}{\linethickness}\end{center}

\hypertarget{header-n88}{%
\subsubsection{光栅化技術}\label{header-n88}}

光栅化之前所有的操作還可以包含 Vertex(頂點)、拓撲關係等等幾何信息;

但光栅化操作會把這些全部拍平;

光栅化之後的所有操作都只能基於 Pixels(像素點)了。

\hypertarget{header-n92}{%
\paragraph{基本圖元繪製算法}\label{header-n92}}

我們要瞭解的僅僅就是基礎的算法原理而已。因為目前所有的基礎庫都很好地替我們提供了優秀的實現,我們不必要自己實現這些算法。

(雖然在下次作業裡,我們還是會需要做它一次。)

\hypertarget{header-n95}{%
\paragraph{圖形的表示}\label{header-n95}}

\begin{itemize}
\item
  WireFrame,線框模型
\end{itemize}

僅僅採用頂點和鄰邊來表示模型。

如果只有三維的線框模型的話,這種表示是存在二義性的。由於不存在遮擋關係,沒辦法唯一地表現出空間的。

只是快速表示一下空間而已。

\begin{itemize}
\item
  Surface,表面模型
\end{itemize}

這種表示方法相較於 WireFrame,是由很多個
Surface(面片)搭建成的。由於面片之間存在遮擋關係,因此就表示精確度而言高於
WireFrame 模型。

但他沒有說明這圖形是由表面的哪一邊組成的。

\begin{itemize}
\item
  Solid,實體模型
\end{itemize}

這種模型表示方法是可以定義形體處於表面的哪一側的。

還有很多種圖形的表示方法:

\begin{itemize}
\item
  表面多邊形法(B-rep)
\item
  解析法、微分法
\item
  掃描表示法
\item
  四叉樹、八叉樹表示法
\item
  分形幾何法
\item
  粒子系統(Particle System)
\end{itemize}

\hypertarget{header-n125}{%
\paragraph{圖形的顯示}\label{header-n125}}

顯示跟表示是不一樣的。

顯示文件僅僅是整個圖形表示的一部分。

\begin{center}\rule{0.5\linewidth}{\linethickness}\end{center}

取景窗口也决定了显示的内容。这个取景窗口的决定叫做 Clipping。

\begin{itemize}
\item
  每一个可以独立显示的点叫做像素点「Pixel Points」。
\item
  一行像素构成一条光栅(Raster),也叫做扫描线(Scanline)。
\end{itemize}

确定屏幕上像素点的集合和颜色,叫做光栅化(Rasterization),也叫做扫描线算法(Scanline
Algorithms)。

多年以前的 CRT 显示器需要扫描显示。因此它就有两种不同的扫描方法:

\begin{itemize}
\item
  奇数场优先;
\end{itemize}

奇数场优先就是优先画出奇数场的扫描线,然后回扫;(回到第一个地方)而偶数场的扫描线就先暂且留在上一帧的位置。

\begin{itemize}
\item
  偶数场优先;
\end{itemize}

和奇数场优先相反。

\begin{itemize}
\item
  逐行显示
\end{itemize}

不区分上下/奇偶场。

\hypertarget{header-n149}{%
\paragraph{帧缓冲存储器}\label{header-n149}}

存储器中包含了每个像素的信息。

拿出对应像素点的信息,放到对应的寄存器(Register)之中,然后送给显示控制器来干活;最终显示出来。

彩色的呢复杂一点。

有三个独立的显示控制器(R/G/B)共同显示彩色。

因此就需要三重显示控制器。

\begin{quote}
P.S. 部分的显示系统还包括了 Alpha
通道(也就是不透明度的控制)。但是并不存在一种「Alpha
显示控制器」。在此之前他就已经被平摊到 R/G/B 通道上了。
\end{quote}

\hypertarget{header-n157}{%
\subsubsection{OpenGL 的基本图元绘制}\label{header-n157}}

\hypertarget{header-n158}{%
\paragraph{定义顶点}\label{header-n158}}

顶点是一切东西的开始和基础。

由它才产生出了点、线、多边形,等等。

\hypertarget{header-n161}{%
\subparagraph{\texorpdfstring{\texttt{glVertex*}}{glVertex*}}\label{header-n161}}

\begin{quote}
\texttt{*} 代表这个函数有后缀。合理的后缀形似:
\end{quote}

\begin{itemize}
\item
  glVertex2i
\end{itemize}

2 代表这个顶点在二维空间之中。

i 代表这个点的坐标值为整数。

s 代表这个点的坐标值为段整数。

\begin{itemize}
\item
  glVertex3f
\end{itemize}

3 代表这个顶点在三维空间之中。

f 代表这个点的坐标值为单精度浮点数。

d 代表这个点的坐标值为双精度浮点数。

以此类推。

\begin{Shaded}
\begin{Highlighting}[]
\NormalTok{glBegin(GL_POINTS);}

\NormalTok{...}

\NormalTok{glEnd();}
\end{Highlighting}
\end{Shaded}

\texttt{glBegin} 和 \texttt{glEnd} 之间的部分就是我们定义的顶点们。

glBegin 中有一些参数,决定了怎么处理 glBegin 和 glEnd 之间定义的图元。

参数去看 PPT。这里记不下来。

\hypertarget{header-n181}{%
\paragraph{Example}\label{header-n181}}

\begin{Shaded}
\begin{Highlighting}[]

\NormalTok{glClear(GL_COLOR_BUFFER_BIT);}

\NormalTok{glColor3f(}\FloatTok{0.2}\NormalTok{, }\FloatTok{0.7}\NormalTok{, }\FloatTok{0.7}\NormalTok{);}

\NormalTok{glPointSize(}\FloatTok{5.0}\NormalTok{);}

\NormalTok{glBegin(GL_POINTS);}

\NormalTok{glVertex2f(}\FloatTok{0.0}\NormalTok{, }\FloatTok{2.0}\NormalTok{);}

\NormalTok{...}

\NormalTok{glEnd();}

\NormalTok{glFlush();}
\end{Highlighting}
\end{Shaded}

P.S. 可以在 \texttt{glBegin()} 和 \texttt{glEnd()} 之间通过
\texttt{glColor3f} 来修改顶点的颜色。

不同的顶点颜色组装在一起会产生渐变颜色效果。这是双线性插值的结果。以后会讲到。

\hypertarget{header-n185}{%
\subsubsection{基本图元光栅化算法}\label{header-n185}}

最基本的事情。我们都很关心。

\hypertarget{header-n187}{%
\paragraph{啥是基本图元啊?}\label{header-n187}}

Primitive Symbols。什么算基本?

点、线、圆、一般函数曲线、字符。

\hypertarget{header-n190}{%
\paragraph{光栅化/扫描线算法}\label{header-n190}}

把原来的一个美好的几何图形给拍扁成一堆像素点。

这个过程就是光栅化/扫描线算法。

\hypertarget{header-n193}{%
\paragraph{Fragment:片元}\label{header-n193}}

光栅化完成之后,原来那个图元对应的像素点集合就叫片元区域。

\hypertarget{header-n195}{%
\paragraph{光栅化}\label{header-n195}}

\hypertarget{header-n196}{%
\subparagraph{点的光栅化}\label{header-n196}}

P(x0, y0) 怎么变呢?

\begin{quote}
法一:直接取整数部分。
\end{quote}

\begin{quote}
法二:四舍五入。
\end{quote}

这两者区别不太大,但是总的来说四舍五入的像素点偏移会更少。

如果说直接取整数部分,有可能会产生整体像左下角(0, 0 位置)的偏移。

但总归,点的光栅化都是容易的。

\hypertarget{header-n205}{%
\subparagraph{直线的光栅化}\label{header-n205}}

在数学中,直线的斜截式方程为 \(y = k x + b\)。

但事实上计算机不会处理无限长的直线。它只会处理有限的线段。

因此我们一般会提供起点和终点 P1(x1, y1) - P2(x2,
y2)。只画这两点构成的线段。

平凡的四种线段:\(y = k\),\(x = k\),\(x = y\),\(x = -y\)。这个光栅化过于简单,不多谈。

我们可以直接把它简化为点的光栅化问题。把这个线段给切分成一些点的集合(按照一定的切分粒度),然后对每一个点进行光栅化。

在斜率 k 绝对值小于 1 的时候就用变化的 x 来求 y。

在 k 绝对值大于 1 的时候就用变化的 y 来求 x。

但这个算法非常之慢。正常人类不会用的。

\hypertarget{header-n214}{%
\subparagraph{直线的光栅化(人类的方法)}\label{header-n214}}

DDA 算法:Digital Difference Algorithm。

这是一个增量计算法;也就是每一步计算都跟上一步的计算结果有关系。

我们知道起点和终点;先把起点给光栅化掉再说。

然后我们就有了另一个参量:当前绘制的点。

我们判断这个当前点是偏向了 y 轴(向上偏了)还是偏向了 x 轴(向下偏了)。

如果斜率偏小,那下一步就往右走;如果斜率偏大,那下一步就往上走。

如果不偏不倚;刚好一样,那就往右上方走。

这么办可以保证比较高的效率,同时 trivial 的四种情况也可以直接处理掉。

当然这种方法就仅仅适用于直线而已。其他的曲线是不可以用的。

附注:一般来说结尾点我们是不画的。因为经常会画连续的方法;

\hypertarget{header-n225}{%
\subparagraph{直线的光栅化(Bresenham 算法)}\label{header-n225}}

这个方法太快啦!所以它也被叫做「快速增量算法」。

基本思想是根据一个决策项 p 的正负,来选择像素点的位置。

本质上,需要在 x 方向上移动 A 格,同时在 y 方向上移动 B 格。

那么我们只需要决定在什么时候往右走,在什么时候往上走,就好了。

不需要进行那么多次的除法运算(说的就是 DDA),

思路是绝对的简单:我可以选择的像素点位置谁离真实值比较近,我就选谁走。

本算法所需要进行的计算仅仅包括「×2」和「-」。对应到位运算就是仅仅包括左移一位和减法器。

比起那些动不动要乘要除的浪费 CPU Cycle 的那些算法,这个算法简直太快了!

\hypertarget{header-n234}{%
\subparagraph{圆的光栅化}\label{header-n234}}

圆有二次多项式表示。

圆也有极坐标表示。

圆也有 Brosenham 栅格化算法。

看书 P110.

\end{document}
