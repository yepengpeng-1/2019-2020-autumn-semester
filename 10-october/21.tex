% Options for packages loaded elsewhere
\PassOptionsToPackage{unicode}{hyperref}
\PassOptionsToPackage{hyphens}{url}
%
\documentclass[
]{article}
\usepackage{lmodern}
\usepackage{amssymb,amsmath}
\usepackage{ifxetex,ifluatex}
\ifnum 0\ifxetex 1\fi\ifluatex 1\fi=0 % if pdftex
  \usepackage[T1]{fontenc}
  \usepackage[utf8]{inputenc}
  \usepackage{textcomp} % provide euro and other symbols
\else % if luatex or xetex
  \usepackage{unicode-math}
  \defaultfontfeatures{Scale=MatchLowercase}
  \defaultfontfeatures[\rmfamily]{Ligatures=TeX,Scale=1}
\fi
% Use upquote if available, for straight quotes in verbatim environments
\IfFileExists{upquote.sty}{\usepackage{upquote}}{}
\IfFileExists{microtype.sty}{% use microtype if available
  \usepackage[]{microtype}
  \UseMicrotypeSet[protrusion]{basicmath} % disable protrusion for tt fonts
}{}
\makeatletter
\@ifundefined{KOMAClassName}{% if non-KOMA class
  \IfFileExists{parskip.sty}{%
    \usepackage{parskip}
  }{% else
    \setlength{\parindent}{0pt}
    \setlength{\parskip}{6pt plus 2pt minus 1pt}}
}{% if KOMA class
  \KOMAoptions{parskip=half}}
\makeatother
\usepackage{xcolor}
\IfFileExists{xurl.sty}{\usepackage{xurl}}{} % add URL line breaks if available
\IfFileExists{bookmark.sty}{\usepackage{bookmark}}{\usepackage{hyperref}}
\hypersetup{
  hidelinks,
  pdfcreator={LaTeX via pandoc}}
\urlstyle{same} % disable monospaced font for URLs
\usepackage{graphicx,grffile}
\makeatletter
\def\maxwidth{\ifdim\Gin@nat@width>\linewidth\linewidth\else\Gin@nat@width\fi}
\def\maxheight{\ifdim\Gin@nat@height>\textheight\textheight\else\Gin@nat@height\fi}
\makeatother
% Scale images if necessary, so that they will not overflow the page
% margins by default, and it is still possible to overwrite the defaults
% using explicit options in \includegraphics[width, height, ...]{}
\setkeys{Gin}{width=\maxwidth,height=\maxheight,keepaspectratio}
% Set default figure placement to htbp
\makeatletter
\def\fps@figure{htbp}
\makeatother
\setlength{\emergencystretch}{3em} % prevent overfull lines
\providecommand{\tightlist}{%
  \setlength{\itemsep}{0pt}\setlength{\parskip}{0pt}}
\setcounter{secnumdepth}{-\maxdimen} % remove section numbering

\date{}

\begin{document}

\hypertarget{header-n0}{%
\section{Oct 21 Mon}\label{header-n0}}

\hypertarget{header-n2}{%
\subsection{SE-342::Computer Vision}\label{header-n2}}

今日头条:Mathematical Morphology and Binary
Operations。是谓「数学形态学及二值运算」。

在数学形态学里,也有一个跟卷积核很类似的东西:结构元(Structure
Elements)。

\hypertarget{header-n5}{%
\subsubsection{Dilation \& Erosion}\label{header-n5}}

也称作 SE,结构元,Kernel(但和卷积核不一样。)

通常来说结构元的结构是基本对称且位于中心的像素值为 1。

用结构元能检测出一些特定的基本形状。

\hypertarget{header-n9}{%
\paragraph{Basic Operations}\label{header-n9}}

我们最关心的是两个基本操作:「膨胀」和「腐蚀」。

留意:本节课中的所有说明都基于二值化。

\hypertarget{header-n12}{%
\paragraph{Dilation}\label{header-n12}}

膨胀,也称为闵可夫斯基和(Minkowski Addition)。

\hypertarget{header-n14}{%
\subparagraph{Definition}\label{header-n14}}

\textbf{膨胀(Dilation)}的定义为「位于某个点的探针(结构元素)是否\emph{有}探测到物件?」

\hypertarget{header-n16}{%
\subparagraph{Equations}\label{header-n16}}

\includegraphics{https://wikimedia.org/api/rest_v1/media/math/render/svg/147d9ba93f224385ffd71be0fdea6b6a42c7301c}。也写作\includegraphics{https://wikimedia.org/api/rest_v1/media/math/render/svg/dc6b3304ed6539f47a042c9eaa15209eb8df1ace}。

\hypertarget{header-n18}{%
\subparagraph{Descriptions}\label{header-n18}}

留意:需要先对 B 做映像(翻转,或者中心对称)。公式中可标识为
\(\hat{B}\),代表需要进行一个中心对称操作。

跟卷积里面先需要进行的操作一样。否则对于非中心对称的 SE
来说,结果是不对的。

\(A\) 为原始图像;\(B\) 为我们采用的 SE。首先把 \(B\)
绕着探针进行一个翻转;然后,将 SE
所提供「探针」在原图形的边缘上滑动时,如果探针所在位置和原图的交集不为空,则把此时探针点对应的像素位置设定为
1;否则将该像素值设定为 0。这些像素都会被放在 \(A ⊕ B\) 结果之中。

\hypertarget{header-n22}{%
\subparagraph{Rules}\label{header-n22}}

\begin{itemize}
\item
  \(D(A, B) = A ⊕ B = B ⊕ A = D(B, A)\) (交换律)
\item
  \((A ⊕ B) ⊕ C = A ⊕ (B ⊕ C)\) (结合律)
\item
  \(A ⊕ (B + x) = (A ⊕ B) + x\) (结合律 II)
\item
  \(nB = (B ⊕ B ⊕ B ⊕ B ... ⊕ B) (for\ n\ times)\)
\item
  \(A ⊕ (B ∪ C) = A ⊕ B ∪ A ⊕ C\)
\end{itemize}

\hypertarget{header-n34}{%
\paragraph{Erosion}\label{header-n34}}

Erosion 跟 Dilation 很相似;只是它比 Dilation 苛刻很多:

\hypertarget{header-n36}{%
\subparagraph{Definition}\label{header-n36}}

\textbf{腐蚀(Erosion)}的定义为「位于某个点的探针(结构元素)是否\emph{全都有}探测到物件?」

注意!此处 Erosion 不强调需要进行翻转。

\hypertarget{header-n39}{%
\subparagraph{Equations}\label{header-n39}}

\includegraphics{https://wikimedia.org/api/rest_v1/media/math/render/svg/e328d1a1cd6516a4e32458d58918974934256113}。另外,通常地,⊖
符号可以使用 \$ 符号代替。

\hypertarget{header-n41}{%
\subparagraph{Descriptions}\label{header-n41}}

将 SE
所提供「探针」在原图形的边缘上滑动时,如果探针所在位置和原图的交集\textbf{完全匹配}(该为
1 的地方都为 1),则把此时探针点对应的像素位置设定为
1;否则将该像素值设定为 0。这些像素都会被放在 \(A ⊖ B\) 结果之中。

因为设定 1 的条件很苛刻,因此一般来说 Erosion
得到的结果相比原图是缩水的。

\hypertarget{header-n44}{%
\subparagraph{Rules}\label{header-n44}}

\begin{itemize}
\item
  \(E(A, B) ≠ E(B, A)\) (不可交换律)
\item
  \((A ⊖ B) ⊖ C = A ⊖ (B ⊕ C)\) (谜样结合律)
\end{itemize}

\hypertarget{header-n50}{%
\paragraph{Warning}\label{header-n50}}

请留意,膨胀跟腐蚀并非可逆运算。他们是对偶操作,但是不可逆。

即,进行了腐蚀操作之后不一定能通过膨胀回到原图形,反过来也一样。即使他们用了相同的「SE」。

他们都造成了部分信息的丢失。

膨胀未必就能保留所有原图中的 1 点,当结构元原点并不为 1 时。

腐蚀也未必就不能包含原来不在原图中的 1 点,当结构元原点并不为 1 时。

不能简单理解成对原图像像素的额外增加和收缩。

\hypertarget{header-n57}{%
\paragraph{Quick Operations}\label{header-n57}}

\((A ⊖ B) ⊖ C = A ⊖ (B ⊕ C)\);

或者我们写作\(A ⊖ (B ⊕ C) = (A ⊖ B) ⊖ C\)。

因为从 \(A ⊖ (B ⊕ C)\) 到 \((A ⊖ B) ⊖ C\) 的转化我们更常用;

因为我们是快速地把结构元减小了;可以极大地提高速度。

\hypertarget{header-n62}{%
\paragraph{Dual-Operations}\label{header-n62}}

对偶操作

\(A ⊕ B = [A^C ⊖ (- B)]^C\)。

反过来也一样。

\hypertarget{header-n66}{%
\subsubsection{Opening \& Closing}\label{header-n66}}

开闭操作都是膨胀跟腐蚀操作的组合而已。

\hypertarget{header-n68}{%
\paragraph{Binary Opening}\label{header-n68}}

\hypertarget{header-n69}{%
\subparagraph{Definition}\label{header-n69}}

二元开操作定义为 \(A ∘ B = (A ⊖ B) ⊕ B\)。

\hypertarget{header-n71}{%
\subparagraph{Description}\label{header-n71}}

先腐蚀再膨胀。不可能比原来更大。

这也就印证了上面的:腐蚀跟膨胀不是可逆操作;否则这里的操作就没有意义了。

这种操作,等于是沿着图形内边缘滚动,比结构元更小的一些细枝末节区域都被开操作给去除掉了。

\hypertarget{header-n75}{%
\subparagraph{Rules}\label{header-n75}}

\begin{itemize}
\item
  \(O(A + x, B) = O(A, B) + x\)
\item
  \(O(A, B) ⊆ A\)
\item
  \((A ∘ B) ∘ B = A ∘ B\)
\end{itemize}

\begin{quote}
用相同的结构元多次进行开操作结果一样。
\end{quote}

\hypertarget{header-n85}{%
\paragraph{Binary Closing}\label{header-n85}}

二元闭操作嘛\ldots{}

\hypertarget{header-n87}{%
\subparagraph{Definition}\label{header-n87}}

猜的出来,\(A ∙ B = (A ⊕ B) ⊖ B\)。

\hypertarget{header-n89}{%
\subparagraph{Description}\label{header-n89}}

实心圆圈:先膨胀再腐蚀。不可能比原图形小。

这种操作可以填充那些小于结构元的孔洞和尖锐拐角。

\hypertarget{header-n92}{%
\subparagraph{Rules}\label{header-n92}}

\begin{itemize}
\item
  \(C(A + x, B) = C(A, B) + x\)
\end{itemize}

\hypertarget{header-n96}{%
\paragraph{Dual-Operations}\label{header-n96}}

D 跟 C 也是对偶的操作。很自然的。

\(A ∘ B = [A^C ∙ (- B)]^C\)。

反过来一样。

\hypertarget{header-n100}{%
\subsubsection{Hit \textbar{} Miss}\label{header-n100}}

上面的四种运算都是基于一个简单的断言:命中/不命中。这是一种很好的图形检测算法啊。

\hypertarget{header-n102}{%
\subsubsection{Pattern Spectrum}\label{header-n102}}

\begin{enumerate}
\def\labelenumi{\arabic{enumi}.}
\item
  进行一个开操作。
\item
  不停地使用不同大小的 SE 进行开操作并对其进行做减;叫做 Distance Size
  Transformation(DST)。
\item
  在二值化的图形中;计算对象数量。
\item
  不停重复进行开操作和减操作,得到不同尺寸大小结构元的出现数目。
\item
  在二维图像为空时停止。此时腐蚀完了。
\end{enumerate}

\hypertarget{header-n114}{%
\subsubsection{Recursive Dilation}\label{header-n114}}

「反复膨胀」

\end{document}
