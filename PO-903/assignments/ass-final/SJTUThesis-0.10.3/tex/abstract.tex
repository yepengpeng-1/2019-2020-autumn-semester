%# -*- coding: utf-8-unix -*-
% !TEX program = xelatex
% !TEX root = ../thesis.tex
% !TEX encoding = UTF-8 Unicode
%%==================================================
%% abstract.tex for SJTU Master Thesis
%%==================================================

\begin{abstract}

作为人类史上最年轻的能源利用形式,核能自从一出生就带着浓重的阴影。
无论是他和具有巨大破坏力的原子弹的同系出身,还是 40 年内两次特别严重级别的核事故,都让人们记忆深刻、不敢忘却。
然而其极高的能量密度和无一般污染物的特性,目前仍然是人类最优质,也是最有潜力的的发展能源。

遗憾的是,面对着人民对「核」的恐惧与质疑,许多国家纷纷放弃了进一步的核电发展计划,有的反应堆甚至被提前退役;这些决定又反过来加深了人民的恐惧。
面对舆论的压力和事实上生存空间的减小,应用核能从业者进入了一波寒冬,而理论核能从业者也受到波及。
核能未来将走向何方?

本文中主要讨论了核能目前受到杯葛的现状、这种事实背后的原因、以及核能未来可能的发展方向。

\end{abstract}

\begin{englishabstract}

\end{englishabstract}

