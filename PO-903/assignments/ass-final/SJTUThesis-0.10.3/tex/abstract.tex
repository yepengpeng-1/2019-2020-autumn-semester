%# -*- coding: utf-8-unix -*-
% !TEX program = xelatex
% !TEX root = ../thesis.tex
% !TEX encoding = UTF-8 Unicode
%%==================================================
%% abstract.tex for SJTU Master Thesis
%%==================================================

\begin{abstract}

作为人类史上最年轻的能源利用形式,核能自从一出生就带着浓重的阴影。
无论是他和具有巨大破坏力的原子弹的同系出身,还是 40 年内两次特别严重级别的核事故,都让人们记忆深刻、不敢忘却。
然而其极高的能量密度和无一般污染物的特性,目前仍然是人类最优质,也是最有潜力的的发展能源。

遗憾的是,面对着人民对「核」的恐惧与质疑,许多国家纷纷放弃了进一步的核电发展计划,有的反应堆甚至被提前退役;这些决定又反过来加深了人民的恐惧。
面对舆论的压力和事实上生存空间的减小,应用核能从业者进入了一波寒冬,而理论核能从业者也受到波及。
核能未来将走向何方?

本文中主要讨论了核能目前受到杯葛的现状、这种事实背后的原因、以及核能未来可能的发展方向。

\end{abstract}

\begin{englishabstract}
As the youngest form of energy use in human history, nuclear power has been under heavy shadow since its birth.

Born in the same family as the hugely destructive atomic bomb, 
and two nuclear accidents of a particularly severe level within 40 years, 
people have deep memory in their mind.

However, its extremely high energy density and the characteristics of no general pollutants 
are still the highest quality and most promising energy for human development.
    
Unfortunately, in the face of people’s fears and doubts about “nuclear”, 
many countries have abandoned further nuclear power development plans, 
and some reactors have even been decommissioned earlier; 
these decisions have in turn deepened people ’s fears.

Facing the pressure of public opinion and the fact that the living space has decreased, 
the applied nuclear energy practitioners have entered a wave of cold winters, 
and the theoretical nuclear energy practitioners have also been affected.

What will nuclear energy be in the future?
    
This article mainly discusses the current status of nuclear energy, 
the reasons behind this fact, 
and the possible future development of nuclear energy.

\end{englishabstract}

