%# -*- coding: utf-8-unix -*-
% !TEX program = xelatex
% !TEX root = ../thesis.tex
% !TEX encoding = UTF-8 Unicode
%%==================================================
%% conclusion.tex for SJTUThesis
%% Encoding: UTF-8
%%==================================================

\begin{summary}

在选修这门课程之前,我对于清洁能源的知识是零碎的,基本全部来源于物理课和化学课中所讲到的知识。

其中关于原电池、原子物理和高分子合成的知识都十分零碎,且这些知识都相对较旧,没有结合最新的科研成果。

而这门课的起点相对较低,在讲解每个知识点之前都会花一些时间来了解隐藏在这类能源背后的基本原理;
在有一定的中学基础的情况下相对容易理解,也能更好的理解后续进一步的讲解。

同时,在课程总体的后一部分还引入了小组报告机制,使得每位同学都有机会深入了解某一样清洁能源,
并将自己所深入了解的部分和所有人一同分享,实现了学习效率的最大化。

选修「清洁能源与技术:原理与应用」的过程,我收获很多。非常感谢老师和助教。

\end{summary}
