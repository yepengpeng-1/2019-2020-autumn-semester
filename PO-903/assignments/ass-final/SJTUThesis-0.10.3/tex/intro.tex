%# -*- coding: utf-8-unix -*-
% !TEX program = xelatex
% !TEX root = ../thesis.tex
% !TEX encoding = UTF-8 Unicode
%%==================================================
%% chapter01.tex for SJTU Master Thesis
%%==================================================

\chapter{历史与现状}

目前人类所能利用在发电上的核能,均基于上世纪 30 年代发现的核裂变技术。
然而,这一基于重核铀的裂变方式的反应物、生成物和副产物都是具有极强辐射性的污染物。
其中的反应终结物质不但无法进行二次利用,且仍具有极强的污染力和破坏力,因此被俗称为「核废料」。

正是这种贯穿核能发电全过程的核能风险,使得核事故一旦发生,必然伴随着对周边环境和生态圈的极大污染。
近代连续三次的严重核事故:三里岛、切尔诺贝利、以及福岛,则见证了世界各国对于核能心态的转变。

\section{恐惧起源}
对于核能的恐惧或许可以追溯到上世纪末的切尔诺贝利核电事故期间。

此时正值冷战高峰,以美国为首的西方阵营抓住这场发生在苏联的事故作为把柄,将其作为国内政治宣传的武器,
在极力渲染反苏反共气氛的同时,也在民众的内心中埋下了对核电站乃至核能的不理性的恐惧感。

这种情绪并没有随着东欧剧变和苏联的解体而告终;
相反,这种恐核、反核的情绪逐渐内化为一个符号,被放置在政治正确的环境保护、推动绿色能源的对立面。
这种民意裹挟着澳大利亞、瑞典、和意大利纷纷在 1978 年、1980 年、和 1987 年进行了核电站建造问题的全民公投\cite{wiki1}。
其中意大利更是在 2011 年以 94\% 的反对率封杀了重启核电的提案\cite{itgov}。

可以说,近四十年来对于核电的恐惧一直在各国蔓延,甚至逐渐固化。

\section{近年现状}

台湾地区的民进党政权在 2008 年惨败下野之后,为了重新执政,提出了一系列前瞻计划,其中一项关键的政见就是「2025非核家园计划」\cite{iwt}。
这一政见要求在 2025 年之前将核能发电从台湾能源结构中完全剔除。
更确切一点说,在不新建任何核电厂的前提下,令核一、核二与核三厂按时退役,且停止核四厂的商业运转。

然而这项政见最终仅仅停留在选举口号阶段;2018 年的全民公投否决了这项提案。
但从中也可以看出核电的定位之尴尬:它被强行放在了清洁、绿色、安全等民众喜闻乐见的「进步特征」的对立面上,被作为各国各政党屡试不爽的政治操作手段。

\section{雪上加霜}

2011 年福岛核电站事故无疑再一次让废核派多了一把有力的武器,刺激了所有对核电抱有怀疑的民众。
事故后直到现在,海洋污染、海产品污染、核废料以及废弃核电站的善后工作等等都还是个无底洞。

此事一出,即使是对核电发展十分热衷的中国也因此收紧了核电站的审批,其他本身就偏保守的国家更不必说。
鉴于这一核事故的善后工作很可能像切尔诺贝利一样持续数十年,这一次核电的低潮很可能不会在短期内结束。

\chapter{成因分析}

\section{思潮源头}

最早的核事故受害者应该是在切尔诺贝利事故后被疏散的东欧居民。
苏联在其东欧加盟国建立的这一核电站引发的事故给当地人民带来了深重的灾难。

加上东欧国家的国土面积基本都不大,一旦发生核事故,扩散范围极可能覆盖全境。
显然国土狭窄的国家和地区对于核废料的处理更为担忧。

在冷战结束后,美国因为自身经济问题以及自身能源结构已经相对完备、能源需求并不高,因此也减缓了现有核电站的建设进程。

\section{保守蔓延}

这种保守主义的势头逐渐蔓延到了其他地方。
历史上,国土狭窄的国家发展其他形式的清洁能源难度极大,而很多时候发展高能量密度的核电反而最为有效。

这也就说明了为何类似于日本、韩国、台湾地区等地都相对较早地建造了核能发电的系统,因为对于这类地方高能量密度的核电可能是发展传统能源之外的唯一选择。

然而这类地域狭窄的地区一般很难找到合理的办法来处理核废料:这就为反核的冲突埋下了伏笔。

1987 年前后,台湾国民党当局将核电厂产生的低阶核废料放置在兰屿附近,引起了当地原住民的抗议和反对\cite{hsl}。
时至今日,这种抗议从最初的单纯反对接受核废料,逐渐结合了对原住民弱势地位的不满、对当局机构先斩后奏的专制作风的不满,发展为一种特殊的社会思潮。

这也就是为何以蔡英文为首的民主进步党在竞选政见发表中,为了争取法统的正当性、以及迎合民众的「废核」、「恐核」心态,提出了所谓的「2025 非核家园」政策\cite{hemx1}。

\section{政治正确}

基于上面的历史原因,大部分目前推出非核政策的国家集中在欧洲。
而欧洲基于其高福利、极稳定的特殊社会形态令大部分人民向往;
而遗憾的是欧洲这一具有一定历史原因的废核社会思潮也被照单全收,成为了政治正确的一部分\cite{hemx2}。

