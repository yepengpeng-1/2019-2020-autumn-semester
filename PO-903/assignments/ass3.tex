% Options for packages loaded elsewhere
\PassOptionsToPackage{unicode}{hyperref}
\PassOptionsToPackage{hyphens}{url}
%
\documentclass[
]{article}
\usepackage{lmodern}
\usepackage{amssymb,amsmath}
\usepackage{ifxetex,ifluatex}
\ifnum 0\ifxetex 1\fi\ifluatex 1\fi=0 % if pdftex
  \usepackage[T1]{fontenc}
  \usepackage[utf8]{inputenc}
  \usepackage{textcomp} % provide euro and other symbols
\else % if luatex or xetex
  \usepackage{unicode-math}
  \defaultfontfeatures{Scale=MatchLowercase}
  \defaultfontfeatures[\rmfamily]{Ligatures=TeX,Scale=1}
\fi
% Use upquote if available, for straight quotes in verbatim environments
\IfFileExists{upquote.sty}{\usepackage{upquote}}{}
\IfFileExists{microtype.sty}{% use microtype if available
  \usepackage[]{microtype}
  \UseMicrotypeSet[protrusion]{basicmath} % disable protrusion for tt fonts
}{}
\makeatletter
\@ifundefined{KOMAClassName}{% if non-KOMA class
  \IfFileExists{parskip.sty}{%
    \usepackage{parskip}
  }{% else
    \setlength{\parindent}{0pt}
    \setlength{\parskip}{6pt plus 2pt minus 1pt}}
}{% if KOMA class
  \KOMAoptions{parskip=half}}
\makeatother
\usepackage{xcolor}
\IfFileExists{xurl.sty}{\usepackage{xurl}}{} % add URL line breaks if available
\IfFileExists{bookmark.sty}{\usepackage{bookmark}}{\usepackage{hyperref}}
\hypersetup{
  hidelinks,
  pdfcreator={LaTeX via pandoc}}
\urlstyle{same} % disable monospaced font for URLs
\setlength{\emergencystretch}{3em} % prevent overfull lines
\providecommand{\tightlist}{%
  \setlength{\itemsep}{0pt}\setlength{\parskip}{0pt}}
\setcounter{secnumdepth}{-\maxdimen} % remove section numbering

\date{}

\begin{document}

\hypertarget{header-n0}{%
\section{PO-903}\label{header-n0}}

\hypertarget{header-n2}{%
\subsection{思考题 3}\label{header-n2}}

\hypertarget{header-n4}{%
\paragraph{Q1:
请分别解释干电池、铅酸电池和锂离子电池的工作原理。}\label{header-n4}}

\hypertarget{header-n11}{%
\subparagraph{干电池}\label{header-n11}}

干电池(Dry Cell)是一种以糊状电解液来产生直流电的化学电池。

最常见的干电池「碳锌电池」(Dry Leclanché Cell)在一个锌罐中存放
NH\textsubscript{4}Cl 和 ZnCl\textsubscript{2}
混合糊状液体,通过一层纸及粉末状的碳及 MnO\textsubscript{2} 隔开。

此时,NH\textsubscript{4}\textsuperscript{+} 离子提供了整体环境中的
H\textsuperscript{+} 离子;

在阳极,Zn 被氧化产生 Zn\textsuperscript{2+},同时产生两个电子;

同时在阴极,二氧化锰和 NH\textsubscript{4}\textsuperscript{+}
产生的氢离子得到电子,被还原成 Mn\textsubscript{2}O\textsubscript{3}。

阳极反应: Zn(\emph{s}) → Zn\textsuperscript{2+}(\emph{aq}) + 2
e\textsuperscript{−}

阴极反应:2MnO\textsubscript{2}(\emph{s}) + 2
H\textsuperscript{+}(\emph{aq}) + 2 e\textsuperscript{−} →
Mn\textsubscript{2}O\textsubscript{3}(\emph{s}) +
H\textsubscript{2}O(\emph{l})

整个反应中,Zn 被氧化为正二价;而 MnO\textsubscript{2} 中的正四价的 Mn
则被还原为 Mn\textsubscript{2}O\textsubscript{3} 中的正三价。

同时产生的 NH\textsubscript{3} 和 Zn\textsuperscript{2+}
离子还会发生一个副反应:

Zn\textsuperscript{2+}(\emph{aq}) + 2 NH\textsubscript{3}(\emph{aq}) +
2Cl\textsuperscript{\^{}−}\^{}(\emph{aq}) →
Zn(NH\textsubscript{3})\textsubscript{2}Cl\textsubscript{2}(\emph{s})

碳锌电池的电动势大约是 \(1.54V\)
。电动势的不确定是由于阴极反应十分复杂,相比来说,阳极反应(锌端)则有一个已知的电势。

而副反应及活性反应物的消耗直接导致电池的内阻增加,电池电动势降低。

\hypertarget{header-n35}{%
\subparagraph{铅酸电池}\label{header-n35}}

铅酸蓄电池,又称铅蓄电池,是蓄电池的一种,电极主要由铅制成,电解液是硫酸溶液的一种蓄电池。

和上面提到的碳锌干电池不同,铅酸电池是可充电电池,因此分充电和放电两部分进行分析。

放电时

整个环境中存在着硫酸电离出的高浓度的 H\textsuperscript{+} 离子和
SO\textsubscript{4}\textsuperscript{2−} 离子。

负极上的 Pb 单质被氧化,和 SO\textsubscript{4}\textsuperscript{2−}
离子生成固体 PbSO\textsubscript{4},同时失去两个电子。

正极上的 PbO\textsubscript{2} 固体被还原,和 H\textsuperscript{+} 离子和
SO\textsubscript{4}\textsuperscript{2−} 离子一同生成固体
PbSO\textsubscript{4},同时得到两个电子。

化学方程式写作:

负极反应:\({\displaystyle {\rm {Pb+SO_{4}^{2-}\rightarrow PbSO_{4}+2e^{-}}}}\)

正极反应:\({\displaystyle {\rm {PbO_{2}+4H^{+}+SO_{4}^{2-}+2e^{-}\rightarrow 2H_{2}O+PbSO_{4}}}}\)

总反应可以写作:\({\rm {PbO_{{2(s)}}+Pb_{{(s)}}+2{H_{2}SO_{4}}_{{(aq)}}\rightarrow 2{PbSO_{4}}_{{(s)}}+2{H_{2}O}_{{(l)}}}}\)

充电时

充电时的反应基本是放电时反应的逆反应。硫酸铅和水转化为二氧化铅、海绵状铅与稀硫酸。

总反应为\({\rm {2{PbSO_{4}}_{{(s)}}+2{H_{2}O}_{{(l)}}\rightarrow PbO_{{2(s)}}+Pb_{{(s)}}+2{H_{2}SO_{4}}_{{(aq)}}}}\)。

应用

铅酸电池寿命大约 2 到 4 年,不过长期处于低电量的铅酸电池寿命会缩短。

铅酸电池最常见故障是硫酸铅结晶过多。

\hypertarget{header-n96}{%
\subparagraph{锂离子电池}\label{header-n96}}

锂离子电池(Lithium-ion
battery)是一种充电电池,它主要依靠锂离子在正极和负极之间移动来工作。

和铅酸电池一样,这也是一种可充电式电池。

正极上发生的反应为
\({\displaystyle \mathrm {Li} _{1-x}\mathrm {CoO_{2}} +x\mathrm {Li^{+}} +x\mathrm {e^{-}} \leftrightarrows \mathrm {LiCoO_{2}} }\)。

负极上发生的反应为
\({\displaystyle x\mathrm {LiC_{6}} \leftrightarrows \ x\mathrm {Li^{+}} +x\mathrm {e^{-}} +x\mathrm {C_{6}} }\)。

现实中由于过度放电时产生的锂钴氧化物和 Li\textsuperscript{+} 产生的
CoO,电池会产生一些不可逆的损耗。

化学式为:\({\mathrm  {Li^{+}}}+{\mathrm  {e^{-}}}+{\mathrm  {LiCoO_{2}}}\rightarrow {\mathrm  {Li_{2}O}}+{\mathrm  {CoO}}\)。

\hypertarget{header-n119}{%
\paragraph{Q2:
请说明锂离子电池隔膜的种类、性能要求及其制备方法。}\label{header-n119}}

性能要求

隔膜(Separator)的主要意义时将两个电极分开,仅仅让需要进行交换的粒子穿过,而不让他们直接接触,防止他们产生类似于「短路」的事故。

由于目前的锂离子电池中,电解液多为有机溶剂体系,因而需要有耐有机溶剂的隔膜材料。实际生产中一般采用高强度薄膜化的聚烯烃多孔膜。

主要的要求包括:

\begin{enumerate}
\def\labelenumi{\arabic{enumi}.}
\item
  电子绝缘,否则会产生直接短路;
\item
  良好的 Li\textsuperscript{2+} 透过率;
\item
  耐电解质腐蚀。
\end{enumerate}

另外,对于移动电子设备而言,安全性也很重要。目前大部分的隔膜可以实现「shutdown
separator」,即在设备温度过高时,直接关闭离子通道以保证设备安全。

种类及制备方法

根据不同的物理、化学特性,锂电池隔膜材料可以分为:

\begin{itemize}
\item
  织造膜
\item
  非织造膜(无纺布)
\item
  微孔膜
\item
  复合膜
\item
  隔膜纸
\item
  碾压膜
\end{itemize}

聚烯烃材料具有优异的力学性能、化学稳定性和相对廉价的特点,因此聚乙烯、聚丙烯等聚烯烃微孔膜在锂电池研究开发初期便被用作锂电池隔膜。

也有用其他材料制备锂电池隔膜的研究,如 1999 年 F. Boudin
等采用相转化法以聚偏氟乙烯(PVDF)为本体聚合物制备锂电池隔膜;Kuribayash
Isao等研究纤维素复合膜作为锂电池隔膜材料。

然而,至今商品化锂电池隔膜材料仍主要采用聚乙烯、聚丙烯微孔膜。

\end{document}
